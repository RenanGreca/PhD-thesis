%----------------------------------------------------------------------------------------
%	PACKAGES AND OTHER DOCUMENT CONFIGURATIONS
%----------------------------------------------------------------------------------------

\documentclass[11pt, a4paper, twoside]{Thesis} % Paper size, default font size and one-sided paper

\graphicspath{{./Pictures/}} % Specifies the directory where pictures are stored

%\usepackage[square, numbers, comma, sort&compress]{natbib} % Use the natbib reference package - read up on this to edit the reference style; if you want text (e.g. Smith et al., 2012) for the in-text references (instead of numbers), remove 'numbers' 
\usepackage[natbib=true, maxbibnames=99, maxcitenames=2, giveninits=true, sorting=none]{biblatex}

\hypersetup{urlcolor=black, colorlinks=true} % Colors hyperlinks in blue - change to black if annoying
\title{\ttitle} % Defines the thesis title - don't touch this

\usepackage{tikz}
\usepackage{setspace}

\definecolor{OrangeGSSI}{RGB}{237,113,45}
\definecolor{blue}{RGB}{25,25,112}

%----------------------------------------------------------------------------------------
%	DOCUMENT VARIABLES
%	Fill in the lines below to update the thesis template
%	If you wish to cite each of the variables defined below, look at the
%	section above for the citation command e.g. \examiner{} below is
%	defined as \examname above so you cite it as \examname
%----------------------------------------------------------------------------------------

\title{Orchestration Strategies for Regression Testing of Evolving Software Systems}

\thesistitle{Orchestration Strategies for Regression Testing of Evolving Software Systems} % Your thesis title - this is used in the title and abstract
%-------------------------------------------------  
\supervisor{Prof. Antonia \textsc{Bertolino}} % You supervisor's name - this is used in the title page
%-------------------------------------------------   
\examiner{Dr. Ludovico  \textsc{Iovino}} % Your examiner's name - this is not currently used anywhere in the template, cite it with \examname if you want it
%-------------------------------------------------   
\degree{Doctor of Philosophy} % Your degree name - this is currently used in the title page and abstract
%-------------------------------------------------   
\authors{Renan Domingos\\ \textsc{Merlin Greca}} % Your name - this is used in the title page and abstract
%-------------------------------------------------

%----------------------------------------------------------------------------------------
%	TYPESETTING
%----------------------------------------------------------------------------------------
\usepackage{pifont}
\usepackage{colortbl}
\usepackage{flushend}
\usepackage[normalem]{ulem} % for \sout
\usepackage{multicol}
\usepackage[most]{tcolorbox}
\usepackage[nameinlink]{cleveref}
\usepackage{algorithm}
\usepackage{algorithmicx}
\usepackage[noend]{algpseudocode}
%\usepackage[english]{babel}
\usepackage{hyphenat}
%\usepackage[resetlabels,labeled]{multibib}
\usepackage{multirow}
%\usepackage{bibentry}
%\nobibliography*
%\usepackage[left=1.4cm, right=1cm, top=0.6cm, bottom=0.8cm, paper=a4paper]{geometry}

%\usepackage[left=1cm,right=2cm,vmargin=2.5cm,footnotesep=0.5cm]{geometry}
%
%\setmarginsrb  { 1.4in}  % left margin
%                        { 0.6in}  % top margin
%                        { 1.0in}  % right margin
%                        { 0.8in}  % bottom margin
%                        {  20pt}  % head height
%                        {0.25in}  % head sep
%                        {   9pt}  % foot height
%                        { 0.3in}  % foot sep

%\hyphenation{non-de-ter-mi-nis-ti-cal-ly}

\newcommand{\ugh}[1]{\textcolor{red}{\uwave{#1}}} % please rephrase
\newcommand{\ins}[1]{\textcolor{blue}{\uline{#1}}} % please insert
\newcommand{\del}[1]{\textcolor{red}{\sout{#1}}} % please delete
\newcommand{\chg}[2]{\textcolor{red}{\sout{#1}}{\ra}\textcolor{blue}{\uline{#2}}} % please change

\renewcommand{\ttdefault}{cmr}

%% Abbreviation commands
\newcommand{\tcp}{TCP\xspace}
\newcommand{\tcs}{TCS\xspace}
\newcommand{\tsr}{TSR\xspace}
\newcommand{\tsa}{TSA\xspace}
\newcommand{\rt}{RT\xspace}
\newcommand{\rea}{IR\&A\xspace}
\newcommand{\sut}{SUT\xspace}
\newcommand{\slr}{SLR\xspace}

\newcommand{\ek}{Ekstazi\xspace}
\newcommand{\ekr}{Ekstazi\xspace}
\newcommand{\rnd}{Random\xspace}
\newcommand{\fs}{FAST\xspace}
\newcommand{\fz}{{Fastazi}\xspace}
\newcommand{\fzs}{\fz-S\xspace}
\newcommand{\fzp}{\fz-P\xspace}
\newcommand{\dfj}{Defects4J\xspace}

\newcommand{\apfd}{APFD\xspace}
\newcommand{\ttff}{TTFF\xspace}
\newcommand{\pttff}{TTFF\xspace}

\newcommand{\rotatedheader}[2] {
	\rotatebox[origin=c]{90}{\textbf{{\color{#1} #2}}}
}

\newcommand{\quoter}[2]{``\textit{#1}'' (#2)}
\newcommand{\quotes}[1]{``\textit{#1}''}

\newcommand{\assignedto}[1]{\textcolor{red}{\ding{46}~{\sf Assigned to:}~#1}\\}
\newcommand{\todo}[1]{\textcolor{blue}{\ding{46}{\sf}~#1}}
\newcommand{\sugg}[1]{\textcolor{orange}{#1}}

\newcommand{\anto}[1]{[[\textbf{Anto:}~{\color{olive}#1}]]}
\newcommand{\breno}[1]{[[\textbf{Breno:}~{\color{teal}#1}]]}
\newcommand{\renan}[1]{[[\textbf{Renan:}~{\color{brown}#1}]]}


\definecolor{bronze}{HTML}{666211}
\definecolor{cyprus}{HTML}{006666}
\definecolor{derby}{HTML}{664422}
\definecolor{diesel}{HTML}{660122}
\definecolor{verdun}{HTML}{366600}
\definecolor{midnight}{HTML}{002766}
\definecolor{bossanova}{HTML}{642B66}

% Circles
\usepackage{tikz}
\newcommand*\emptycirc[1][1ex]{%
	\begin{tikzpicture}
		\draw (0,0) circle (#1);
	\end{tikzpicture}} 
\newcommand*\halfcirc[1][1ex]{%
  \begin{tikzpicture}
  \draw[fill] (0,0)-- (90:#1) arc (90:270:#1) -- cycle ;
  \draw (0,0) circle (#1);
  \end{tikzpicture}}
\newcommand*\fullcirc[1][1ex]{%
  \begin{tikzpicture}
	\draw[fill] (0,0) circle (#1);  	
  \end{tikzpicture}} 
  
%\geometry{%
%    paper = a4paper,%
%    top = 0.6in,%
%    left = 1.4in,%
%    right = 1.0 in, %
%    bottom = 0.8in,%         <---- Changing this has no effect
%    headsep = 0.25in%
%}

  
% Defining bibliographies
%\newcites{P}{Papers produced during the development of this thesis}
%\newcites{C}{Conferences related to this thesis}
%\newcites{S}{Papers identified in the Systematic Literature Review}
%\newcites{Bibliography}{Bibliography}

\defbibheading{P}{}
\defbibheading{S}{Papers identified in the Systematic Literature Review}
\defbibheading{R}{Bibliography}

\addbibresource[label=P]{production.bib}
\addbibresource[label=S]{slr.bib}
\bibliography{bibliography.bib, secondaries.bib}


\newcommand{\numpapers}{79\xspace}
\defcitealias{srikanth_requirements_2016}{S1}
\defcitealias{noor_similarity-based_2016}{S2}
\defcitealias{schwartz_cost-effective_2016}{S3}
\defcitealias{hirzel_graph-walk-based_2016}{S4}
\defcitealias{lu_how_2016}{S5}
\defcitealias{vost_trace-based_2016}{S6}
\defcitealias{wang_enhancing_2016}{S7}
\defcitealias{srikanth_test_2016}{S8}
\defcitealias{blondeau_test_2017}{S9}
\defcitealias{pradhan_search-based_2016}{S10}
\defcitealias{buchgeher_improving_2016}{S11}
\defcitealias{tahvili_dynamic_2016}{S12}
\defcitealias{oqvist_extraction-based_2016}{S13}
\defcitealias{magalhaes_automatic_2016}{S14}
\defcitealias{aman_application_2016}{S15}
\defcitealias{busjaeger_learning_2016}{S16}
\defcitealias{yoshida_fsx_2016}{S17}
\defcitealias{tahvili_cost-benefit_2016}{S18}
\defcitealias{ramler_tool_2017}{S19}
\defcitealias{strandberg_experience_2016}{S20}
\defcitealias{marijan_effect_2016}{S21}
\defcitealias{gotlieb_using_2017}{S22}
\defcitealias{chi_multi-level_2017}{S23}
\defcitealias{bach_coverage-based_2017}{S24}
\defcitealias{spieker_reinforcement_2017}{S25}
\defcitealias{vasic_file-level_2017}{S26}
\defcitealias{celik_regression_2017}{S27}
\defcitealias{ouriques_test_2018}{S28}
\defcitealias{kwon_cost-effective_2017}{S29}
\defcitealias{garousi_multi-objective_2018}{S30}
\defcitealias{shi_evaluating_2018}{S31}
\defcitealias{haghighatkhah_test_2018}{S32}
\defcitealias{zhang_hybrid_2018}{S33}
\defcitealias{miranda_fast_2018}{S34}
\defcitealias{yilmaz_case_2018}{S35}
\defcitealias{chen_optimizing_2018}{S36}
\defcitealias{celik_regression_2018}{S37}
\defcitealias{zhu_test_2018}{S38}
\defcitealias{azizi_retest_2018}{S39}
\defcitealias{guo_decomposing_2019}{S40}
\defcitealias{zhong_testsage:_2019}{S41}
\defcitealias{fu_resurgence_2019}{S42}
\defcitealias{eda_efficient_2019}{S43}
\defcitealias{goyal_test_2019}{S44}
\defcitealias{yu_terminator_2019}{S45}
\defcitealias{correia_motsd_2019}{S46}
\defcitealias{machalica_predictive_2018}{S47}
\defcitealias{najafi_improving_2019}{S48}
\defcitealias{leong_assessing_2019}{S49}
\defcitealias{cruciani_scalable_2019}{S50}
\defcitealias{philip_fastlane:_2019}{S51}
\defcitealias{magalhaes_hsp_2020}{S52}
\defcitealias{wu_time_2019}{S53}
\defcitealias{land_industrial_2019}{S54}
\defcitealias{noemmer_evaluation_2020}{S55}
\defcitealias{lubke_selecting_2020}{S56}
\defcitealias{yackley_simultaneous_2019}{S57}
\defcitealias{shi_understanding_2019}{S58}
\defcitealias{lima_multi-armed_2022}{S59}
\defcitealias{zhou_beating_2020}{S60}
\defcitealias{peng_empirically_2020}{S61}
\defcitealias{bertolino_learning--rank_2020}{S62}
\defcitealias{chen_multi-objective_2021}{S63}
\defcitealias{zarges_artificial_2021}{S64}
\defcitealias{bagherzadeh_reinforcement_2022}{S65}
\defcitealias{elsner_empirically_2021}{S66}
\defcitealias{pan_dynamic_2020}{S67}
\defcitealias{mehta_data-driven_2021}{S68}
\defcitealias{xu_requirement-based_2021}{S69}
\defcitealias{zhou_parallel_2022}{S70}
\defcitealias{sharif_deeporder_2021}{S71}
\defcitealias{li_aga_2021}{S72}
\defcitealias{chen_context-aware_2021}{S73}
\defcitealias{zhang_comparing_2022}{S74}
\defcitealias{abdelkarim_tcp-net_2022}{S75}
\defcitealias{cingil_black-box_2022}{S76}
\defcitealias{yaraghi_scalable_2022}{S77}
\defcitealias{omri_learning_2022}{S78}
\defcitealias{greca_comparing_2022}{S79}
  

%----------------------------------------------------------------------------------------
%	End of TYPESETTING
%----------------------------------------------------------------------------------------

\begin{document}

\frontmatter % Use roman page numbering style (i, ii, iii, iv...) for the pre-content pages

\setstretch{1.3} % Line spacing of 1.3

% Define the page headers using the FancyHdr package and set up for one-sided printing
\fancyhead{} % Clears all page headers and footers
\rhead{\thepage} % Sets the right side header to show the page number
\lhead{} % Clears the left side page header

% \pagestyle{fancy} % Finally, use the "fancy" page style to implement the FancyHdr headers
\newcommand{\HRule}{\rule{\linewidth}{0.5mm}} % New command to make the lines in the title page



% PDF meta-data
\hypersetup{pdftitle={\ttitle}}
\hypersetup{pdfsubject=\subjectname}
\hypersetup{pdfauthor=\authornames}
\hypersetup{pdfkeywords=\keywordnames}

%----------------------------------------------------------------------------------------
%	TITLE PAGE
%----------------------------------------------------------------------------------------

%\newgeometry{left=1in}

\begin{titlepage}
\begin{center}

\includegraphics[width=0.4\textwidth]{./figures/logo_GSSI}~\\[1cm]
\textsc{\Large Doctoral Thesis}\\[0.5cm] % Thesis type

\HRule \\[0.1cm] % Horizontal line
{\huge \bfseries  Orchestration Strategies for Regression\\[0.3cm] Testing of Evolving Software Systems }\\[0.3cm] % Thesis title
\HRule \\[0.9cm] % Horizontal line

{\Large \textsc{PhD Program in Computer Science: 34th cycle}}\\[2cm]

\begin{minipage}{0.4\textwidth}
\begin{flushleft} \large
\emph{Author:}\\
\bigskip \authornames \\
\href{mailto:renan.greca@gssi.it}{renan.greca@gssi.it}
%\href{http://www.johnsmith.com}{\authornames} % Author name - remove the \href bracket to remove the link
\end{flushleft}
\end{minipage}
\begin{minipage}{0.5\textwidth}
\begin{flushright} \large
\emph{Supervisors:} \\
%\href{http://www.jamessmith.com}{\supname} % Supervisor name - remove the \href bracket to remove the link  
\bigskip \supname \\
\href{mailto:antonia.bertolino@isti.cnr.it}{antonia.bertolino@isti.cnr.it} \\
Prof. Breno \textsc{Miranda}\\
\href{mailto:bafm@cin.ufpe.br}{bafm@cin.ufpe.br} \\
\bigskip \bigskip
\emph{Internal advisor:} \\
\bigskip \examname \\
\href{mailto:ludovico.iovino@gssi.it}{ludovico.iovino@gssi.it}
\end{flushright}
\end{minipage}\\[2.0cm]
 
%\large \textit{A thesis submitted in fulfilment of the requirements\\ for the degree of \degreename}\\[0.3cm] % University requirement text
%\textit{in the}\\[0.4cm]
%\groupname\\\deptname\\[2cm] % Research group name and department name


{\large \today}\\[2.2cm] % Date

\univname \\
\addressnames
%\includegraphics{Logo} % University/department logo - uncomment to place it
 
\vfill
\end{center}

\end{titlepage}

\cleardoublepage

%\restoregeometry

%----------------------------------------------------------------------------------------
%	DECLARATION PAGE
%	Your institution may give you a different text to place here
%----------------------------------------------------------------------------------------

%\Declaration{
%
%\addtocontents{toc}{\vspace{1em}} % Add a gap in the Contents, for aesthetics
%
%I, \authornames, declare that this thesis titled, '\ttitle' and the work presented in it are my own. I confirm that:
%
%\begin{itemize} 
%\item[\tiny{$\blacksquare$}] This work was done wholly or mainly while in candidature for a research degree at this University.
%\item[\tiny{$\blacksquare$}] Where any part of this thesis has previously been submitted for a degree or any other qualification at this University or any other institution, this has been clearly stated.
%\item[\tiny{$\blacksquare$}] Where I have consulted the published work of others, this is always clearly attributed.
%\item[\tiny{$\blacksquare$}] Where I have quoted from the work of others, the source is always given. With the exception of such quotations, this thesis is entirely my own work.
%\item[\tiny{$\blacksquare$}] I have acknowledged all main sources of help.
%\item[\tiny{$\blacksquare$}] Where the thesis is based on work done by myself jointly with others, I have made clear exactly what was done by others and what I have contributed myself.\\
%\end{itemize}
% 
%Signed:\\
%\rule[1em]{25em}{0.5pt} % This prints a line for the signature
% 
%Date:\\
%\rule[1em]{25em}{0.5pt} % This prints a line to write the date
%}
%
%\clearpage % Start a new page

%----------------------------------------------------------------------------------------
%	QUOTATION PAGE
%----------------------------------------------------------------------------------------

\pagestyle{empty} % No headers or footers for the following pages

\null\vfill % Add some space to move the quote down the page a bit

\textit{``There are many potentially valuable academic insights that just wither and die on the vine because they aren’t pushed far enough to entice corporations or policy makers to adopt them. 
This intermediary step is time consuming, it's tedious, it's maybe expensive, and it's really not rewarded enough in academics."}

\begin{flushright}
Steven Levitt
\end{flushright}

\vfill\vfill\vfill\vfill\vfill\vfill\null % Add some space at the bottom to position the quote just right

\clearpage % Start a new page

%----------------------------------------------------------------------------------------
%	ABSTRACT PAGE
%----------------------------------------------------------------------------------------

\addtotoc{Abstract} % Add the "Abstract" page entry to the Contents

\abstract % Add a gap in the Contents, for aesthetics
\paragraph{Context:} 
Software is an important part of modern life, and in most cases, it provides tremendous benefits to society.
Unfortunately, software is highly susceptible to faults.
Faults are often harmless, but even small errors can cause massive damage depending on the context.
Thus, it is crucial for software developers to adopt testing techniques that can help locate faults and guarantee the functionality of both individual components and the system as a whole.
Today, there is a trend towards \textit{continuously evolving software}, in which it is desired that changes such as new features and corrections are delivered to end users as quickly as possible.
To ensure correct behavior upon release, development teams rely on \textit{regression testing} suites, which serve to validate previously-correct features and, when well-designed, avoid the propagation of faults to end users.
However, the desire for velocity that comes with continuously evolving software places an additional burden on regression testing practices, as running complete test suites can be a costly process in large-scale software.
This challenge has generated a need for novel regression testing techniques, a topic which now enjoys a robust literature within software engineering research.
However, there is limited evidence of this research finding its way into practical usage by the software development community; in other words, there is a disconnect between academia and industry on the subject of software testing techniques.
%While there is a broad field of research in this topic, few state-of-the-art techniques are actually put into practice by software developers.

\paragraph{Objective:} 
To improve applicability of regression testing research, we must identify what are the main causes of this apparent gap between software engineering academics and practitioners.
This is a multifaceted goal, involving an investigation of the literature and of the state of practice.
A related goal is to provide examples of \textit{test suite orchestration strategies} that draw from academic advancements and could provide benefits if implemented on real software.

\paragraph{Method:}
This thesis tackles the aforementioned challenge from multiple directions.
It includes a comprehensive systematic literature review covering research published between 2016 and 2022, focusing on papers that bring techniques and discussions that are relevant to the applicability of regression testing research.
Along with data extracted from the papers themselves, this discussion of the existing literature includes information received directly from authors through a questionnaire, as well as a survey performed with practitioners, seeking to validate some of the reported findings.

Test suite orchestration strategies can be a step towards bridging the so-called \textit{industry-academia knowledge gap}.
To that end we propose a combined approach for regression testing, including techniques extracted from the literature that have promising qualities.
This approach is an initial experiment with full test suite orchestration and extended approaches are also discussed.

To get a closer understanding of the state of regression testing in a practical sense, a series of interviews were conducted in collaboration with a large technology company.
During a seven-week process, we were able to interact with the team and learn the test practices performed on a daily basis and have some insight on the long-term test strategies for the company.
The responses of the interviews are reported, edited for readability and confidentiality reasons, and these results are discussed within the larger context of the study.

The results from the above components of the studies are then aggregated into two notable outputs.
First, a live repository of literature is made available online, containing the current results of the literature review and with the opportunity of expansion as more research is performed in this topic.
Then, we provide a list of the most notable challenges for the implementation of regression testing techniques in practice, that were identified during the development of this entire study.

%This thesis brings forth a discussion on the so-called \textit{industry-academia knowledge gap} when it comes to software regression testing, bringing a comprehensive literature review along with \textit{in loco} interviews with practitioners, with the objective of identifying the ongoing challenges that can be addressed in order to bridge that gap.

\paragraph{Results:}
This thesis provides the following contributions:
a comprehensive literature review of applicable regression testing research;
additional context on the literature provided by the authors of cited papers;
a preliminary test suite orchestration strategy combining robust techniques from the literature;
interviews with practitioners at a major technology company that highlight the challenges faced daily by developers and testers;
a live repository of papers to aggregate relevant literature in one online location;
a list of challenges that can serve as guidelines for researchers or even as research directions in their own right.

\paragraph{Conclusion:}
There is still much work to be done by the software engineering research and development communities in order to completely close the gap that exists between them.
To a great extent, the motivations of researchers and practitioners are not aligned --- while in academia, proposing theoretically sound novel approaches is encouraged to obtain publications, in industry there is a need for techniques that are proven to reduce effort and/or costs.
This can only be solved by close collaboration between the two sides, yet a question of who is willing to fund these experiments remain.
The data and discussions provided in this thesis show that, although difficult, this is not an impossible problem to solve and there are certain clear steps that can be taken by researchers and practitioners alike to begin addressing it.

%As an initial step towards a solution, we propose \textit{orchestration strategies} for regression testing of continuously-evolving software; to do so, we perform experiments using combinations of certain regression testing techniques that have displayed potential for real-world use and draw conclusions on the advantages and drawbacks of adopting such strategies.

%In this research proposal, we discuss the reasons and challenges behind this disconnection, and propose the development of new test orchestration strategies that aim to combine relevant software testing techniques into a complete solution that can be applied by practitioners.
%In order to help bring academia and industry closer together, we also propose the usage of applicability metrics that will be used to measure results of software testing techniques and compare how well each of them can perform in an industrial setting. 
\cleardoublepage % Start a new page

%----------------------------------------------------------------------------------------
%	DEDICATION
%----------------------------------------------------------------------------------------

\setstretch{1.3} % Return the line spacing back to 1.3

\pagestyle{empty} % Page style needs to be empty for this page

\dedicatory{In memory of Prof. Dr. Luiz Felipe Paula Soares\\ and Prof. Dr. Francisco de Paula Soares Filho} % Dedication text

\addtocontents{toc}{\vspace{2em}} % Add a gap in the Contents, for aesthetics


%----------------------------------------------------------------------------------------
%	ACKNOWLEDGEMENTS
%----------------------------------------------------------------------------------------

\setstretch{1.3} % Reset the line-spacing to 1.3 for body text (if it has changed)
%
\acknowledgements{\addtocontents{toc}{} % Add a gap in the Contents, for aesthetics
%
%The acknowledgements and the people to thank go here, don't forget to include your project advisor\ldots

I am incredibly grateful for the guidance and orientation provided by my advisors, professors Antonia Bertolino, Breno Miranda and Ludovico Iovino.
I would also like to thank GSSI professors Luca Aceto and Michele Flammini, who have been supportive in times of need since my early days at the institute.

Thanks to Milos Gligoric, from University of Texas at Austin, who provided significant contributions to the discussions and results presented in \Cref{chap:orchestration}.
I also thank Sigrid Eldh, from M\"alardens Universitet, who arranged the collaboration that allows the discussion present in \Cref{chap:industry}.
I am also grateful to the people at the industrial partner, who received me with grace and gave me the opportunity of interacting with their team, but who shall remain anonymous due to confidentiality concerns.
If you are reading this, you know who you are.
%Carene Österberg and Tommie Haag, from Ericsson, who gave me the opportunity of interacting with them and their team and without whom there would be no \Cref{chap:industry} in this thesis.

Thanks to my parents, Lizmari and Edison Greca, and my grandparents, Glacial and Aquiles Merlin, who gave me the foundational values and education that have guided me through life until this point.
Additionally, I would like to remember my grandparents Berenice and Eros Greca, who I wish could be here to share their part in this achievement, and my uncles Felipe and Chico, who encouraged me to explore the world.

Thanks to my colleagues Debashmita Poddar, Konstantin Prokopchik and Alex Coto for the companionship throughout the PhD and the many board game nights that kept us sane during the lockdowns.

I would also like to thank professors Luiz Albini and Eduardo Todt from UFPR, for the education I received before coming to Italy and for the warm welcome I received when visiting my \textit{alma mater}.

Thanks to Marco Rotilio, whose music lessons were an outlet of creativity.

Finally, I would like to acknowledge and thank my friends: André Ramos, Arthur Alves, Arthur Pieri, Cainã Trevisan, Dácio Augusto, Darren Kerwin, Douglas Novelli, Eric Bunese, Felipe de Lara, Felipe Gugelmin, Gustavo Henrique, Janderson Oliveira, Lucas Knopki, Luiz Roveran, and Pedro Vicente.
Even thousands of kilometers away, you are all special to me.

}
\clearpage % Start a new page

%----------------------------------------------------------------------------------------
%	LIST OF CONTENTS/FIGURES/TABLES PAGES
%----------------------------------------------------------------------------------------

\pagestyle{fancy} % The page style headers have been "empty" all this time, now use the "fancy" headers as defined before to bring them back

\lhead{\emph{Contents}} % Set the left side page header to "Contents"
\tableofcontents % Write out the Table of Contents

\lhead{\emph{List of Figures}} % Set the left side page header to "List of Figures"
\listoffigures % Write out the List of Figures

\lhead{\emph{List of Tables}} % Set the left side page header to "List of Tables"
\listoftables % Write out the List of Tables

%%----------------------------------------------------------------------------------------
%%	ABBREVIATIONS
%%----------------------------------------------------------------------------------------
\clearpage % Start a new page
\setstretch{1.5} % Set the line spacing to 1.5, this makes the following tables easier to read
\lhead{\emph{Abbreviations}} % Set the left side page header to "Abbreviations"
\listofsymbols{ll} % Include a list of Abbreviations (a table of two columns)
{
\textbf{APFD}: \textbf{A}verage \textbf{P}ercentage of \textbf{F}aults \textbf{D}etected\\
\textbf{CI/CD}: \textbf{C}ontinuous \textbf{I}ntegration/\textbf{C}ontinuous \textbf{D}elivery (or \textbf{D}eployment) \\
\textbf{FAD}: \textbf{F}unctional \textbf{A}rea \textbf{D}omain \\
\textbf{FOSS}: \textbf{F}ree and \textbf{O}pen-\textbf{S}ource \textbf{S}oftware \\
\textbf{IR\&A}: \textbf{I}ndustrial \textbf{R}elevance and \textbf{A}pplicability \\
\textbf{LIRT}: \textbf{L}ong \textbf{I}nterval \textbf{R}egression \textbf{T}est(ing) \\
\textbf{MCT}: \textbf{M}ulti-\textbf{C}omponent \textbf{T}est(ing) \\
%\textbf{RAN}: \textbf{R}adio \textbf{A}rea \textbf{N}etwork \\
\textbf{RT}: \textbf{R}egression \textbf{T}esting \\
%\textbf{SBT}: \textbf{S}ource \textbf{B}aseline \textbf{T}est \\\
\textbf{SIRT}: \textbf{S}hort \textbf{I}nterval \textbf{R}egression \textbf{T}est(ing) \\
\textbf{SLR}: \textbf{S}ystematic \textbf{L}iterature \textbf{R}eview \\
\textbf{SUT}: \textbf{S}ystem \textbf{U}nder \textbf{T}est \\
\textbf{TCP}: \textbf{T}est \textbf{C}ase \textbf{P}rioritization \\
\textbf{TCS}: \textbf{T}est \textbf{C}ase \textbf{S}election \\
\textbf{TSA}: \textbf{T}est \textbf{S}uite \textbf{A}mplification or \textbf{A}ugmentation \\
\textbf{TSR}: \textbf{T}est \textbf{S}uite \textbf{R}eduction \\
\textbf{TR}: \textbf{T}rouble \textbf{R}eport \\
\textbf{XFT}: \textbf{X} (Cross) \textbf{F}unctional \textbf{T}eam \\



%\textbf{AMI} & \textbf{A}dvanced \textbf{M}etering \textbf{I}nfrastructure \\
%\textbf{CPS} & \textbf{C}yber-\textbf{P}hysical \textbf{S}ystems \\
%\textbf{DoS} & \textbf{D}enial-\textbf{o}f-\textbf{S}ervice \\
%\textbf{MAC} & \textbf{M}edium \textbf{A}ccess {C}ontrol \\
%\textbf{MAPE} & \textbf{M}onitor, \textbf{A}nalyse, \textbf{P}lan, \textbf{E}xecute \\
%\textbf{MCN} & \textbf{M}ulti-hop \textbf{C}ontrol \textbf{N}etwork \\
%\textbf{MILS} & \textbf{M}ultiple \textbf{I}ndependent \textbf{L}evels of \textbf{S}ecurity/Safety\\
%\textbf{MRMC} & \textbf{M}arkov \textbf{R}eward \textbf{M}odel \textbf{C}hecker \\
%\textbf{MTD} & \textbf{M}oving \textbf{T}arget \textbf{D}efence \\
%\textbf{NICS} & \textbf{N}etworked \textbf{I}ndustrial \textbf{C}ontrol \textbf{S}ystems \\
%\textbf{OPC} & \textbf{O}bject linking and embedding for \textbf{P}rocess \textbf{C}ontrol \\
%\textbf{PCS} & \textbf{P}rocess \textbf{C}ontrol \textbf{S}ystems \\
%\textbf{PLC} & \textbf{P}rogrammable \textbf{L}ogic \textbf{C}ontroller \\
%\textbf{PRISM} & \textbf{P}robabilistic \textbf{S}ymbolic \textbf{M}odel \textbf{C}hecker \\
%\textbf{RTU} & \textbf{R}emote \textbf{T}erminal \textbf{U}nit \\
%\textbf{SCADA} & \textbf{S}upervisory \textbf{C}ontrol \textbf{A}nd \textbf{D}ata \textbf{A}cquisition\\
%\textbf{SLR} & \textbf{S}ystematic \textbf{L}iterature \textbf{R}eview \\
%\textbf{SMC} & \textbf{S}tatistical \textbf{M}odel \textbf{C}hecking \\
%\textbf{VCSE} & \textbf{V}irtual \textbf{C}ontrol \textbf{S}ystem \textbf{E}nvironment \\
%\textbf{Acronym} & \textbf{W}hat (it) \textbf{S}tands \textbf{F}or \\
}

%\clearpage % Start a new page
%\setstretch{1.5} % Set the line spacing to 1.5, this makes the following tables easier to read
%\lhead{\emph{Glossary}} % Set the left side page header to "Abbreviations"
%\listofglossary{ll} % Include a list of Abbreviations (a table of two columns)
%{
%\textbf{Tool} & Something Something \\
%}

%----------------------------------------------------------------------------------------
%	PHYSICAL CONSTANTS/OTHER DEFINITIONS
%----------------------------------------------------------------------------------------

%\clearpage % Start a new page
%
%\lhead{\emph{Physical Constants}} % Set the left side page header to "Physical Constants"
%
%\listofconstants{lrcl} % Include a list of Physical Constants (a four column table)
%{
%Speed of Light & $c$ & $=$ & $2.997\ 924\ 58\times10^{8}\ \mbox{ms}^{-\mbox{s}}$ (exact)\\
%% Constant Name & Symbol & = & Constant Value (with units) \\
%}

%----------------------------------------------------------------------------------------
%	SYMBOLS
%----------------------------------------------------------------------------------------

%\clearpage % Start a new page
%
%\lhead{\emph{Symbols}} % Set the left side page header to "Symbols"
%
%\listofnomenclature{lll} % Include a list of Symbols (a three column table)
%{
%$a$ & distance & m \\
%$P$ & power & W (Js$^{-1}$) \\
%% Symbol & Name & Unit \\
%
%& & \\ % Gap to separate the Roman symbols from the Greek
%
%$\omega$ & angular frequency & rads$^{-1}$ \\
%% Symbol & Name & Unit \\
%}

%----------------------------------------------------------------------------------------
%	THESIS CONTENT - CHAPTERS
%----------------------------------------------------------------------------------------

\mainmatter % Begin numeric (1,2,3...) page numbering

\pagestyle{fancy} % Return the page headers back to the "fancy" style

% Include the chapters of the thesis as separate files from the Chapters folder
% Uncomment the lines as you write the chapters

%----------------------------------------------------------------------------------------
%	Introduction
%----------------------------------------------------------------------------------------
\chapter{Introduction}\label{chap:intro}
\lhead{\emph{Introduction}} % Set the left side page header to "Introduction"

Software has become an ubiquitous part of every-day life, be it in computers, smartphones, vehicles, or other devices.
Well-functioning software can be a major asset for most people, helping to reduce operational costs, reduce time spent on tasks, and even save lives.
However, it is already in the common sense that software is not always perfect, and, from time to time, it may behave incorrectly or unexpectedly.

A recent and devastating example is the Boeing 737 Max, an aircraft that was involved in two fatal crashes in 2018 and 2019, and had to be grounded for months, due to what was likely a software fault \cite{levin_latest_2019}.
Needless to say, this software fault caused the tragic loss of many lives, as well as billions of dollars of expenses to Boeing, airlines, airports and passengers.

As such, it is important for companies and communities developing software to utilize methods to mitigate the possibility that faulty software will reach production.
Today, software is often accompanied by a \textit{test suite}, a series of automated tests that are used to provide a level of certainty that parts of a software, both in isolation and in conjunction, correctly perform the tasks to which they are assigned.
One widely-adopted software testing technique is called \textit{regression testing}; its primary role is to execute the test suite with a certain frequency, in order to guarantee that recently introduced changes to the software have not affected previously-correct behavior.

However, in large-scale software development (that is, with multiple developers and a large codebase), it is usually unfeasible to execute every test after every change, either because changes are too frequent, or because there are too many tests, or both.
This is aggravated by the fact that most software is now developed in a \textit{continuous} manner, in which it is desirable that recent changes are put in production as fast as possible.

To alleviate this issue, software developers and testers can design \textit{test orchestration} strategies, which will automatically aid the process of regression testing.
These strategies can be used for various purposes, such as selecting and prioritizing the most relevant test cases, or generating new test cases based on recent changes to the codebase.

A major challenge in software testing research is that, until recently, there was little concern regarding the practical applicability of methods and techniques proposed by academia.
Due to this, academia and industry diverged into different paths that wish to reach similar goals, although with different priorities.
Many academic works on the topic aim for highly-precise techniques that, when applied in practice, are too slow to be useful or unrealistically require resources that are not commonly available.
On the other hand, many practitioners already apply coarse-grained techniques that provide some reduction in costs, but that could do much better with further thought and research. 
Therefore, it is now a major concern for researchers and practitioners that new software testing techniques are designed and evaluated considering their applicability for real-world software.

In this research proposal, we discuss several challenges that are currently keeping academic and industrial techniques from converging.
Our aim is to develop new test orchestration strategies for continuously-evolving software that combines the best techniques proposed in academia into a complete solution applicable in industry.
In order to rank and categorize techniques, we will introduce certain \textit{applicability metrics} that determine how well-suited a technique is for industrial application.

We hope that this research will aid both future researchers and practitioners in developing and applying software testing techniques, thus resulting in a general improvement of quality in software.
Within possibility, we would like to work in conjunction with professionals from the software industry in order to tune the development and evaluation of our strategies according to the real-world needs.

The remainder of this document is structured as follows.
First, \autoref{chap:background} introduces the concepts of regression testing and test orchestration.
Then, \autoref{chap:sota} reviews recent literature on the industrial applicability of software testing techniques, and summarizes some proposed techniques that show promising results for regression testing.
Finally, \autoref{chap:proposal} elaborates on the challenges we wish to tackle with this research, and provides an overview of how work will be conducted in the upcoming years.
%----------------------------------------------------------------------------------------
%	Background
%----------------------------------------------------------------------------------------
\chapter{Background}\label{chap:background}
\lhead{\emph{Background}} % Set the left side page header to "Introduction"

This work is about test orchestration strategies for regression testing.
Both concepts are widely studied within the software engineering.
In this chapter we provide a brief introduction to both, focusing on the challenges that will be tackled by this research.

\section{Summary}\label{sec:background_summary}

%\section{Evolving Software Systems}

In the early stages of commercial software development, computer programs were designed, produced and distributed mostly like physical retail goods.
That is, there was an initial planning and design phase, followed by an extensive development period and, on a certain deadline, the software was shipped embedded with hardware, or pressed onto floppy disks or CDs to be made available in shelves.
The advent of the Internet made it possible to completely alter this paradigm.
Now, these three phases still exist in commercial software, but happen much faster and can be repeated iteratively as needed.
In other words, software companies can initially design and develop the ``minimum viable product'' to be delivered to customers online and, with the software already in use, updates can be develop to add new features, improve existing ones, or correct bugs that can be detected\footnote{This is not the same for all types of software; embedded systems cannot always rely on the ability of pushing patches; and video games, for example, generally deliver complete products on a given deadline to account for distribution and marketing schedules, but even they almost always have an extensive post-release update cycle.}.

The shift to evolving software, which correlates to the pivot to agile development practices in the mid-2000s, also caused a significant change to how software testing is viewed and addressed.
Previously, it was common to have team members solely responsible for testing the developers' code, often times manually.

In this work, we are interested in regression testing in industrial settings.
We use ``industrial setting'' as a general term for large-scale software in the real world.
In practice, it can mean several different kinds of software, such as software developed as the primary product of a corporation (in the technology industry), software that provides essential features to other products (such as in the automotive or telecommunication industries), or open-source software that is developed by a community instead of a team within a company.


\todo{Explain context of modern evolving software}


\todo{Perhaps also explain here the categories of software in terms of scale or domain}

\section{Regression Testing of Evolving Software Systems}
\label{sec:regression}

In the early stages of commercial software development, computer programs were designed, produced and distributed mostly like physical retail goods.
That is, there was an initial planning and design phase, followed by an extensive development period and, on a certain deadline, the software would be shipped embedded with hardware, or pressed onto disquettes or CDs that could be mailed to customers or made available in store shelves.
The advent of the Internet made it possible to completely alter this paradigm.
Now, these three phases still exist in commercial software, but happen much faster and can be repeated iteratively as needed.
In other words, software companies can initially design and develop the ``minimum viable product'' to be delivered to customers online and, with the software already in use, updates can be develop to add new features, improve existing ones, or correct bugs that can be detected\footnote{This is not the same for all types of software; for example, embedded systems cannot always rely on the ability of online updates; meanwhile video games generally deliver complete products on a given deadline to account for distribution and marketing schedules, often followed by an extensive post-release update cycle which resembles the evolving software paradigm.}.
 
We denote software developed and released as ever-evolving products as \textit{continuously evolving software}.
The concept of evolving software was introduced in the 1970s by \citet{Lehman1980}, although it was in the 1990s that the term and paradigm gained widespread use, due to the accelerated delivery methods becoming available \cite{Mens2008}.
It can also happen that programs that were originally designed according to the traditional release cycle are, at some point, adapted and converted to be continuously evolving (e.g. Microsoft Windows shifted from yearly ``service packs'' to weekly online updates).

The shift to evolving software, which correlates to the pivot to agile development practices in the mid-2000s, also caused a significant change to how software testing is viewed and addressed.
Previously, it was common to see testing as its own stage of development; certain teams had members solely responsible for testing the source code, which was usually a manual process.
Nowadays, it is common practice for developers to write and test their own code, and have an active role in the maintenance of the regression testing suite, a practice encouraged by the agile method \cite{planview_agile_testing}.
This has the advantage of speeding up the testing process, although as a drawback it can cause testing to be seen as a ``second-class citizen'' by developers, who would rather create new features than test existing ones.

\textit{Regression testing} is the part of software testing concerned with testing previously existing components of a system to guarantee that recent changes in the codebase did not affect the originally specified functionality of components.
This process is one of the costliest aspects of software development \cite{rothermel_improving_2018}, as it should ideally be performed every time a code change is committed, and involves much repetition of previously performed tests.
It is defined in \cite{minhas_regression_2017} as ``an activity which makes sure that everything is working correctly after changes to the system.''
That is, its primary objective is to assure that, after each change to the software, previously existing code continues to comply to specification (or simply to expectations, in case no formal specification exists).

The term \textit{regression testing} (\rt) has its origins in pre-agile days and, as a research topic, has been studied since the 1980s~\cite{leung1989insights,yoo2012regression}.
At the time, release schedules were centered around a hard deadline, so \rt was an activity that was only performed near the end of the cycle, after the important features of the release had already been developed.
At that point, testers would check if any of the new changes interfered with previous functionality of the software; in some places this was a manual process, in others semi-automated.
Doing so earlier was not advantageous --- if a bug is detected in the middle of development but new features are not yet complete, it is possible that another bug will be detected on another round of testing.
Since the software could only be shipped once all features were done, intermediary regression testing provided little benefits.

Continuously evolving software shifted this dynamic.
With smaller and more frequent release cycles, regression testing too became a more frequent activity.
At the same time, the incredible feature speed demanded by customers and consumers means that it is not viable to postpone testing until right before release --- if a bug is detected at that point, it might be too late to fix it before delivery.
Thus, with the development of Continuous Integration/Continuous Delivery (CI/CD) practices and tools, automated regression testing became commonplace, sometimes executed as frequently as new code changes are pushed into a repository.

Test automation mostly solves the problem in small projects, where it takes only a few seconds or maybe minutes to run a full test suite.
Large-scale software demand additional attention because of two factors: the test suite is large and takes a long time to execute, and code commits arrive at such a high frequency that there is not enough time to run the test suite between each commit.
Often, a combination of both factors become a major challenge in large-scale software development \cite{memon_taming_2017}.

In order to maintain the health of the testing process and the availability of testing equipment, the execution time of a suite should be less than the average time between commits pushed by developers.
In reality, this is difficult to achieve and maintain, as test suites tend to increase in scale (according to the necessities of an ever-growing software) and commit frequency remains stable or can even increase if new developers are added to the team.
The straightforward solution is to increase the computational power of testing servers, so testing time reduces by brute force, although obviously this incurs additional costs.

The concept of software size and scale is fundamental for the motivation of this research.
There are multiple ways of measuring software scale --- it could refer to a large number of lines of code (LOC); it can also mean high-complexity algorithms that run for a long time, or software that needs to serve multiple users simultaneously.
For this research, we are considered primarily with the number of test cases that the program needs to be reliable.
Thus, other aspects of software scalability go beyond the scope of this thesis and, here, the term ``scalability'' itself refers primarily to the ability of managing an ever-growing number of test cases.

That said, in this work, we are interested in ``industrial-scale evolving software''.
Understand ``industrial-scale'' as a general term for large-scale software in the real world.
In practice, it can mean several different kinds of software, such as software developed as the primary product of a corporation (in the technology industry), software that provides essential features to other products (such as in the automotive or telecommunication industries), or open-source software that is developed by a community instead of a team within a company.

It is also noteworthy that software can exist in a multitude of contexts --- e.g. embedded software, distributed systems, web or mobile applications, cloud-based solutions, and so forth.
Each context is associated with unique challenges that inevitably alter how they are designed.
For the most part, this thesis explores testing strategies that can be generalized into most contexts, as long as the software is continuously evolving in nature, although the ultimate implementation of these strategies might require adjustments.


\section{Test Suite Orchestration}\label{sec:orchestration}

Given the challenges associated with ever-expanding regression testing suites of continuously evolving software, we 
\textit{Test suite orchestration} is the art of generating, choosing, prioritizing and executing tests in order to maximize the effectiveness of testing while keeping costs within a desired budget.
Today, research on test orchestration is quite granular, with individual researchers mostly focusing on specific challenges within this topic.
While this is important for the continuity and advancement of research, it fails in addressing the practical concerns of software developers, who desire a complete solution to aid the development cycle.

Features such as \textit{test case generation}, \textit{test case prioritization}, handling of \textit{flaky tests}, \textit{mutation testing}, \textit{test suite augmentation} and others can be considered under the broader scope of test orchestration.
While improvements in each of these features can provide substantial benefits, it is their combination that can produce the desired solution.

In general, test suite orchestration can be thought of as a broad challenge with the ultimate goal of improving regression testing in multiple aspects, composed of several sub-challenges.
These sub-challenges include, but are not limited, to the following:
\begin{itemize}
	\item \textbf{Test case selection (\tcs):} the challenge of determining a sub-set of tests that, when executed, provides sufficiently high confidence that recent changes have not introduced failures in the software, while substantially reducing the execution costs.
	\item \textbf{Test case prioritization (\tcp):} the challenge of ordering tests to detect potential faults as early as possible, prioritizing tests that are most likely to reveal faults or that cover critical parts of the program.
	\item \textbf{Test suite reduction or minimization (\tsr):} the challenge of reducing the test suite by finding and possibly removing redundant tests. Unlike \tcs, which is change-aware, \tsr.
	\item \textbf{Test suite amplification or augmentation:} the challenge of expanding and improving an existing test suite through various different means. A survey on test suite amplification is found in \cite{danglot_snowballing_2019}; out of the categories presented, the synthesis of new tests with respect to changes is the most relevant for a continuously-evolving system. 
	\item \textbf{Handling of unreliable/flaky tests:} a test that might pass or fail non-deterministically without changes to the SUT is designated as unreliable or flaky. This can happen due to poor test design, misconfiguration of the test suite or the testing environment, or timing errors in asynchronous tasks. These tests make it difficult for developers to identify true faults in the system and thus they should ideally be detected and flagged as such.
	\item \textbf{Test uncertainty:} the challenge of considering uncertain factors in software development, such as human input, values generated by machine learning, or cyber-physical interactions. In \cite{garlan_software_2010}, the sources, implications and challenges of uncertainty in software engineering are explored.
	\item \textbf{Compositional testing:} the challenge of guaranteeing correctness of a whole system by individually testing its distinct components. For example, a system using multi-component testing should be able to rely on the preceding single-component tests being correct. This challenge is mentioned in \cite{harman_start-ups_2018}, where the authors suggest the need of mock functions and analysis that can ``begin anywhere''.
	\item \textbf{Incremental testing:} the challenge of testing new parts of a software without necessarily having to re-test the whole software. In \cite{harman_start-ups_2018} and \cite{ohearn_continuous_2018}, the notion of using procedure summaries as a way of handling incrementality is mentioned. This way, it is possible to use previous executions of the test suite to accelerate its execution in the future when only small parts of code are added or changed.
\end{itemize}

Individually, each of these challenges can be its own field of research, and indeed many works have been published on them.
However, an ideal test orchestration solution should consider all or most of these challenges in unison, as solving each one alone is not sufficient to solve the problems faced by software developers in practice.

Due to the breadth of the orchestration challenge, for this thesis the decision was made to restrict the scope and focus primarily on four aspects: test case prioritization, test case selection, test suite reduction/minimization and test suite amplification/augmentation.
Other topics remain tangential to the research and may occasionally be part of the discussion, but are not the focus of this work.
The following subsection describes in more detail the four challenges that this study focuses on.

\subsection{Test Case Prioritization}
\label{sec:tcp}

Another challenge of regression testing is to detect failing tests fast.
The objective of \tcp is to re-order test cases according to some definition of priority, in order to get faster feedback from the test execution.
A prioritized test suite still contains all test cases, 
so there is no loss of failures detection ability (assuming that test results are independent) -- what changes is the amount of time that it takes for one or more failures to be detected.
\tcp can be described as a function $P(T)$ that provides a permutation of $T$.
Some criteria often used for \tcp include similarity-based, coverage-based, and history-based~\cite{khatibsyarbini_test_2018}.


\subsection{Test Case Selection}
\label{sec:tcs}

In regression testing, not all tests are relevant to a particular code change:
if only a small part of one file was updated, it is unlikely that the entire project would be affected and the full regression test suite would have to be run.
\tcs addresses the challenge of selecting a subset of tests that is representative of the entire suite in a given situation~\cite{YooHarman10RegressionTestingSurvey, RothermelHarrold94FrameworkForEvaluationRTS}.
In other words, given a test suite $T$, 
\tcs can be described as a function $S(T)$ that selects a subset of $T$ to be used for testing the current version of the system under test.
We say that a \tcs technique is \emph{safe} if it guarantees that all tests whose outcome may be affected by a change are included in the selected subset~\cite{RothermelHarrold94FrameworkForEvaluationRTS}. 
Common approaches for \tcs are change-based, history-based, and model-based~\cite{kazmi_effective_2017}.

\subsection{Test Case Reduction and Minimization}\label{sec:tsr}

\subsection{Test Case Amplification and Augmentation}\label{sec:tsa}

%----------------------------------------------------------------------------------------
%	Literature Review
%----------------------------------------------------------------------------------------
\chapter{Literature Review}\label{chap:literature_review}
\lhead{\emph{Literature Review}} % Set the left side page header to "Introduction"
%----------------------------------------------------------------------------------------
%	Test Suite Orchestration
%----------------------------------------------------------------------------------------
\chapter{Test Suite Orchestration}\label{chap:orchestration}
\lhead{\emph{Test Suite Orchestration}} % Set the left side page header to "Introduction"

%\section{Summary}\label{sec:orch_summary}

%Software regression testing is actively researched~\cite{rosero_15_2016,bin_ali_search_2019,YooHarman10RegressionTestingSurvey}, and many techniques have been proposed, including test case selection~\cite{kazmi_effective_2017} and prioritization~\cite{khatibsyarbini_test_2018}.

%Both test case selection (\tcs) and test case prioritization (\tcp) techniques aim at detecting regression failures, but they follow different strategies. 
%In \tcs, when a new software version is released, a subset of test cases is selected from the available test suite aiming at exercising all the latest code changes. 
%\tcs is proposed as an alternative to a \textit{retest-all} (i.e., running all tests at each version) strategy that is not sustainable in many practical cases.
%On the other hand, in \tcp the test suite is re-ordered, aiming at executing first those test cases that are more likely to fail. As, of course, we cannot know in advance which test cases discover which failures, different \tcp criteria have been proposed, such as code coverage. 
 
%Many \tcs and \tcp approaches have been proposed~\cite{soetens2016change,legunsen2016,henard2016,luo2018static}.
%Our research goal here is not that of inventing yet another approach, but rather to understand if and how \tcs and \tcp  should be used in combination, i.e.:
%when a new software version is released, is it more convenient to apply a \tcs approach or instead a \tcp one?
%Intuitively, a combination of both techniques would provide the most benefit, but what are the resulting challenges and drawbacks of this approach?
%Notwithstanding the vast literature on  regression testing, 
%such type of questions remain largely unanswered.

We understand \textit{test suite orchestration} as a series of steps that can be performed before the execution of a test suite, with the objective of improving the effectiveness and efficiency of the suite.
This can have an impact in the moments that immediately follow the orchestration, as well as in the long-term evolution, health and usefulness of the suite, as it adapts to an SUT that is also continuously evolving.

As a step towards the goal of fully automated orchestration of test suites that can be useful in real-world software, 
we focus here on regression testing techniques extracted from the existing literature that have been conceived for practical relevance and scalability.
Specifically, as a representative \tcs approach we adopt \ek~\cite{gligoricEk} while for \tcp we use \fs~\cite{miranda_fast}.
The criteria used for selecting these tools were: their cost-effectiveness and simplicity of application; their availability as open-source programs; finally, also for convenience as authors of both tools were also involved with the development of the current study.

Concerning \tcs, in an empirical study conducted in 2014~\cite{gligoric_empirical}, the authors observed that many techniques were not adopted in practice and developers mostly continued to perform manual selection of test cases.  
Motivated by this study, Gligoric et al.~\cite{gligoricEk} proposed  \ek,  a lightweight \tcs technique that leverages file dependencies.
Besides the original paper on \ek, several follow-up studies showed the benefit of file-based selection over other approaches~\cite{legunsen2016, Zhang18HybridRTS}.

Concerning \tcp, in a recent  study Miranda et al.~\cite{miranda_fast} showed that many existing techniques do not scale-up to large test suites.
They hence proposed the \fs approach that applies Locality-Sensitive Hashing (LSH) techniques~\cite{Leskovec:2014} for similarity-based prioritization.
In the original work, the authors assess \fs against several competing \tcp techniques, showing that it gives comparable effectiveness but with higher efficiency.

This work stems from the simple yet powerful idea of comparing these two approaches---\tcs by \ek and \tcp by \fs---and possibly taking the advantages of each while overcoming their potential shortcomings. 
%
We make the following two observations:
\begin{itemize}
\item \ek comes with no notion of test case priority: it assumes that all the selected test cases are run and makes no distinction about whether a failure is found by the first or the last executed test case;
\item \fs reorders tests with the goal to detect failures early, but does not consider recent code changes, whereas we know from practice that these are related with failures, e.g.,~\cite{knauss2015supporting,elbaum2014techniques}.
% quickly allows anticipated
\end{itemize}

By combining \ek and \fs, we  aim at developing 
a \textit{practical} and \textit{effective} approach to regression testing that we call \emph{\fz}.
This is meant to be practical because it combines two scalable techniques, and effective because it overcomes the above shortcomings of each.
In particular, this combined approach aims to decrease \textit{developer feedback time}, which is the time it takes for a developer to receive a test failure notification once testing begins.

%On the one side, \fz could be understood as just a technique that integrates \ek and \fs. 
%On the other side, we see an interesting scientific novelty in this approach. 
%In fact, by combining \textit{file-based} selection with \textit{similarity-based} prioritization it implements a \textit{multi-objective regression technique} in line with the case made by Harman~\cite{harman2011making}.
%Indeed, research in combining multiple criteria in the context of one technique is very active, e.g.,~\cite{epitropakis2015empirical,garousi2018multi}.
%%, as it has been shown that hybrid or multi-criteria are more effective.
%Less attention has been devoted to combining multiple criteria while addressing regression techniques together, which we call \textit{regression test orchestration}.
%For example, Silva et al.~\cite{silva2016hybrid} proposed to combine prioritization and selection based on function criticality (hence manually); 
%Shi et al.~\cite{shi2015comparing} combined and evaluated test reduction (based on coverage) and selection (based on changes). 
%To the best of our knowledge, \textit{\fz is the first regression test orchestration approach that combines file-based selection with similarity-based  prioritization.}

Clearly \fz is one instance within a plethora of possible combinations of many existing \tcs and \tcp approaches, and further studies should be conducted to evaluate  different combinations.
Indeed, following the case made by Harman~\cite{harman2011making},
research in combining multiple criteria in the context of one regression technique is very active, e.g.,~\cite{epitropakis2015empirical,garousi2018multi}.
Much less attention has been devoted so far to using multiple criteria while combining different regression techniques, which we see as an essential part of test suite orchestration.
Di Nardo et al.~\cite{di2015coverage} applied and assessed minimization, selection and prioritization techniques on a single industrial case study, but only considering coverage-based criteria;  Silva et al.~\cite{silva2016hybrid} proposed to combine prioritization and selection based on function criticality (assessed manually); Najafi et al.~\cite{najafi2019improving} evaluated selection and prioritization based on test execution history on a large industrial system;
Shi et al.~\cite{shi2015comparing} combined and evaluated test reduction (based on coverage) and selection (based on changes). 
\fz is the first regression test orchestration approach that combines file-based \tcs with similarity-based \tcp.

We compared \ek, \fs and their orchestration through \fz using a set of 12 projects (from the \dfj repository~\cite{just2014defects4j}).
Our results shows that for most subjects, executing a change-aware selection of test cases (in random ordering) detects the first failure faster than executing the whole prioritized suite (based on similarity). However, we also observed that adding \fs ordering on top of \ek selection further improves effectiveness at negligible additional cost.

To conclude this study and provide a direction for future evolution of test orchestration strategies, we provide a discussion on methods of incorporating other \tcs or \tcp techniques, along with potential combinations with \tsr and \tsa approaches.

\noindent
In summary, our contributions include:
\begin{itemize}
\item an empirical study comparing \tcs against \tcp, and their orchestration against each technique alone;
\item the novel \fz approach to regression testing that combines filed-based \tcs and similarity-based \tcp;
%\item the evaluation of \fz under limited test budgets;
%the effectiveness and time efficiency of two different approaches to combine \tcs and \tcp, also under decreasing test budgets; 
\item a replication package\footnote{Available at: \url{https://doi.org/10.5281/zenodo.5851288}} including \fz implementation and all data from the study.
%In this paper, we mark with~\textdagger~any sentence that refers to additional information found in the replication package.}.
%\footnote{Reference is omitted now to ensure anonymity, and will be made available with the CR.}.
\end{itemize}

For practitioners our results signify not only a further confirmation of change-aware selection validity, but also the convenience of executing the selected test cases in prioritized order based on their similarity.
In fact, using state-of-art scalable techniques as \fs over the selected test subset can help detect failures faster at virtually no cost.
For researchers, this paper signifies the importance of studying regression techniques as an orchestration rather than individually, and 
opens up the floor for many potential experiments in which various \tcs techniques are compared against, or combined with, various \tcp techniques. 

In~\Cref{sec:orch_background} we provide a short summary of the \tcs and \tcp approaches that we compare and combine, while in~\Cref{sec:orch_fastazi} we present \fz. 
The study methodology is described in~\Cref{sec:orch_experiments} and the results are discussed in~\Cref{sec:orch_results}.
%In Section~\ref{sec:orch_} we overview related work, 
Finally, in~\Cref{sec:orch_discussion} we draw brief conclusions derived from this study and discuss the possible next steps in the evolution of this work. 


\section{Research Questions}\label{sec:orch_rqs}

We evaluate \ek against \fs, and their combination (\fz) against either of them, considering first  their \textit{effectiveness} in  failure detection  (RQ1). Then, based on the real example shown in~\autoref{fig:fastazi}, we hypothesize that the potential gain in effectiveness of a combined approach could be better observed under a \textit{limited test budget} (RQ2). Finally we also compare their \textit{efficiency} (RQ3). Precisely,  we formulate the following research questions:

\paragraph{RQ\ref{chap:orchestration}.1: How do \ek, \fs, and \fz compare in terms of effectiveness?}
For the scope of this study, the comparison between the respective effectiveness of the three approaches can be based on how quick they are in detecting the failures.  As \fs uses the whole test suite, we know it will detect all regression failures as a retest-all technique.
Also, \ek is developed as a safe \tcs technique, thus it should, as well, detect all failures found by retest-all.
Consequently, \fz too detects all failures.
Thus, we refine the above question into the following two sub-questions:

\paragraph{RQ\ref{chap:orchestration}.1.1: Between \ek and \fs, which tool detects failures running fewer tests?}
While both \ek and \fs have been shown to be effective in failure  detection,  we do not  know whether when a new project version is released, potential regression failures would be revealed earlier by selecting those test cases that are affected by the changes (and randomly ordered) or instead by prioritizing test cases based on their similarity. 

\paragraph{RQ\ref{chap:orchestration}.1.2: How does \fz compare against \ek and \fs with respect to feedback time?} 
It is unclear if, and by how much, a combination of both techniques would provide lower feedback time from a test suite.
With this question, we aim to discover if the orchestration of \tcs and \tcp has a positive and substantive impact to the regression testing workflow.

\paragraph{RQ\ref{chap:orchestration}.2: How does a limited testing budget affect the effectiveness of the three approaches?}
While in RQs 1.1 and 1.2 we compared \ek, \fs, or \fz without considering possible time constraints, with this RQ we aim at assessing whether, and how, testing under limited resources impacts each of the three approaches. 
This problem is similar to cost-bounded selection \cite{cibulski2011regression} (i.e., selecting
test cases according to a predetermined budget), which can be a concern in large-scale industrial projects~\cite{elbaum2014techniques}. 
\tcs and \tcp each provide benefits when it is not possible to test 100\% of the test suite in each execution, but they cannot assumed to be safe in these circumstances.
Perhaps an orchestrated test suite would viable at even stricter testing budgets.

\paragraph{RQ\ref{chap:orchestration}.3: How do \ek, \fs and \fz compare in terms of time efficiency?}
With this question, we aim to discover what is the additional cost in terms of time required by either technique alone, and then by their orchestration.
% instead of just one, and what are possible ways to mitigate any increase. 
Inevitably, the orchestration increases total testing time, and we aim at assessing such drawback.


\section{Background}\label{sec:orch_background}

\subsection{\ek}\label{sec:orch_ekstazi}
\ek~\cite{gligoricEk} is a \textit{change-based} and \textit{coarse-grained} approach to \tcs.
It works by collecting test case dependencies (i.e., set of used classes by each test case) during an initial run of the entire test suite, then by selecting the test cases based on the changes applied to those dependencies from one version of the software to another.
In doing this \ek applies a \textit{file-level granularity}: any code changed within a file that is related to a test case will result in that test being selected.  To compare two versions of a file, \ek uses cyclic redundancy check (CRC).
For example, consider a test $t$ that invokes a function $a$.
If a change is made to another function $b$ located in the same file as $a$,
$t$ will be selected (as the CRC of the file changed).

The result of this approach is an over-approximation of the subset of selected tests.
Although \ek selects, on average, more tests than fine-grained \tcs solutions (e.g., those that track dependencies on methods), the authors demonstrated that the actual selection time is much faster than the alternatives. 
Consequently, the total end-to-end time (i.e., time to select tests + time to execute selected tests) tends to be lower, even if more tests are selected. 

We chose \ek for our study for its efficiency and ease of use: \ek is publicly available as a plug-in for various Java build systems.
Furthermore, aiming eventually at an orchestration of \tcs plus \tcp with the objective of reducing feedback time, we considered that 
a prioritized test suite could mitigate the drawbacks of over-selecting test cases.

\subsection{\fs}\label{sec:orch_fast}

\fs~\cite{miranda_fast_2018} utilizes test source code as input for a \emph{similarity-based algorithm} to prioritize the tests.
Inspired by big data techniques, string representations of test cases are transformed using minhashing signatures, which are then ordered according to their similarity.
The benefits of \fs are low overhead and scalability, which make it usable for large software projects.
We chose it because of its low running times and relatively simple implementation.

\fs authors~\cite{miranda_fast_2018} examined several possible variations of it
that trade off efficiency for accuracy when choosing the next test(s).
These are all stochastic by nature;
as the authors point out, if two test cases are ranked equally, the tie is solved randomly.
In our experiments with \fs we observed that \fs-pw (which is one of the variations) produced consistently similar permutations when executed more than once with the same test suite. 
This was an expected result given that \fs-pw is designed to always select the test case that is the furthest away from the set of already-prioritized tests.
It does so by computing the similarity between each candidate test and the set of already-prioritized tests in a pairwise fashion.
Furthermore, \fs-pw was able to rank failing tests  higher than other variations. 
Therefore, in this paper  we consider the \fs-pw variant, and in the following we refer to it simply as \fs.


\section{\fz}\label{sec:orch_fastazi}

Many researchers have shown that \tcs and \tcp  provide substantial benefits 
to regression testing~\cite{bin_ali_search_2019, kazmi_effective_2017, khatibsyarbini_test_2018, RothermelHarrold94FrameworkForEvaluationRTS}:
a good selection decreases the overall testing time, while a good prioritization allows for detecting failures faster.
However the two concepts are not mutually exclusive, and an orchestration of both may provide even further improvements, e.g.,~\cite{spieker_reinforcement_2017,elbaum2014techniques}.

If a test suite is selected \textit{and} prioritized, both testing time and feedback time can be decreased.
Recall that \tcs can be defined as a function $S(T)$ that produces a subset of tests and \tcp as a function $P(T)$ that outputs a permutation of $T$.
Then, the goal of an approach that orchestrates \tcs and \tcp is to generate another function, $O(T)$, 
whose output is smaller than $T$ and ordered to speed up failures detection.
When discussing possible ways of orchestrating \tcs and \tcp, two approaches stand out.

\paragraph{Parallel execution.} 
One approach is to independently perform the prioritization and selection of the entire test suite, and then arranging the selected tests according to the ordering given by prioritization.
This approach has the advantage of allowing parallel execution of $S(T)$ and $P(T)$ and merging their outputs, instead of having one depend upon the other.
To combine the outputs, it is sufficient to go through the prioritized list of tests and remove the ones that are not included in the selection.

\begin{figure}[t]
  \centering
    \includegraphics[width=0.8\linewidth]{figures/Fastazi}
    \begin{center}
  	\footnotesize
  	T: complete test suite; S: selection by \ek;\\ P: prioritization by \fs; O: orchestration by \fz. 
	\end{center}

    \caption{Sample outputs of \ek, \fs and \fz}
    \label{fig:fastazi}
\end{figure}


%While it could be possible to execute $S(P(T))$ sequentially, there is no practical benefit to that; the selection does not depend on the test ordering, and performing the intersection of the two isolated techniques is much faster than waiting for the prioritization to happen and then performing the selection.
%This happens because the output of $P(T)$ is the same size as $T$, and thus the running time for $S$ would be the same.

%is the same regardless of the input. 

%The first would be to select and prioritize test cases in isolation.
%In this case, $C$ is defined as a set containing the same tests as $S$, but ordered as they appear in $P$.
%This approach has the advantage of allowing us to run selection and prioritization in parallel and, although there is an additional combination step to perform, the cost of generating $C$ from $S$ and $P$ is negligible.
%We name this approach \fzp (P from ``parallel'' and also from ``prioritize, then select'').

\paragraph{Sequential execution.}
Another possible approach to the idea is performing selection first and then prioritizing the output.
The advantage of this approach is reducing the running time of the prioritization, which would focus on the tests impacted by the changes and thus more likely to fail.
However, this also means that the selection and prioritization steps cannot be performed simultaneously (although it is still possible to parallelize the preparation steps).
Intuitively, it is not clear which option should be more effective or efficient than the other.
Indeed, our experiments show that the effectiveness and efficiency of the parallel and sequential approaches are statistically equivalent (according to the same analysis detailed in \Cref{sec:orch_results}).
For lack of space, henceforth \fz results always refer to the sequential execution, while the results of the parallel combination are available in the replication package.

As an example, \Cref{fig:fastazi} contains sample outputs from \ek, \fs and \fz\footnote{This example is based on results of the experiments on \texttt{Chart v26}. Actual names of test cases are omitted for clarity.}.
Colored red, $t_{170}$ is the failing test within a test suite $T$ of 363 test cases.
In $S$, the output of \ek, this test is found in the 78th position, because several tests were excluded during the selection, while in the output $P$ of \fs, it is moved up to the 17th position.
Finally, the output of \fz, $O$, which is selected and prioritized, promoted the test to the 9th position.

\Cref{algo:fastazi} provides an abstract view of \ek and \fs\footnote{For a complete understanding of \ek and \fs, refer to~\cite{gligoricEk,miranda_fast_2018}.}, and outlines how \fz works in practice.
\ek requires tests to be compiled before performing selection, while \fs needs the hash signature of each test before prioritizing the suite.
These two steps are independent and can be performed in parallel (they are both abstracted by the function GetHashesAndModified).
After that, \ek can perform its selection normally, and \fs prioritizes the resulting list of tests.

\newcommand{\TSenc}{$T$}
%\newcommand{\P}{$P$}
\newcommand{\I}{$I$}
\newcommand{\CardP}{$|P|$}
\newcommand{\CardI}{$|I|$}
\newcommand{\MT}{$M$}
\newcommand{\MV}{$M$($v$)}
\newcommand{\MC}{$M$($c$)}
\newcommand{\V}{$v$}
\newcommand{\B}{$B$}
\newcommand{\Csim}{$C_{s}$}
\newcommand{\Cdis}{$C_{d}$}
%\newcommand{\C}{$C$}
%\newcommand{\SS}{$s$}
\newcommand{\f}{$f$}
%\newcommand{\cc}{$c$}
%\newcommand{\d}{$d$}
\newcommand{\e}{$e$}

\begin{algorithm}
\caption{\ek, \fs and \fz overview}
\label{algo:fastazi}
\begin{algorithmic}[1]

%\Function{UpdateHashes}{test suite $T$}
%	\State $M \gets $ \Call{ExistingHashes}{T}
%%	\State $N \gets$
%	\For{$t \in $ \Call{GetNewOrUpdatedTests}{T}} 
%		\State $M[t] \gets$ \Call{ComputeHash}{$t$}
%%		\If{$t$ is new or updated}
%%			\Call{ComputeHash}{$t$}
%%		\EndIf
%	\EndFor
%%	\State $D \gets$
%%	\For{$t \in $ \Call{GetDeletedTests}{T}} 
%%		\State $M[t] \gets $ \texttt{nil}
%%	\EndFor
%	\State \Return $M$ \Comment{Updated minhash signatures of all tests.}	
%\EndFunction

\Function{GetHashesAndModified}{files $F$}
%\Function{UpdateHashesAndGetModified}{files $F$}
        \State $M, C \gets $ \Call{ExistingHashes}{F}
        \State $M' \gets \emptyset$ \Comment{Minhashes for \fs and CRC for \ek}
		\State $F' \gets \emptyset$
	\For{$f \in F$}
		\State $M'[f] \gets$ \Call{ComputeHashes}{$f$}
%		\If{$t$ is new or updated}
%			\Call{ComputeHash}{$t$}
%		\EndIf
            \If{$M[f] \neq M'[f]$}
                \State \Call{Append}{$F'$, $f$}
            \EndIf
        \EndFor
%	\State $D \gets$
	%% \For{$t \in $ \Call{GetDeletedTests}{T}} 
	%% 	\State $M[t] \gets $ \texttt{nil}
	%% \EndFor
	\State \Return $M', F'$ \Comment{Updated hashes and modified files.}	
\EndFunction

\Function{\ek}{test suite $T$, files $F$}
        \State $S \gets \emptyset$
        \For{$f \in F$}
            \For{$t \in $ $T$}
                \If{\Call{TestDependsOn}{$t$, $f$}}
                    \State \Call{Append}{$S$, $t$}
                \EndIf
            \EndFor
        \EndFor
        \State \Return $S$ \Comment{A selected test suite.}
\EndFunction

\Function{\fs}{test suite $T$, hashes $M$}
	\State $P \gets \emptyset$
%	\State $B \gets $ \Call{LSHBuckets}{$M$}
%	\State \MV $\gets$ \Call{MHSignature}{$\emptyset$} \Comment{\MV : Cumulative signature of so-far-ordered test cases}
	\While{$|P| \neq |T|$}
%		\State \Csim $\gets$ furthest test cases from what is already prioritized %\Call{LSHCandidates}{\B, \MV}
%		\If{\Csim $= \emptyset$}
%			\State \MV $\gets$ \Call{MHSignature}{$\emptyset$}
%			\State \Csim $\gets$ \Call{LSHCandidates}{\B, \MV}
%		\EndIf
%		\State \Cdis $\gets$ ($I - P -$ \Csim) \Comment{Complement of \Csim}
%		\State $S$ $\gets$ \Call{Select}{\MV, \MT, \Cdis, \f}
		\State $t$ $\gets$ \Call{PickNextTest}{$T$, $P$, $M$} \Comment{Pick the test that is furthest away from the so-far-ordered tests $P$ based on $M$.}
%		\State \MV $\gets$ \Call{UpdateMHSignature}{\MV, \MT, $S$}
%		\State \MT $\gets$ \Call{Remove}{\MT, $S$}
		\State $P$ $\gets$ \Call{Append}{$P$, $t$}
	\EndWhile
	\State \Return $P$ \Comment{A prioritized test suite.}
\EndFunction


\Function{\fz}{test suite $T$, files in the project $F$}
	\State \Call{Compile}{$T$} $\bullet$ $M, F \gets $ \Call{GetHashesAndModified}{$F$} \Comment{Compiles the test suite using the build system and, in parallel, computes hashes and detects modified files.}
	\State $S \gets $\Call{\ek}{$T$, $F$}
	\State $P \gets $\Call{\fs}{$S$, $M$}
	\State \Return $P$ \Comment{A selected and prioritized test suite.}
\EndFunction


%\State clone $v_1$...$v_n$ from \dfj
%\State compile $v_1$
%\State $P_1\gets P(T)$
%\State $i\gets 1$
%\While {$i\geq n$}
%	\State copy \ek and \fs metadata from $v_{i-1}$ to $v_i$
%	\State compile $v_i$
%	\State $S_i\gets S(T_i)$
%	\State $P_i\gets P(T_i)$
%	\State $C_{Pi}\gets P_i \cap S_i$
%	\State $C_{Si}\gets P(S_i)$
%	\State test $v_i$
%\EndWhile

\end{algorithmic}
\end{algorithm}


%In the examples from the figures, the outputs of \fzp and \fzs are the same, which can happen with a real test suite, but there is no such guarantee.
%From now, we use the generic name \fz when referring to the combination without distinguishing between the two flows, whereas we will specify \fzp or \fzs when the combination flow is relevant. 
%\anto{general comment to be applied throughout in last revisions: should all tool names be using same format? }

%The other alternative is to first select test cases, and then prioritize only the selected subset of tests.
%That is, $C$ still contains the same tests as $S$, but the prioritization is defined in terms of $S$ rather than $T$.
%The advantage in this case would be to avoid spending resources prioritizing test cases that are deemed irrelevant by \tcs, with the disadvantage that the \tcs and \tcp steps cannot be performed simultaneously.
%This approach is named \fzs (S from ``sequential'' or from ``select, then prioritize'').

% \anto{added previous notice; also should all names \fz \fs \ek use same type of fonts as \fzp?} 
%Intuitively, it is not clear whether \fzp should be more effective than \fzs or vice-versa.
%While the selection step is identical between the two, the prioritization can change substantially.
%On the one hand, \fzp allows the \tcp algorithm to leverage data from the entire test suite.
%In our experiment, since \fs uses a similarity-based prioritization algorithm, it means that the output of \fzp tends to be highly diverse.
%On the other hand, utilizing information from tests that would ultimately be excluded from execution by \tcp could be detrimental.
%For example, if two test cases, $t_1$ and $t_2$, are similar according to the measures used by \fs, but only $t_1$ is selected and detects the failures, there is a chance that $t_2$ will be prioritized highly and $t_1$ will be given a lower priority than other, less relevant, tests.
%In this situation, \fzs ensures that the prioritization considers only tests that will actually be executed and avoids odd situations such as the one described.
%The results of our experiments regarding effectiveness can be found in \Cref{subsec:rq1}.

%Similarly, there is no obvious reason why \fzp should be more efficient than \fzs or the other way around.
%Both of them add costs beyond simply using \ek or \fs, but in both cases these costs can be mitigated.
%\fzp does so by allowing the parallel execution of both approaches, and adds only a small combination step beyond whichever approach takes the longest.
%Meanwhile, \fzs has the advantage of reducing the prioritization time by using only the selected tests as input.
%Furthermore, regarding \fs, the tool requires a preparation step that comes before the prioritization itself, and this step can be executed simultaneously with \tcs.
%So the additional cost of \fzs is only the time to prioritize the selected tests.
%The results of our efficiency analysis of the tools is found in \Cref{subsec:rq3}.

It is important to observe that, while we utilize \ek and \fs as representative implementations of \tcs and \tcp, the idea behind \fz could be attempted using different approaches.
There are many proposed techniques that address \tcs and \tcp; combining different ones would inevitably result in changes to effectiveness and efficiency.


\section{Experiments}\label{sec:orch_experiments}

\subsection{Evaluation metrics}
\label{subsec:metrics}

The primary objective of \tcs is to reduce the total number of tests executed per run, while \tcp, on the other hand, 
has the goal of detecting failures quickly and reduce the feedback time of the test suite.
Thus, the metric for an orchestration should somehow measure both of these objectives.

For \textbf{RQ1}, we utilize a metric called \textit{Time To First Failure} (\ttff) \cite{yoo2011faster}.
Given a test suite $T$, its \ttff indicates the position of the first test to detect a failure.
A low \ttff indicates that the test suite provides quick feedback.
\ttff is a useful metric to evaluate both \tcs and \tcp, because it simultaneously encourages a tight selection of truly relevant tests and a prioritization that puts a failing test at the top of the list.
However, since the output $S$ of \tcs is a subset of $T$, its size might be smaller than the output $P$ of \tcp.
Therefore, for fairness, all \ttff results in this paper are normalized according to size of $T$.
For example, if $|T| = |P| = 1000$, $|S| = 100$ and a failing test is in the 100th position of $P$ but the the 50th position of $S$, then $TTFF(P) = 0.10$ and $TTFF(S) = 0.05$.

We also utilize \textit{Average Percentage of Faults Detected} (\apfd), the most popular metric for evaluating \tcp solutions \cite{khatibsyarbini_test_2018}.
It is not designed to evaluate \tcs and thus may not provide a fair comparison for \ek; however, as previously explained, we assess here effectiveness in terms of how fast failures are detected by the compared techniques, and for this \apfd provides an intuitive, well known assessment.  

Regarding \textbf{RQ2}, when considering a limited testing budget, we use the output from \textbf{RQ1} and create versions of the suites that are cut off at certain points, according to the budget restriction.
This data is analyzed in two ways: first, we observe, for each version of each subject, the proportion of the 30 variations that were able to detect the failure or not.
Then, we also reduce this number into a binary form: 1 if the test suite detects the failure in all of its 30 variations, and 0 otherwise.
This has the effect of punishing suites that are somehow inconsistent, rewarding those that catch the failure every time, 
since it can be
important that an approach is consistent and reliable.

Finally, for \textbf{RQ3}, we measure the time taken to execute the discrete steps of the approach.
For this, we use the GNU \texttt{time} utility (user+sys CPU time) to measure each step of the experiment individually, allowing us to understand where are the bottlenecks of the approaches.

\subsection{Experiment design and execution}
\label{subsec:experiment}

%\begin{table}[!t]
  \centering
  \begin{tabular}{lrrr}
    \toprule
    Subject & \# Versions & Min. \# Tests & Max. \# Tests \\
    \midrule
    Chart & 26 & 303 & 363 \\
    Cli & 30 & 24 & 85 \\
    Closure & 168 & 236 & 258 \\
    Codec & 8 & 34 & 52 \\
    Collections & 4 & 157 & 165 \\
    Compress & 39 & 44 & 133 \\
    Gson & 18 & 77 & 119 \\
    Jsoup & 93 & 12 & 39 \\
    JxPath & 4 & 27 & 33 \\
    Lang & 28 & 87 & 178 \\
    Math & 100 & 137 & 821 \\
    Time & 23 & 121 & 123 \\
    \midrule
    Total & 541 & n/a & n/a\\
    \bottomrule
  \end{tabular}
  \begin{flushleft}
  \footnotesize
%  The subjects have a varying number of test cases due to the multiple versions used. 
  Min. \# Tests and Max. \# Tests show the smallest and largest test suites, respectively, among all versions of a certain subject.
  \end{flushleft}
  \caption{Subjects Used in the Evaluation.}
%  \vspace{-3mm}
  \label{tab:projects}
\end{table}

%\renan{\autoref{fig:projects}: Number of versions instead of range.}
%
%\renan{\autoref{fig:projects}: Add total number of versions.}


The goal of the experiment is to compare four possible arrangements of the test suite: 
the tests selected by \ek; the test suite prioritized by \fs; the orchestration of both with \fz; and a random ordering of the test suite to provide a base case.
Considering that both \ek and \fs have been previously compared to several competing \tcs and \tcp approaches~\cite{legunsen2016, Zhang18HybridRTS,miranda_fast}, we deemed it not necessary to add further alternatives in a direct comparison between the two tools.

We utilize as subjects 12 projects available as part of the \dfj repository~\cite{just2014defects4j} that contains multiple versions of Java-based open-source software projects of different sizes.
Each version is comprised of one commit containing a bug, the commit that fixed the bug, and metadata such as the files related to the bug, and which tests would detect it.
Table~\ref{tab:projects} shows basic statistics about each project
used in our evaluation.
For each project, we show the number of versions used and minimum and maximum number of test cases (across versions).
A few versions were skipped, either because their bugs are listed as deprecated by \dfj, or because we ran into compilation issues for them (e.g., due to Java version incompatibility).

We used \ek version 5.3, available on the project's website\footnote{\url{http://ekstazi.org}}, as a plug-in for the Maven and Ant build systems.
A script is used to automatically incorporate the \ek task into a project's build script, allowing us to easily perform test selection over multiple versions of different subjects.

In the case of \fs, we used the source code from the replication package of the original paper\footnote{\url{https://github.com/icse18-FAST/FAST}}.
This code was modified by us with two purposes.
The first was to make \fs version-aware by storing the hash signatures of test cases between versions so they do not need to be re-computed unless there is a modification.
This is important because computing the hashes is the most time-consuming part of \fs, so storing these representations for unchanged tests greatly reduces overhead after an initial execution.
In addition, it was updated to guarantee that the input and output of both \ek and \fs are in the same format.

\fz was not incorporated into the build system, but its results can be easily generated by using the output of \ek as input for \fz, as shown in Algorithm \autoref{algo:fastazi}.
Observe that change-based \tcs provides no benefit in the initial version of a project, since there are no changes to be detected; thus the first output of \fz, for each experiment subject, is identical to using \fs in isolation.

To collect the metrics, we did not actually execute the test suites given by each approach. First we collect the outputs of the approaches as text files containing lists of tests and then we calculate the metrics according to the position of the failing test(s) (ground truth given by \dfj).

When measuring TTFF, the default order of test executions could have a large impact on (unprioritized) test suites; hence, for fairness we shuffled the output of \ek 30 times and reported the average of these repetitions.
Similarly, to account for the nondeterministic behavior of \fs, \fz and random, their outputs are also generated 30 times to reduce any potential noise in the data\footnote{We experimented with values between 10 and 50 and found that 30 provided a good amount of data without severely impacting the running time.}.

The experiments were executed in a Docker container running Ubuntu 20.04 LTS, using Java OpenJDK 1.8.0, Apache Maven 3.6.3, and Apache Ant 1.10.7.
On all the projects, JUnit version was set to 4.12.
The host computer was running macOS 11.0.1 on a 6-core Intel Core i7 processor, with 32GB RAM and SSD storage.


\section{Results}\label{sec:orch_results}

\subsection{RQ1: Effectiveness}
\label{subsec:rq1}

The answer to \textbf{RQ1} contains two parts: first, we compare the effectiveness of \ek and \fs against each other (\textbf{RQ1.1}), and then, we assess
whether orchestration TCS and TCP ultimately improves effectiveness (\textbf{RQ1.2}). 
For the sake of space we show the results for both subquestions within unified plots and tables.

%\begin{figure}[t]
%  \centering
%  \includegraphics[width=\linewidth]{figures/pTTFF-avg}
%  \caption{Normalized \ttff of different approaches}
%  \label{fig:ttff_avg}
%  \vspace{-3mm}
%\end{figure}

The \pttff results 
are displayed as violin plots in Figure~\ref{fig:ttff_avg}, in which each version of each subject is one data point (totaling 541).
The violin plots display, in addition to the median and interquartile ranges, the full distribution of the data, 
which allows us to identify the different peaks in a distribution. 
For the \pttff metric, the lower the result, the better.

The visual assessment of the data shows us that the median \pttff achieved by \ek and \fs are both close to 45\% (the two leftmost plots in Figure~\ref{fig:ttff_avg}), although there is a large difference in the distribution of the results.
This can be explained in part by the experiment design -- since \ek's \pttff is an average of 30 permutations of $S$, the value tends to be close to the center.
Indeed, we can see that the median for \rnd is very close to 50\%, while \ek is lower than that because $S$ is frequently smaller than $T$.

When adding \fz to the comparison, we can see that its median \pttff is much lower, at around 25\%, which is slightly over half the medians of \ek and \fs.
Both \fs and \fz can, in some instances, produce a \pttff close to 100\%, meaning that the failing test is found at the very end of the test suite.
In the case of \fs, this is explained by the fact that similarity-based \tcp can occasionally produce poor results if there are multiple similar test cases out of which only one reveals the failure.

With \fz, this happens less frequently; when it does, it is caused by performance of both \ek (selecting nearly 100\% of the test suite) and \fs (ranking the failing test low) in specific subject versions.

After the visual inspection we proceeded with the statistical analysis of the data. 
As we could not assume our data to be normally distributed, we adopted a non-parametric statistical hypothesis test, 
the Kruskal-Wallis rank sum test\footnote{We used \texttt{kruskal.test()} from the \texttt{Stats} package in \texttt{R}.}.
We assessed at a significance level of 5\% the null hypothesis that the differences in the \ttff values are not statistically significant.
The observed differences in \ttff were statistically significant at least at the 95\% confidence level 
(\textit{p-value} $<$ 2.2e-16).

Provided that significant differences were detected by the
Kruskal-Wallis test we performed pairwise comparisons to
determine which approaches are different\footnote{A significant Kruskal-Wallis test indicates that 
there is a significant difference between approaches, 
but does not identify which pairs of approaches are different.}. 
The results are displayed in Table~\ref{tab:ttff_apfd} (column \textit{Group} for TTFF).
If two approaches have different letters they are significantly different
(with $\alpha = 0.05$). If, on the other hand, they share the same letter, the
difference between their ranks is not statistically significant. 
The approach (or group of approaches) that yields the best performance is
assigned to the group (a).
Looking at the results in Table~\ref{tab:ttff_apfd}, we can tell that
\fz is different from (better than) \ekr (b).
\ekr, on its turn, is different from (better than) \fs (c), 
and all the approaches are different from (better than) \rnd (d).

%\begin{table}[!hbt]
\centering
\begin{tabular}{lrrcrrc}
\toprule
\multicolumn{1}{l}{\multirow{2}{*}{Approach}} &
  \multicolumn{3}{c}{TTFF} &
  \multicolumn{3}{c}{APFD} \\ %\cmidrule(l){2-7} 
\multicolumn{1}{c}{} &
  \multicolumn{1}{c}{\textit{Med}} &
  \multicolumn{1}{c}{\textit{SD}} &
  \multicolumn{1}{c}{\textit{Group}} &
  \multicolumn{1}{c}{\textit{Med}} &
  \multicolumn{1}{c}{\textit{SD}} &
  \multicolumn{1}{c}{\textit{Group}} \\ %\cmidrule(r){1-1}
\midrule
\fz      & 0.25 & 0.27 & (a) & 0.75 & 0.27 & (a) \\
%\fzp      & 0.26 & 0.27 & (a) & 0.74 & 0.27 & (a) \\
\ekr & 0.39 & 0.14 & (b) & 0.62 & 0.14 & (b) \\
\fs        & 0.41 & 0.29 & (c) & 0.60 & 0.29 & (c) \\
Random         & 0.49 & 0.09 & (d) & 0.51 & 0.09 & (d) \\ \bottomrule
\end{tabular}
\label{tab:ttff_apfd}
\begin{flushleft}
\footnotesize
%\scriptsize 
\textit{Med} is the median, \textit{SD} is the standard deviation, and \textit{Group} displays the result for the pairwise comparisons after the Kruskal-Wallis test. 
\end{flushleft}
\caption{TTFF and APFD for the different approaches.}
%\vspace{-3mm}
\end{table}

%
%%\begin{table}[!hbt]
%\centering
%\caption{TTFF and APFD for the different approaches}
%\begin{tabular}{lrr|crrc}
%\toprule
%\multicolumn{1}{l}{\multirow{2}{*}{}} &
%  \multicolumn{2}{c}{TTFF} &
%  \multicolumn{2}{c}{APFD} \\ %\cmidrule(l){2-7} 
%\multicolumn{1}{c}{} &
%  \multicolumn{1}{c}{\textit{$\hat{A}_{12}$}} &
%  \multicolumn{1}{c}{\textit{Qualitative}} &
%  \multicolumn{1}{c}{\textit{$\hat{A}_{12}$}} &
%  \multicolumn{1}{c}{\textit{Qualitative}} \\ %\cmidrule(r){1-1}
%\midrule
%\ek vs. \fs  & 0.56 & Negligible & 0.56 & Negligible \\
%\fz vs. \ek  & 0.59 & Small 	 & 0.59 & Small \\
%\fz vs. \fs  & 0.61 & Small 	 & 0.61 & Small \\
%%Random         & 0.49 & 0.09 & (d) & 0.51 & 0.09 & (d) \\ 
%\bottomrule
%\end{tabular}
%\label{tab:effectsize}
%\begin{flushleft}
%\footnotesize
%%\scriptsize 
%\textit{$\hat{A}_{12}$} is effect size value, \textit{Qualitative} is the description of the value. 
%\end{flushleft}
%\end{table}

% Please add the following required packages to your document preamble:
% \usepackage[table,xcdraw]{xcolor}
% If you use beamer only pass "xcolor=table" option, i.e. \documentclass[xcolor=table]{beamer}
%\begin{table}[]
%\caption{Effect size per subject}
%\begin{tabular}{lllll}
%\toprule
%\multicolumn{1}{l}{\multirow{2}{*}{Subject}} &
%  \multicolumn{1}{c}{\fz vs} &
%  \multicolumn{1}{c}{\fz vs} &
%  \multicolumn{1}{c}{\fz vs} &
%  \multicolumn{1}{c}{\ek vs} \\ %\cmidrule(l){2-7} 
%\multicolumn{1}{c}{} &
%  \multicolumn{1}{c}{\rnd} &
%  \multicolumn{1}{c}{\fs} &
%  \multicolumn{1}{c}{\ek} &
%  \multicolumn{1}{c}{\fs} \\ %\cmidrule(r){1-1}
%%                 & \textbf{\fz vs\\ \rnd} & \textbf{\fz vs\\ \fs} & \textbf{\fz vs \ek} & \textbf{\ek vs \fs}    \\
%\midrule
%Chart            & \cellcolor[HTML]{A4C2F4}L    & \cellcolor[HTML]{A4C2F4}L  & \cellcolor[HTML]{FFE599}S     & \cellcolor[HTML]{A4C2F4}L   \\
%Cli              & \cellcolor[HTML]{A4C2F4}L    & \cellcolor[HTML]{FFE599}S  & \cellcolor[HTML]{A4C2F4}L     & \cellcolor[HTML]{EA9999} \\
%Closure & \cellcolor[HTML]{FFE599}S    & \cellcolor[HTML]{F9CB9C}N  & \cellcolor[HTML]{F9CB9C}N     & \cellcolor[HTML]{F9CB9C}N   \\
%Codec            & \cellcolor[HTML]{A4C2F4}L    & \cellcolor[HTML]{B6D7A8}M  & \cellcolor[HTML]{B6D7A8}M     & \cellcolor[HTML]{F9CB9C}N   \\
%Collections      & \cellcolor[HTML]{F9CB9C}N    & \cellcolor[HTML]{B6D7A8}M  & \cellcolor[HTML]{EA9999}   & \cellcolor[HTML]{FFE599}S   \\
%Compress         & \cellcolor[HTML]{A4C2F4}L    & \cellcolor[HTML]{B6D7A8}M  & \cellcolor[HTML]{FFE599}S     & \cellcolor[HTML]{FFE599}S   \\
%Gson             & \cellcolor[HTML]{B6D7A8}M    & \cellcolor[HTML]{FFE599}S  & \cellcolor[HTML]{FFE599}S     & \cellcolor[HTML]{EA9999} \\
%Jsoup            & \cellcolor[HTML]{FFE599}S    & \cellcolor[HTML]{B6D7A8}M  & \cellcolor[HTML]{F9CB9C}N     & \cellcolor[HTML]{B6D7A8}M   \\
%JxPath           & \cellcolor[HTML]{F9CB9C}N    & \cellcolor[HTML]{FFE599}S  & \cellcolor[HTML]{EA9999}   & \cellcolor[HTML]{FFE599}S   \\
%Lang             & \cellcolor[HTML]{B6D7A8}M    & \cellcolor[HTML]{B6D7A8}M  & \cellcolor[HTML]{B6D7A8}M     & \cellcolor[HTML]{FFE599}S   \\
%Math             & \cellcolor[HTML]{A4C2F4}L    & \cellcolor[HTML]{B6D7A8}M  & \cellcolor[HTML]{B6D7A8}M     & \cellcolor[HTML]{FFE599}S   \\
%Time             & \cellcolor[HTML]{FFE599}S    & \cellcolor[HTML]{FFE599}S  & \cellcolor[HTML]{F9CB9C}N     & \cellcolor[HTML]{FFE599}S \\
%\bottomrule 
%\end{tabular}
%\label{tab:effectsize}
%\begin{flushleft}
%\footnotesize
%L, M, S and N indicate Large, Medium, Small and Negligible effect size, respectively. Empty cells indicate cases where the effect size was below 0.5. 
%\end{flushleft}
%\end{table}

\begin{table}[]
\centering
\begin{tabular}{lllll}
\toprule
\multicolumn{1}{l}{\multirow{2}{*}{Subject}} &
  \multicolumn{1}{c}{\fz vs} &
  \multicolumn{1}{c}{\fz vs} &
  \multicolumn{1}{c}{\fz vs} &
  \multicolumn{1}{c}{\ek vs} \\ %\cmidrule(l){2-7} 
\multicolumn{1}{c}{} &
  \multicolumn{1}{c}{\rnd} &
  \multicolumn{1}{c}{\fs} &
  \multicolumn{1}{c}{\ek} &
  \multicolumn{1}{c}{\fs} \\ %\cmidrule(r){1-1}
\midrule
Chart       & 0.82 (L)                     & 0.79 (L)                   & 0.57 (S)                      & 0.78 (L)                 \\
Cli         & 0.85 (L)                     & 0.56 (S)                   & 0.81 (L)                      & 0.23 (L)                 \\
Closure     & 0.62 (S)                     & 0.55 (N)                   & 0.56 (S)                      & 0.51 (N)                 \\
Codec       & 0.88 (L)                     & 0.66 (M)                   & 0.66 (M)                      & 0.52 (N)                 \\
Collections & 0.50 (N)                     & 0.66 (M)                   & 0.44 (N)                      & 0.63 (S)                 \\
Compress    & 0.82 (L)                     & 0.65 (M)                   & 0.59 (S)                      & 0.58 (S)                 \\
Gson        & 0.69 (M)                     & 0.58 (S)                   & 0.60 (S)                      & 0.47 (N)                 \\
Jsoup       & 0.63 (S)                     & 0.64 (M)                   & 0.51 (N)                      & 0.66 (M)                 \\
JxPath      & 0.50 (N)                     & 0.56 (S)                   & 0.44 (N)                      & 0.57 (S)                 \\
Lang        & 0.66 (M)                     & 0.63 (S)                   & 0.65 (M)                      & 0.58 (S)                 \\
Math        & 0.83 (L)                     & 0.66 (M)                   & 0.64 (M)                      & 0.57 (S)                 \\
Time        & 0.60 (S)                     & 0.61 (S)                   & 0.52 (N)                      & 0.61 (S)                 \\

\bottomrule 
\end{tabular}
\label{tab:effectsize}
\begin{flushleft}
\footnotesize
L, M, S and N indicate large, medium, small and negligible effect size, respectively.
\end{flushleft}
\caption{Effect size per subject.}
%\vspace{-3mm}
\end{table}


To understand the effect of choosing one technique over another on the effectiveness of the test suite, we measured the effect size using the Vargha and Delaney $\hat{A}_{12}$ measure~\citep{vargha2000critique}, which tells us the probability of an observation from one group being larger than an observation from the other group.
The results are displayed in \Cref{tab:effectsize}. For interpreting the results, the $\hat{A}_{12}$ measure ranges from 0 to 1, and when the measure is exactly 0.5 the two techniques (in the column name) have equal performance. When $\hat{A}_{12}>0.5$, the first technique outperforms the second, and when $\hat{A}_{12}<0.5$, the second technique outperforms the first.
Vargha and Delaney suggest that the effect size is \textit{small} if the measure is over 0.56, \textit{medium} if over 0.64, and \textit{large} if the measure is over 0.71. As an example, when comparing \fz against \rnd for the subject Chart, \fz outperforms \rnd with a \textit{large} effect ($\hat{A}_{12}=0.82$) on the testing effectiveness.  
%
We can see that \ek generally outperforms \fs, most of the time with a negligible or small effect, but there are cases where \fs outperforms \ek.
\fz, on its turn, outperforms \ek and \fs with a non-negligible effect in the vast majority of the cases (18 out 24). The effect of choosing \fz over \ek or \fs on the test effectiveness is large or medium in 11 cases.


While \ttff captures how many test cases are required to reveal the first failures, 
the \apfd metric measures the speed at which failures are revealed.

%\begin{figure}[h]
%  \centering
%  \includegraphics[width=\linewidth]{figures/APFD-avg}
%  \caption{\apfd of different approaches.}
%  \label{fig:apfd_avg}
%  \vspace{-3mm}
%\end{figure}
The observed \apfd results are displayed as violin plots in Figure~\ref{fig:apfd_avg}.
For the \apfd metric, the higher
the better.
Visual assessment of the results lead to the same conclusion as for \pttff:
\ek and \fs have similar medians, although \fs sometimes performs very poorly, while \fz has a higher median than both and mitigates most instances of poor performance from \fs.
It is also visible that the peak of the distribution of \fz leans towards the highest possible values, while \ek peaks at around 0.6.

Statistical analysis results are reported in Table~\ref{tab:ttff_apfd} (right side). 
We performed again the Kruskal-Wallis rank sum test, followed by the pairwise multiple
comparisons.
All  results in Table~\ref{tab:ttff_apfd} are statistically
significant at the 5\% significance level.
Both the groups assigned to each approach 
and the results of the effect size analysis were the same as the ones observed for the \ttff metric.


\begin{tcolorbox}%[boxsep=0mm,boxrule=0mm,size=minimal]
\textbf{Summary of RQ1}: While statistically significant differences were observed for the comparison between \ek and \fs, a further investigation of the effect size revealed that the effect of choosing \ek over \fs is either small or negligible in almost all the cases.
\fz, on the other hand, outperformed \ek and \fs with a non-negligible effect in the vast majority of the cases, suggesting that adopting \fz can help improving the testing effectiveness.
\end{tcolorbox}


\subsection{RQ2: Effectiveness under a limited budget}
\label{subsec:rq2}

To answer RQ2, we proceeded with a detailed analysis of the impact of limiting the number of test cases with respect to those that would be run by \ek. 
We investigated the impact on the failure detection capability of all the approaches 
when the testing budget is gradually reduced from 100\% (no budget restrictions) 
%to 10\% of the test suite selected by \ek, at steps of 10\%.
to 25\% of the test suite selected by \ek, at steps of 25\%.
%
We discuss our findings first at a higher level, then with a more in-depth analysis of the results for each of the subjects considered in our study.

%\begin{figure*}[h]
%  \centering
%  \includegraphics[width=\linewidth]{figures/HitCount-PerBudget-box-plot-4.pdf}
%  \begin{flushleft}
%	\footnotesize
%	Each panel represents a different budget constraint (100\% is defined as the percentage of the test size selected by \ek).
%	The vertical axis shows how many times, out of the 30 repetitions, each approach is able to reveal the failure.
%  \end{flushleft}
%  \caption{Impact on failure detection capability in a budget-constrained scenario.}
%  \label{fig:perbudget}
%\end{figure*}

Figure~\ref{fig:perbudget} depicts the impact on failure detection capability on the different approaches.
The results are grouped per budget 
(25\% to 100\%) 
and each approach is represented by a violin plot.
For each version of each subject 
we counted how many times, out of the 30 repetitions (see~\Cref{subsec:metrics})
each approach would be able to reveal the failure under the different budget restrictions 
(the number of observation in each violin plot is thus the same as the total number of versions, i.e., 541).
The vertical axis varies from 0 to 30, respectively the minimum and maximum number of times an approach could reveal the failure across the 30 repetitions.
Notice that for this RQ it is not a concern whether the failure is revealed by the first or the last test case, 
as this was already answered by RQ1; the  concern here is whether the failure is revealed.

We can draw several observations from Figure~\ref{fig:perbudget}:
\textit{i}) the median number of times the random approach can reveal the failure decreases almost uniformly as the budget becomes  stricter;
\textit{ii}) because \ekr is the result of \ek selection with random ordering, the observed medians and distributions 
are always slightly better than random, but following a similar trend as the one observed for random;
\textit{iii}) \fz outperforms the other approaches up to a budget restriction of 50\%;
\textit{iv}) for the more restrictive budget of 25\% the median of \ekr and even random are better than those of the \fz approach.
Looking at the shape of the violin plots, however, we can see that even with a lower median  \fz  appears to have more observations leaning towards the maximum possible value.

%\begin{figure*}[h]
%  \centering
%  \includegraphics[width=\linewidth]{figures/PercentageFaultsPerBudgetPerSubject-ekstazi-10.pdf}
%  \begin{flushleft}
%	\footnotesize
%	The vertical axes represent the number of failures revealed in absolute (left) and in relative terms (right),
%	whereas the horizontal axes show the budgets w.r.t the number of tests selected by \ek (bottom) and w.r.t the total number of tests in the subject's test suite (top).
%  \end{flushleft}
%  \caption{Impact on failure detection capability grouped by subject and by budget.}
%  \label{fig:persubject}
%\end{figure*}

To better understand such a behavior we analyze the data again from a different perspective in Figure~\ref{fig:persubject}, in which 
we observe the impact on failure detection capability on a per subject basis.
This time, however, instead of counting how many times the failure would be revealed across the 30 repetitions,
we are interested in the cases where the approach would consistently reveal the failure across all the repetitions for a given version.
In this way we do not reward the cases where an approach would be able to reveal a failure by pure chance. 
Each subject is represented by a grouped bar plot and the height of each bar
represents the number of times the approach was able to consistently reveal the failure, 
both in absolute (left vertical axis) and in relative terms (right vertical axis).
For example, the maximum value in the left vertical axis for \texttt{Closure} is 168, 
which is the number of versions we considered for that subject and, at the same time, the maximum number of failures that can be revealed (one per version).

The primary horizontal axis (bottom) represents the budgets, from 10\% to 100\%,
whereas the secondary horizontal axis (top) shows what a given budget restriction would mean with regards to the whole test suite.
This is important because the size of the test suite varies greatly across the subjects.
For example, while a budget restriction of 50\% for \texttt{Collections} means that 45\% of the whole test suite is selected, 
only 23\% of the whole test suite would be selected for \texttt{Chart} under the same budget restrictions
(we recall that the budget restriction is calculated over the size of the test subset selected by \ek).


By analyzing Figure~\ref{fig:persubject} we can draw the following observations:
\textit{i}) with no budget restrictions (budget = 100\%), \ekr and \fz were able to consistently reveal all the failures across the 30 repetitions;
\textit{ii}) for any other budget value below 100\% \fz outperformed \ek alone and \fs alone --- in a very few cases \fs appears tied to \fz;
\textit{iii}) \ekr can consistently reveal some failures for almost all the budgets for \texttt{Chart}. 
For all the other subjects, it cannot reveal any failure for budgets restricted below 50\%.
For the particular cases of \texttt{Collections} and \texttt{Lang}, \ekr cannot reveal any failure consistently in the constrained budget scenario;
\textit{iv}) with the exception of \texttt{Codec}, \texttt{Collections}, and \texttt{JxPath}, 
\fz was able to consistently reveal some failures across the 30 repetitions for all the budgets, including the more restrictive budget of 10\%.

\begin{tcolorbox}%[boxsep=0mm,boxrule=0mm,size=minimal]
\textbf{Summary of RQ2}: 
Without controlling for the differences across subjects, 
\fz exposes the best failure detection capability even under restricted budgets, 
except for under 25\% reductions in which \ekr and even random appear to show better median values.
However, when we look from a per subject perspective and reward the approaches that consistently reveal failures,
\fz outperform \ek alone (with random ordering) and \fs alone (without \tcs) for all the budgets considered.
\end{tcolorbox}



\subsection{RQ3: Efficiency comparison}
\label{subsec:rq3}
To compare the time efficiency of \ek, \fs, and \fz, 
we isolated the individual steps of each approach and measured the average time each step took, across the different versions of each subject program. 
In our measures, displayed in \Cref{tab:execution_time}, 
the average build time (column~2) for each project was substantially longer than any cost added by \ek, \fs, or \fz.
This is an important observation because \fs can run its preparation phase (column~3), i.e, computing hashes of added/modified test cases, in parallel with the building process as it requires only test code.
\fz takes advantage of this aspect to minimize the time overhead. %by running \fs preparation in parallel with the building process.
\ek, on the other hand, requires the code to be compiled before it can perform selection, so it cannot be run in parallel with the build.

%%\begin{table}
%  \caption{Average execution time of each step.\vspace{-10pt}}
%%  \scriptsize
%  \begin{tabular}{|lrrrrrr|}
%    \hline
%    \textbf{Project} & Appr. & \tcs & Prep. & \tcp & Comb. & Total \\
%    \hline
%    \hline
%    \multirow{2}*{Chart}
%    	& P & 3.2244s & 1.1048s & 0.1121s & 0.0009s & 3.7759s \\
%        & S & 3.2244s & 1.1048s & 0.0353s & N/A & 3.7905s \\
%%    \hline
%    \multirow{2}*{Cli}
%    	& P & 0.1366s & 0.1645s & 0.0095s & 0.0004s & 0.2149s \\
%        & S & 0.1366s & 0.1645s & 0.0096s & N/A & 0.2203s \\
%%    \hline
%    \multirow{2}*{Closure}
%    	& P & 1.5708s & 1.4028s & 0.0826s & 0.0008s & 2.4102s \\
%        & S & 1.5708s & 1.4028s & 0.0656s & N/A & 2.4318s \\
%%    \hline
%    \multirow{2}*{Codec}
%    	& P & 0.1629s & 0.5514s & 0.0057s & 0.0004s & 0.6021s \\
%        & S & 0.1629s & 0.5514s & 0.0029s & N/A & 0.6015s \\
%    \hline
%  \end{tabular}
%  \label{fig:execution_time}
%\end{table}
%
%
%\begin{table}
%  \caption{Average execution time of each step.\vspace{-10pt}}
%%  \scriptsize
%  \begin{tabular}{cp{10mm}p{10mm}p{10mm}p{10mm}p{10mm}p{10mm}}
%    \hline
%     & Project & \tcs & Prep. & \tcp & Comb. & Total \\
%    \hline
%    \hline
%    \rotatebox[origin=c]{90}{\fzp} &
%		\multicolumn{6}{c}{
%		\begin{tabular}{p{10mm}p{10mm}p{10mm}p{10mm}p{10mm}p{10mm}}
%			Chart & 3.2244s & 1.1048s & 0.1121s & 0.0009s & 3.7759s \\
%			Cli & 0.1366s & 0.1645s & 0.0095s & 0.0004s & 0.2149s \\
%			Closure & 1.5708s & 1.4028s & 0.0826s & 0.0008s & 2.4102s \\
%			Codec & 0.1629s & 0.5514s & 0.0057s & 0.0004s & 0.6021s
%		\end{tabular}
%	}\\
%	\hline
%	
%	\rotatebox[origin=c]{90}{\fzs} &
%		\multicolumn{6}{c}{
%		\begin{tabular}{p{10mm}p{10mm}p{10mm}p{10mm}p{10mm}p{10mm} }
%			Chart & 3.2244s & 1.1048s & 0.0353s & N/A & 3.7905s \\
%			Cli & 0.1366s & 0.1645s & 0.0096s & N/A & 0.2203s \\
%			Closure & 1.5708s & 1.4028s & 0.0656s & N/A & 2.4318s \\
%			Codec & 0.1629s & 0.5514s & 0.0029s & N/A & 0.6015s \\
%		\end{tabular}
%	}\\
%	\hline
%  \end{tabular}
%  \label{fig:execution_time}
%\end{table}
%
%\begin{table}
%  \caption{Average execution time of each step for \fzp.\vspace{-10pt}}
%%  \scriptsize
%  \begin{tabular}{|rrrrrr|}
%    \hline
%    Project & \tcs & Prep. & \tcp & Comb. & Total \\
%    \hline
%    \hline
%	Chart & 3.2244s & 1.1048s & 0.1121s & 0.0009s & 3.7759s \\
%	Cli & 0.1366s & 0.1645s & 0.0095s & 0.0004s & 0.2149s \\
%	Closure & 1.5708s & 1.4028s & 0.0826s & 0.0008s & 2.4102s \\
%	Codec & 0.1629s & 0.5514s & 0.0057s & 0.0004s & 0.6021s \\
%	\hline
%  \end{tabular}
%  \label{fig:execution_time}
%\end{table}
%
%\begin{table}
%  \caption{Average execution time of each step for \fzs.\vspace{-10pt}}
%%  \scriptsize
%  \begin{tabular}{|rrrrrr|}
%    \hline
%    Project & \tcs & Prep. & \tcp & Comb. & Total \\
%    \hline
%    \hline
%	Chart & 3.2244s & 1.1048s & 0.0353s & N/A & 3.7905s \\
%	Cli & 0.1366s & 0.1645s & 0.0096s & N/A & 0.2203s \\
%	Closure & 1.5708s & 1.4028s & 0.0656s & N/A & 2.4318s \\
%	Codec & 0.1629s & 0.5514s & 0.0029s & N/A & 0.6015s \\
%	\hline
%  \end{tabular}
%  \label{fig:execution_time}
%\end{table}

% Build | Ekstazi | Prep. | FAST-P | FAST-S | Comb.

%\begin{table}[!t]
%  \caption{Average Execution Time of Each Step (in ms).}
%  \begin{tabular}{rrrrrrr}
%    \hline
%    Project & \ek & Prep. & \fs-P & \fs-S & \fzp & \fzs \\
%    \hline
%    \hline
%	Chart       & 3224 & 1105 & 0112 & 0035 & 3776 & 3791 \\
%	Cli         & 0137 & 0165 & 0010 & 0010 & 0215 & 0220 \\
%	Closure     & 1571 & 1403 & 0083 & 0066 & 2410 & 2432 \\
%	Codec       & 0163 & 0551 & 0006 & 0003 & 0602 & 0602 \\
%	Collections & 0237 & 1043 & 0029 & 0022 & 1073 & 1065 \\
%	Compress    & 0309 & 0326 & 0010 & 0005 & 0482 & 0482 \\
%	Gson        & 0297 & 0221 & 0020 & 0016 & 0415 & 0426 \\
%	Jsoup       & 0222 & 0195 & 0003 & 0002 & 0282 & 0283 \\
%	JxPath      & 0227 & 0067 & 0003 & 0003 & 0227 & 0229 \\
%	Lang        & 0262 & 0621 & 0037 & 0025 & 0670 & 0698 \\
%	Math        & 0747 & 1129 & 0265 & 0160 & 1575 & 1625 \\
%	Time        & 0500 & 1439 & 0020 & 0015 & 1631 & 1638 \\
%	\hline
%  \end{tabular}
%  \label{tab:execution_time}
%\end{table}

%\begin{table}[!t]
%  \caption{Average Running Times (in ms).}
%  \centering
%   \tabcolsep=0.105cm
%  \begin{tabular}{lrrrrr}
%    \toprule
%%    Project & \ek & Prep. & \fs-P & \fs-S & Comb. \\
%%\toprule
%\multicolumn{1}{l}{\multirow{2}{*}{Project}} &
%  \multicolumn{1}{c}{\multirow{2}{*}{Build}} &
%  \multicolumn{1}{c}{\ek} &
%  \multicolumn{1}{c}{\fs} &
%  \multicolumn{1}{c}{Prioritization} &
%  \multicolumn{1}{c}{Prioritization} \\ %\cmidrule(l){2-7} 
%\multicolumn{1}{c}{} &
%  \multicolumn{1}{c}{} &
%  \multicolumn{1}{c}{\textit{Selection}} &
%  \multicolumn{1}{c}{(Prep.)} &
%  \multicolumn{1}{c}{(Full suite)} &
%  \multicolumn{1}{c}{(Sel. suite)} \\ %\cmidrule(r){1-1}
%%\midrule
%    \midrule
%	Chart       & 4167 & 3224 & 1105 & 112 & 35 \\% & 0.6 \\
%	Cli         & 2997 & 137 & 165 & 10 & 10 \\%& 0.2 \\
%	Closure     & 6627 & 1571 & 1403 & 83 & 66 \\%& 0.5 \\
%	Codec       & 4581 & 163 & 551 & 6 & 3 \\%& 0.1 \\
%	Collections & 6627 & 237 & 1043 & 29 & 22 \\%& 0.3 \\
%	Compress    & 4986 & 309 & 326 & 10 & 5 \\%& 0.2 \\
%	Gson        & 4901 & 297 & 221 & 20 & 16 \\%& 0.2 \\
%	Jsoup       & 6098 & 222 & 195 & 3 & 2 \\%& 0.1 \\
%	JxPath      & 3643 & 227 & 67 & 3 & 3 \\%& 0.1 \\
%	Lang        & 5032 & 262 & 621 & 37 & 25 \\%& 0.3 \\
%	Math        & 6903 & 747 & 1129 & 265 & 160 \\%& 1.5 \\
%	Time        & 11521 & 500 & 1439 & 20 & 15 \\%& 0.2 \\
%	\bottomrule
%  \end{tabular}
%  \label{tab:execution_time}
%\begin{flushleft}
%%\footnotesize
%%\textit{\fs (Prep.)} is the time to compute the minhash signatures.
%%\textit{Hash} is the time \fs took to compute test case hashes.\anto{I think you need to update the table?}
%%\textit{Full} is the prioritization time of the full test suite.
%%\textit{Sel.} is the prioritization time of the selected test suite.
%\end{flushleft}
%\end{table}

\begin{table*}[!t]
  \centering
   \tabcolsep=0.105cm
  \begin{tabular}{lrrrrr}
    \toprule
%    Project & \ek & Prep. & \fs-P & \fs-S & Comb. \\
%\toprule
\multicolumn{1}{l}{\multirow{2}{*}{Project}} &
  \multicolumn{1}{c}{\multirow{2}{*}{Build}} &
  \multicolumn{1}{c}{\fs} &
  \multicolumn{1}{c}{\fs} &
  \multicolumn{1}{c}{\ek} &
  \multicolumn{1}{c}{\fz} \\ %\cmidrule(l){2-7} 
\multicolumn{1}{c}{} &
  \multicolumn{1}{c}{} &
  \multicolumn{1}{c}{\small(setup)} &
  \multicolumn{1}{c}{\small(\tcp)} &
  \multicolumn{1}{c}{\small(\tcs)} &
  \multicolumn{1}{c}{\small(\tcs + \tcp)} \\ %\cmidrule(r){1-1}
%\midrule
    \midrule
	Chart       & 4167   & 1105 & 112 & 3224 & 3259 (35) \\% & 0.6 \\
	Cli         & 2997   & 165  & 10  & 137  & 147 (10) \\%& 0.2 \\
	Closure     & 6627   & 1403 & 83  & 1571 & 1637 (66) \\%& 0.5 \\
	Codec       & 4581   & 551  & 6   & 163  & 166 (3) \\%& 0.1 \\
	Collections & 6627   & 1043 & 29  & 237  & 259 (22) \\%& 0.3 \\
	Compress    & 4986   & 326  & 10  & 309  & 314 (5) \\%& 0.2 \\
	Gson        & 4901   & 221  & 20  & 297  & 313 (16) \\%& 0.2 \\
	Jsoup       & 6098   & 195  & 3   & 222  & 224 (2) \\%& 0.1 \\
	JxPath      & 3643   & 67   & 3   & 227  & 230 (3) \\%& 0.1 \\
	Lang        & 5032   & 621  & 37  & 262  & 287 (25) \\%& 0.3 \\
	Math        & 6903   & 1129 & 265 & 747  & 907 (160) \\%& 1.5 \\
	Time        & 11521  & 1439 & 20  & 500  & 515 (15) \\%& 0.2 \\
	\bottomrule
  \end{tabular}
  \label{tab:execution_time}
\begin{flushleft}
%\footnotesize
%\textit{\fs (Prep.)} is the time to compute the minhash signatures.
%\textit{Hash} is the time \fs took to compute test case hashes.\anto{I think you need to update the table?}
%\textit{Full} is the prioritization time of the full test suite.
%\textit{Sel.} is the prioritization time of the selected test suite.
\end{flushleft}
\caption{Average Running Times (in ms).}
%\vspace{-3mm}
\end{table*}

%
%\begin{table}[!hbt]
\centering
\label{tab:efficiency_stats}
\begin{tabular}{@{}llll@{}}
\toprule
Comparison      & \textit{p-value}  & Significance & Effect Size ($\hat{A}_{12}$) \\ \midrule
FAST-Ekstazi    & 0.000462 & ***          & 0.04 (large)      \\
FAST-Fastazi    & 0.000366 & ***          & 0.03 (large)      \\
Fastazi-Ekstazi & 1        & ns           & 0.55 (negligible) \\ \bottomrule
\end{tabular}
\begin{flushleft}
\footnotesize
%\scriptsize 
\hspace{2em}ns = not significant, *** means \textit{p-value} < 0.001
\end{flushleft}
\caption{Time Efficiency Comparison.}
%\vspace{-3mm}
\end{table}


Looking at the average execution times for \fs, \ek, and \fz 
(the three rightmost columns in~\Cref{tab:execution_time}) 
the two main things we can observe are: 
\textit{i}) overall, \fs is the technique that incurs the least time overhead;
and \textit{ii}) the overhead of \fz with respect to \ek running time is generally very small.

To confirm our observations we performed the non-parametric Kruskal-Wallis rank sum test, 
and the result (\textit{p-value} $=$ 4.5e-05) confirmed that at least one of the approaches was different from the others with respect to the time efficiency.
Provided that significant differences were detected, we proceeded with pairwise comparisons to determine which approaches were different
and the results are displayed in \Cref{tab:efficiency_stats}.
Statistically significant differences were observed when comparing \fs with \ek and \fz, but not when comparing \fz with \ek.
Finally, to understand if the observed differences in time efficiency are not only statistically significant but also meaningful to support practitioners in the decision of whether \fz should be adopted, we measured the effect size.
The results can be interpreted in an analogous way of that explained in Section~\ref{subsec:rq1}.
The effect size for the comparison of \fs with \ek and \fz was $\hat{A}_{12}=0.04$ and $\hat{A}_{12}=0.03$, respectively, indicating that \emph{the effect on the time overhead when running \ek or \fz is large}.
On the other hand, the effect size for the comparison between \fz and \ek was $\hat{A}_{12}=0.55$, indicating that \emph{the additional time overhead incurred by \fz when compared with \ek is negligible}.

It is important to notice that such results concern the overhead time required by the studied techniques, which are anyhow one or two orders of magnitude shorter than the time required for actually running the whole test suites.

\begin{tcolorbox}%[boxsep=0mm,boxrule=0mm,size=minimal]
\textbf{Summary of RQ3}: When considering the three approaches in isolation, \fs is the most efficient one and the difference with respect to the time overhead incurred by the other approaches is \textit{large}.
The additional time overhead incurred by \fz for prioritizing the test cases selected by \ek is not statistically significant and the effect size is negligible.
\end{tcolorbox}

\section{Discussion}\label{sec:orch_discussion}

Software regression testing has undergone extensive research in the last several decades.
The largest part of solutions, though, addressed separately one dimension of the problem at a time.
While many \tcs and \tcp techniques have been proposed, they have not been directly compared, only
few authors look into integrated approaches for combined selection and prioritization, and no work empirically assessed the advantages of using \tcs and \tcp in combination over their individual application.
In contrast, we believe that, by merging differing criteria for selection and prioritization, we can achieve the most from the restricted subset of test cases that can be executed at each new release.

Towards this direction, we presented a study directly comparing 
two recent practical and effective approaches to \tcs and \tcp, namely 
file-based selection (by \ek) and similarity-based prioritization (by \fs).
Our results show that \ek generally outperforms \fs, although the effect size is negligible or small;
however, their orchestration by \fz outperforms both with a non-negligible effect. 
Moreover, considering a limited test budget, \fz exposed a higher effectiveness in consistent way. 
After assessing the overhead imposed by each of the studied approaches, we can conclude that \fz is quite practical: if we parallelize the preparation steps, the additional cost of similarity-based prioritization of the test cases selected by \ek is negligible.  

We aim at further improving the effectiveness and efficiency of \fz by refining several technical aspects. 
In particular, to make the approach more easily usable, it should be integrated into build systems as a plug-in as \ek is now.
In addition to that, we would also like to try orchestrating other \tcs and \tcp techniques from the literature to understand the resulting challenges and outcomes.

%More generally, this work paves the way to exploring a full range of potential strategies of combining differing criteria for selection and prioritization. 
%It can be worthwhile to also expand the study to the orchestration of techniques along other dimensions of regression testing, e.g., also test reduction or test amplification.
%Overall, we consider that for maximized efficacy under restricted budgets the problem of regression testing should be addressed in a holistic strategy that we called regression test orchestration. 

\subsection{Future directions for test suite orchestration}

As previously discussed, the experiments we performed with \fz is a starting point for test suite orchestration.
This work paves the way to exploring a full range of potential strategies of combining differing criteria for selection and prioritization. 
It can be worthwhile to also expand the study to the orchestration of techniques along other dimensions of regression testing, e.g., also test suite reduction (\tsr) or test suite amplification (\tsa).

When combining multiple techniques into a cohesive orchestration strategy, the first and perhaps most important aspect to consider is the sequence of operations.
We see in \Cref{sec:orch_fastazi} that there are two ways of using \tcs and \tcp together: we can either select a set of test cases and the prioritize these, or prioritize the entire test suite and run the selected tests in that given order.
Including more techniques in the orchestration inevitably leads to more possible sequences.

For example, if we add \tsr to the orchestration, the operation could be performed before or after the selection and prioritization.
By using it before, we already restrict the number of test cases the other techniques must deal with; doing it afterwards, the results of the reduction will only be used in the next execution of the test suite.

The combination becomes more interesting when adding \tsa to the strategy.
\tsa could be the first technique to run, updating or adding test cases that will then serve as input for selection, prioritization and reduction.
Or, it could be placed in between selection and prioritization, modifying the suite only according to the results of the selection.
This could be desired if the \tsa process is costly and running it with fewer targets greatly reduces the time it consumes.

Continuing this line of thought, \Cref{fig:orchestration} shows an example of a fully orchestrated test suite execution.
In it, we consider three subsequent versions of the \sut (\textbf{v\textsubscript{i-1}}, \textbf{v\textsubscript{i}}, \textbf{v\textsubscript{i+1}}).
The chevron boxes represent some process being applied to the tests, while the cut rectangles represent variations of the test suite (e.g., a list of test cases).

The target of the orchestration is \textbf{T}, which is the test suite corresponding to version \textbf{v\textsubscript{i}} of the \sut.
The first technique to be applied is \tcs, generating a subset of tests \textbf{S}.
Additionally, from previous test execution logs, historical data, such as test that have recently failed, can be extracted, forming the set \textbf{H}.

This subset is then used as input for \tsa techniques, in this example displayed separately as augmentation and amplification.
The results are one set of \textit{newly generated} tests \textbf{G} and one set \textbf{A} containing the amplified versions of the tests in \textbf{S}.
At this point, information from \textbf{H}, \textbf{G} and \textbf{A} is merged into a list of tests \textbf{M}.

\textbf{M} is then used as input for three different techniques.
On one side, \tsr is used, using information from \textbf{M} and \textbf{T} to eliminate excessive redundancies in the suite and produces a tighter suite \textbf{R} that can be used as a starting point for the next cycle of orchestration (when it is time for version \textbf{v\textsubscript{i+1}} of the \sut to be tested).
On the other, \tcp prioritizes the test cases to \textbf{P} and a test flakiness detection technique provides a list \textbf{F} of potentially unreliable tests, which should be handled differently during execution.

Finally, the orchestrated test suite \textbf{O} is produced, which can be used to test the \sut version \textbf{v\textsubscript{i}}.


\begin{figure*}[h]
  \centering
  \includegraphics[width=\linewidth]{figures/Orchestration.pdf}
  \begin{flushleft}
	\footnotesize Legend: 
	\textbf{v\textsubscript{i-1}}, \textbf{v\textsubscript{i}}, \textbf{v\textsubscript{i+1}}: previous, current and next version of the \sut; 
	\textbf{H}: output of history-based criteria;
	\textbf{T}: the test suite as of version \textbf{v\textsubscript{i}};
	\textbf{S}: the output of \tcs;
	\textbf{G}, \textbf{A}: the outputs of test suite augmentation and amplification, respectively;
	\textbf{M}: a \textit{selected and enhanced} test suite combining the outputs of the previous steps;
	\textbf{R}: the output of \tsr;
	\textbf{P}: the output of \tcp;
	\textbf{F}: the output of a flaky test detection technique;
	\textbf{O}: the \textit{orchestrated test suite} that should be executed for \textbf{v\textsubscript{i}}.
  \end{flushleft}
  \caption{Diagram showing an example of a fully orchestrated approach to the test suite execution and evolution.}
  \label{fig:orchestration}
\end{figure*}

It is worthy of reiteration that this is simply an example, built upon the goals of each \rt technique and considering how they can be used to each others' advantages.
Validating such a model requires extensive experimentation, which unfortunately poses a technical challenge, as not every technique has an available and easily usable implementation.
Even when the tools exist, the way each one handles inputs and outputs can be incompatible, so some alteration is needed.

Some questions remain unanswered regarding a fully orchestrated strategy.
The possibility of executing all \rt techniques at each new version of the \sut largely depends on the intervals between versions; if new versions are committed frequently, there might not be enough time to execute the full process.
In such cases, an additional point to consider is which techniques are important for frequent execution, and which ones can become part of a nightly testing cycle.

%\todo{Discuss possible ways of combining other \tcs and \tcp techniques along with \tsr and \tsa}
%----------------------------------------------------------------------------------------
%	Insights from Industry
%----------------------------------------------------------------------------------------
\chapter{Insights from Industry}\label{chap:industry}
\lhead{\emph{\nameref{chap:industry}}}

%\section{Summary}\label{sec:ind_summary}

As seen in the discussion from \Cref{chap:literature_review}, not many ideas proposed in academia make their way into practical usage.
To dig deeper into this problem and understand the underlying challenges, a crucial part of this research involves direct communication with members of industry who work on software testing.

Upon contact, members from a large technology company granted the opportunity to spend some time at their offices to observe practices, collect data and interview team members.
This chapter synthesizes the findings of a seven-week period which was spent in direct contact with the company.

The interacted team is responsible for a software system, which is an integral component of the company's delivered product.
Specifically, practices related to multi-component testing (MCT) were investigated; at the time, there were thousands of MCTs in the system.

The interviews cover a variety of topics, including education in software testing, current practices and procedures employed by the team, their relationship with members and research from academia, and the most notable challenges they face.
Analysis of the responses show that there is a strong desire to improve processes, which is hampered by reasons including technical challenges, bureaucratic hurdles and even skepticism by some team members of automated solutions.

\Cref{sec:ind_rqs} describes the questions that drive the discussions brought here.
\Cref{sec:ind_overview} provides an overview of the testing process at the company and of the system for which the interviewed team is responsible.
\Cref{sec:ind_interviews} brings quotes from the interviews to form a picture of the state of practice from the perspective of the practitioners themselves.
Finally, \Cref{sec:ind_observations} contains the answers to the research questions, in the form of a series of observations derived the interviews.

%Recently, there has been research specifically about the ``Industry-academia gap''.

%In this paper, we aim to bring more light into this topic by drawing some comparisons between points frequently discussed in the literature and the actual state of practice in an industrial partner.

%Section \ref{sec:background} gives some background on regression testing. Section \ref{sec:rqs} contains the research questions and their discussions. Section \ref{sec:related} highlights relevant related work. Section \ref{sec:threats} points potential threats to the validity of this study. Finally, \ref{sec:conclusion} offers our closing thoughts.


\section{Research Questions}
\label{sec:ind_rqs}

The following research questions represent the main objectives of the collaborative interaction with Ericsson.
Generally, the contribution of this period is the identification and explanation of notable challenges involved with industry-academia collaboration and the advancement of practical software testing techniques.

Keep in mind these are not the questions asked directly to the practitioners (these are found in \autoref{app:surveys}), but they were used a starting point to design the interviews.

\paragraph{RQ\ref{chap:industry}.1.} What are the issues most frequently noted by software testing practitioners?
\paragraph{RQ\ref{chap:industry}.2.} What are the challenges that arise when trying to incorporate academic insight in practice?
\paragraph{RQ\ref{chap:industry}.3.} What are potential paths to improve collaboration between academics and practitioners?
\todo{Note: The RQs are 5.x because this is chapter 5 of the thesis, it appears as ch.1 because I've omitted the others}
%How can academics make their work more accessible for interested practitioners?


\section{Overview of Testing at Ericsson}\label{sec:ind_overview}

%Regression testing is a topic that has been extensively studied for decades.
%
%It includes topics such as \textit{test case selection}, \textit{test case prioritization} and \textit{test suite reduction}.

%\subsection{Regression Testing in the Literature}
%
%Most academic studies are focused on unit tests or lower-level integration tests.
%
%\subsection{Regression Testing at Ericsson}

Since the mid-2000s, Ericsson has adopted agile development principles.
This had a notable effect on testing practices, as previously there were distinct roles for developers and testers and now developers are mostly in charge of testing their own code.

Currently, Ericsson's test design is centered around a ``test strategy pyramid'' that provides the hierarchy of tests for a given system.
At the bottom of the pyramid, there are unit tests, which should be the most atomic and numerous tests.
Other layers are formed by component tests, multi-component tests, module tests, node tests, RAN(radio area network) tests and, finally, network tests.
At each layer, the complexity of tests increases, as each one becomes responsible for covering a larger amount of source code.

At least at the lower layers (unit, component, multi-component), testing can generally happen in three spots.
Naturally, developers run the test suite on their development machine as a mechanism to aid the writing of new code.
At this point, new tests can also be written, or old ones can be updated, to account for changes in requirements.
During the day, multiple developers commit changes that should be merged into a component's main branch and, at that point, a source delivery check (SBC) is queued for execution on a testing server, running a selection of tests to ensure critical features are functional.
Finally, at the end of a work day, a source baseline test (SBT) is started, which runs all test cases and ensures none of the day's updates caused a system-breaking error.

Generally speaking, although the unit tests are more numerous, the higher-level tests are responsible for a great part of the testing costs, as they involve multiple software components along with device/network simulators and, in some instances, actual physical hardware.
Additionally, when a failure is detected at higher levels, it is more challenging and time-consuming to identify the cause of the issue.
In comparison, unit tests are often executed completely by a developer at their local computer and checked again while merging the code in continuous integration; as each test covers well-defined pieces of code, a failure in this level leads to a quicker understanding of what could be going wrong.

For this reason, certain teams at Ericsson have been implementing a ``shift left'' policy for testing.
The objective is to bring as much fault-finding capability as possible to the lower-level tests.
One notable way of doing this is by writing new unit or component tests whenever a failure happens in a complex test.
However, this policy is also to be incorporated in test strategies still under development.



%In the mid-2000s there was a shift to agile development which affected the testing workflow.
%
%Ericsson employs a ``test strategy pyramid'' that starts with unit tests at the lower end and goes up to network tests at the high end.

%A substantial part of the testing cost comes from the high-level network tests, which are complex and expensive, as they involve multiple layers of software as well as physical and simulated hardware.

There is a ``shift left'' objective that aims to bring fault-finding into lower levels of the pyramid.

There are frequently executed ``checks'' (that may employ some sort of test case selection) and overnight ``tests''.


\subsection{Overview of the system}

For this study, a period of seven weeks was spent at the office of Ericsson in Linköping, Sweden, in order to understand fundamental aspects of the testing procedures at the company.
It cannot be said that this is a comprehensive account, because the data is extracted from only one small part of the entire corporation, and the overall scope of the projects being conducted is too large for full comprehension in such a short time frame.

The investigated system is called Traffic Control, which is part of the software stack that Ericsson deploys to telecommunication infrastructure (e.g. cell towers).
Currently, there are 4G and 5G versions of the software in active development/maintenance mode and in use by end users.

Furthermore, the team we interacted with is mainly concerned with multi-component testing (MCT), mostly written in Erlang, which involves the integration of multiple software components in addition to hardware and infrastructure simulators.
Reports indicate that the 4G version of Traffic Control has $\sim14,000$ multi-component tests and the 5G version has over 5,000.



%When an overnight test fails, it is often due to test flakiness or environment misconfiguration. Often it is not an error in the SUT proper.
%
%In the integration/multi-component level, it is often the case that higher-level/more complex tests cover lower-level/simpler functionality by definition. However, there is no system in place to refactor and remove older tests that are no longer necessary.
%
%


\section{Interviews}
\label{sec:ind_interviews}


Information about how testing is performed at Ericsson and regarding the challenges still faced at the company was gathered via a series of interviews.
This was initially in the form of unstructured conversation, while the interviewer understood the central details.
After the basics were covered, we performed a series of 30- to 60-minute sessions with 1 to 3 people at a time asking more focused questions.

\autoref{table:interviewees} lists the roles of the interviewees, who are anonymized for this study.

\begin{table}[]
\centering
%\scriptsize
\rowcolors{1}{}{gray!10}
%\setlength{\tabcolsep}{6pt}
\begin{tabular}{ll}
\toprule
\textbf{ID} & \textbf{Role} \\
\midrule
R1 & Functional Area Domain Tester \\
R2 & Functional Area Domain Tester \\
R3 & Functional Area Domain Tester \\
R4 & Continuous Integration Test Manager \\
R5 & Module Test Manager \\
R6 & Module Test Manager \\
R7 & Senior Test Specialist \\
R8 & Senior Test Specialist \\
\bottomrule
\end{tabular}\\
\caption{Interviewees at Ericsson.}
\label{table:interviewees}
\end{table}

\subsection{Current practices}

Regarding the current testing practices at Ericsson, we want to understand what are the day-to-day activities performed by the team.
We would also like to know if there are any implementations of the regression testing techniques classically studied in research (selection, prioritization, reduction, amplification).

The interviewed team members are mostly working on multi-component testing (MCT) and have all mentioned two key acronyms for this layer: \quoter{SDC for delivery, SBT for nightly runs}{R1}.
SDC, or Source Delivery Check, is a short execution of the test suite that happens whenever a developer pushes changes to be merged into the main branch of a module.
Due to time and resource constraints, \quoter{for SDC there is selection, in nightly we run everything}{R1}.
This is also called the ``gating loop'', since a failing test prevents the change from being merged.
At the time of the interviews, \quoter{it's manually decided what goes in SDC}{R6} and \quoter{the running time [...] for MCT is 15-20 mins}{R5}.
SBT, or Source Baseline Test, is the nightly run of the test suite, also known as ``assessment loop''.
Here, all tests in the suite are executed, which can take up to 10 hours.
As a general rule, the tests in SDC are a subset of the SBT.

\paragraph{Selection.} In MCT, it is manually determined whether a test should be included in the SDC: \quoter{It's manually decided what goes in SDC}{R6}; \quoter{yesterday, we got the question, `should we include this in gating?' I don't know, I just go by feeling. We have to see if the feature seems fundamental or important to Ericsson somehow}{R5}.
Despite this practice, most of the interviewees are aware that there could be a better way of doing things: \quoter{I think there should be more strategy than just me thinking}{R5}.
When this finding was reported to R8, who oversees testing all over Ericsson and does not work with R1-7 on a daily basis, they mentioned that the company does have an internal tool for test selection named RENSA and was surprised to find out this team did not use it.
R4 explains: \quotes{we had attempts to integrate RENSA but it did not work out so well. I think we never built a business case for 5G and it ended up way in the backlog. In 2017 or 2018 we said `this is something we should add to our plan', but we did not do anything for 5G.} 
However, \quoter{it might be deployed in other parts of Ericsson}{R7}.
That said, R8 also expressed some skepticism towards \tcs: \quoter{similarity-based selection misses boundary values. Why do you need to select? The agile principle says we should test everything}{R8}.

\paragraph{Prioritization.} This does not appear to be an active concern for the team and there are no techniques in place for advanced prioritization: \quoter{for prioritization, nothing specific}{R1}.
\quoter{We have suites, which are groups, e.g. for sub-modules, gating or not. Then I think it's just the order they're written.}{R5}.
However, \quoter{now we have introduced shuffling for the assessment}{R5}, i.e. random ordering of the test cases.
The motivation for this is not decreasing the feedback time, but rather \quoter{to help find problems with unstable tests}{R1}, which might be flaky according to the execution order.
R8 again expressed a concern regarding \tcp: \quoter{using the same TCP approach every time, wouldn't the same test be top priority every time?}{R8}.

\paragraph{Reduction and Amplification.} These techniques are absent in the workflow. \quoter{We don't have any tools that helps us in any way in shrinking or expanding the tests}{R5}. \quoter{No, we don't have anything like that}{R6}.
Regarding reduction, \quoter{if it happens that there are too many tests for overnight, there would be an initiative to reduce}{R1}, but \quoter{we can put more machines and we can run more}{R6}.
Interviewees generally agree that it is more cost-efficient to increase computing capacity of testing servers than to spend human time determining which tests to remove.
For amplification, there appears to be no interest in automating the process.
There is a protocol, however, to manually augment the test suite in the situations where a fault slips through a layer of testing: \quoter{if a fault was found later in test or in the field, we try to investigate why did it slip through, and we should have a test for that}{R1}; then, \quoter{after the fact, if there is time, we can go back and understand why a needed test was not in the plan}{R2}.


\subsection{Roles and experience}

R1, R2 and R3 are members of the Functional Area Domain team.
They describe themselves as \quoter{the owners of the test suite}{R2} and are responsible for \quoter{monitoring the test results}{R2}.
In particular, they \quoter{are monitoring the nightly runs}{R1}, i.e. the SBT or non-blocking tests.

R4, R5 and R6 are test managers, albeit R4 has a different responsibility.
The module test managers are responsible for the testing of particular modules at Ericsson, meaning \quoter{not working on specific features}{R5}.
\quoter{I also guide the teams on how to write test cases [...], how we organize test suites and so on}{R6}.
Meanwhile, the CI Test Manager is \quoter{responsible for the machines and environment and some of the test framework parts. To simplify, we are doing the framework, the developers are writing the tests, the managers are designing the strategy}{R4}.

Finally, R7 and R8 are designated as test specialists, meaning they handle longer-term strategy.
\quoter{I work a lot with test strategies. How we should test features and products [...] from now until 2025}{R7}.
Despite the similar titles, there is a key difference between the two specialists: R7 is responsible for a long-term strategy of a specific core of Ericsson products, while R8 has an overview of the entire company, so their job also includes sharing technologies and strategies among distant teams.

Aside from R8, who has a PhD on the topic of software testing, most of the interviewees did not have testing as a focus of their education while at university.
When asked about their education, R4, R5 and R6 are computer scientists and/or engineers, and most of the testing knowledge they had prior to working at Ericsson was \quoter{standard university [curriculum], which contains tests}{R5}.
Meanwhile, R7 \quotes{graduated in media and communication}, so testing was \quotes{not covered in university}.
R4, R5 and R7 mention having a certification by the ISQTB\footnote{International Software Testing Qualifications Board}.
Additionally, R4 also mentions taking testing courses led by R8 internally at Ericsson.

\subsection{Pain points}

\subsection{Collaboration with Academia}


\section{Observations}
\label{sec:ind_observations}

\subsection{RQ1: Common Issues}
\label{sec:ind_rq1}

From the literature examined in \Cref{chap:literature_review}, it is possible to observe that test case prioritization is the regression testing technique most widely addressed in academic research.
The conversations with the team, however, show that it is not a topic of active concern for the practitioners.
Indeed, when questioned about methods of determining the execution order of tests, respondents claim that, until recently, it was \quoter{just the order in which they were written}{R5}.
Due to concerns with flaky tests and developers who were explicitly asking to run their tests before all others, shuffling (random prioritization) was introduced.
This helps the identification of tests that are flaky due to interdependency — e.g. one test that presumes another has been executed previously — or due to improper cleanup — tests that affect the global state of the program and do not revert that change before concluding.
However, it is a far simpler approach than what is studied in the literature and has an ultimately different goal: detecting flaws in the test suite itself rather than speeding up feedback time for faults detected in the SUT itself.

Indeed, the detection of poor test case design appears to be a common problem to the test managers and specialists.
Although they did not provide raw data to analyze this claim, their intuition indicates that 30\% to 50\% of test failures are caused by poorly implemented tests and do not lead to a true error in the product.
It's well-accepted that most developers are not adequately educated or trained in software testing principles and learn by imitating previously-existing tests written by people who might be in different roles by that point.

The practice of running shorter test suites during the day and the full range of tests at night is common at the company and, for that reason, test case selection can be a valuable asset.
In the unit test level, ``configuration-based selection'' is used, which is similar to the file-based selection frequently discussed in the literature.
However, for multi-component tests, it is manually decided what tests should be executed in SIRT, following a few guidelines and procedures.
Each test in this level is certainly more complex than a unit test, and perhaps techniques for addressing each type of test need to be distinct.
There is an internal tool used for selection, but the team interviewed for this study claim that attempts to integrate it into their workflow were unsuccessful.

Since most software products are in constant evolution and growth (meaning new requirements and features), it is expected that the test suite will also grow accordingly.
In the long term, this means thousands of test cases for a product, sometimes spread across multiple modules.
It is often the case that, in the early stages of development, tests will be designed to cover the most basic functionality of the software, which serve as the foundation to more complex features later on.
Then, as these complex features are introduced, more specialized tests must also be added, covering intricate details of the functionality.
Many of these specialized tests must also cover basic functionality by definition, even if indirectly through the higher-order features that depend on core elements\footnote{This is not always true for unit tests, which attempt to isolate functionality as much as possible for testing, but is common in component and multi-component testing.}.
Thus, in reality, a lot of older tests are made obsolete, at least in purely fault-finding terms.
Detecting which tests are redundant is a challenge that was mentioned by several of the interviewees, as currently this would need to be performed manually and the time/budget restrictions do not allow this.
An automated solution could be helpful, although it is not desirable to completely delete test cases — when a fault is detected, tests that cover similar parts of code can be analyzed together to help identify the cause of the issue.
Thus, there are tests that do not need to be executed daily but could be added to a weekend-only testing schedule, for example.
That said, despite acknowledging the challenge of reducing the size of a test suite, it is much easier to simply add resources to the testing server than to spend human time and effort into analyzing tests or even implementing an automated solution.

There did not appear to be a strong desire for automated amplification or augmentation of the test suite.
As it is, the suite already grows substantially over time, so adding machine-generated tests to it could create more problems than it solves.
Regarding new tests, test managers are more concerned with their quality than quantity, so perhaps methods for aiding developers in manually writing good tests would be better accepted than simply offloading that task to an automated tool.

\begin{tcolorbox}%[boxsep=0mm,boxrule=0mm,size=minimal]
\textbf{Summary of RQ5.1}: The issue most frequently brought up by respondents has to do with test flakiness, which is very common in their regression test suite; testers and managers have found ways of dealing with them, but it still makes it more difficult to detect true errors in the test suite.
More generally, the main challenge is detecting and improving poorly written tests, which is the leading cause of test flakiness.
There is a manual process for deciding which tests should be executed at each new commit, while removing redundant tests is something that is talked about, but not usually done because it's cheaper to simply add more compute power to testing servers.
\end{tcolorbox}

\subsection{RQ5.2: Challenges of Incorporation}
\label{sec:ind_rq2}

The interviews make it possible to identify a few notable obstacles that prevent the usage of state-of-the-art research techniques in the corporate environment.

From a technical perspective, the first challenge is that academic tools can barely ever be used as-is.
As discussed in \Cref{sec:lit_rq3}, a lot of additional effort must be exerted in order to go from an algorithm to a replication package, to a functional prototype and, finally, to a commercially viable tool that can be used by developers.
Even if the tool does exist and is available as FOSS, it might be geared specifically to a certain programming language or require certain environment characteristics; if these don't match, the interested party would need to re-implement the algorithm for a new target.
Those conditions being met, the testing tool must still go through extensive security screening, which adds to the time and cost needed to implement a technique.

Assuming that the above requirements are met, there are additional barriers that might not even be a matter of time and cost.
One issue that was pointed out by the test specialists is that academic tools generally assume that the SUT is perfectly designed and has plenty of data available to work with, but that might not always be the case.
Even the best designed software will have the occasional oddity, peculiar characteristics that humans might be used to handling, but could produce unexpected results in automation.
Regarding the data, there was a discussion during the execution of this study about collecting historical test execution data in order to run experiments; as it turns out, it is not a simple task.
Many of the test executions happen in developers' local computers and this data is not included in the CI/CD history.
Since these local executions are likely to have frequent failures, their execution history can provide important data about which tests are most important in the initial stages of testing, for example.
Furthermore, even considering only the data available from the CI/CD tools, logs are not meant for long-term storage and not organized in a way that can easily be fed into, for example, a selection or prioritization algorithm.

Considering these challenges above, it is not exactly tackling them that is the greatest obstacle.
Instead, many interviewees say they would be excited to try new techniques and perform experiments with their software to try and find more efficient ways to handle their testing workflow.
However, all this would require a time investment from the team members, which means setting aside other tasks, such as feature delivery.
Since new features are the most desirable output in the perspective of customers, it is difficult to convince people in managerial and decision-making positions to slow down deliveries in order to perform experiments that, realistically, might not be successful.
An argument to managers would need to include time and cost estimates, including forecasts showing that, in the mid- to long-term, performing these experiments now will lead to notable cost savings in the future.
To avoid slowdown, an alternative would be to hire people dedicated specifically to experiment with new techniques and identifying avenues of improvement; however, hiring employees is an expensive process by itself and, given the choice, managers would prefer buying new computers to run more tests before hiring an employee to optimize the existing ones.

\begin{tcolorbox}%[boxsep=0mm,boxrule=0mm,size=minimal]
\textbf{Summary of RQ5.2}: 
Practitioners describe several technical challenges involved with the implementation of new techniques into their workflow.
Notably, adapting algorithms to their environment, screening for security risks and collecting appropriate data for input are frequently cited.
Regardless, bureaucratic hurdles are more difficult to address than technicalities and, in order to convince managers to invest time and money into the effort, there needs to be robust evidence that a technique will provide meaningful benefits.
\end{tcolorbox}


\subsection{RQ3: Paths to Improve Collaboration}

The interviews make it clear that, at least within the interacted team, there have not been many attempts of collaboration with academia, in the sense of bringing techniques from theory into practice and/or attempting experimentation in realistic software.
\sugg{This is not a general assessment of the company, as even in the literature review (\Cref{chap:literature_review}) there is an example of a paper that was written in collaboration with the company \cite{najafi_improving_2019}.}
\todo{I assume we also have to remove this part, as I cite a paper that clearly mentions Ericsson as the industry partner.}
Nevertheless, the conversations highlights the current desire for such collaboration and some avenues for improvement.

The team is not completely isolated from academia, as the company frequently funds Master's programs jointly with \sugg{the local university}, although the motivation for this is not necessarily scientific advancement nor development of novel techniques.
Students in this program get the opportunity to interact more closely with the inner workings of the team and acquire specialized knowledge before graduating.

Less common are collaborations with PhD candidates, post-doctoral fellows or professors.
Occasionally a researcher will invite an employee to participate in a study, such as the aforementioned study with \sugg{a neighboring company}, or even this present study.
In such cases, there is usually some data provided by the company or by one of its employee that are used by the researchers in some work meant for academic publication.
The next step of this type of collaboration would be to bring the results of that study back into the company and attempt to internal experimentation and potentially implementation of the technique.
As far as the interviewees are aware, this either has not happened, or happens very rarely.

With few exceptions, practitioners admit that they do not have the time to keep up with Software Engineering research and no longer maintain contact with former colleagues who might now be professors.
There is no simple solution to this problem, as following research trends is not only time-consuming; it can be mentally draining for someone preoccupied with other tasks and it is rarely obvious how the results of a paper can be beneficial for an individuals' workflow.

Internally, there is a team whose responsibility is to interact with different parts of the company and update workflows with new techniques.
This dynamic is similar to the ``ideal'' scenario of industry-academia relations, cutting down some major obstacles (the company maintains control over the developed tool, there is a budget allocated for this, etc.).
However, even a tool proposed and experimented with this team did not see adoption by the team that was interviewed, due to the limited benefits observed in the multi-component testing layer.

\begin{tcolorbox}%[boxsep=0mm,boxrule=0mm,size=minimal]
\textbf{Summary of RQ5.3}: 
There is interest and desire of collaboration among interviewed practitioners, but it is something they find difficult to take initiative upon.
Most of their time is consumed performing day-to-day tasks and ensuring the delivered product is constantly improved, and there is little time or energy left to keep up with the quickly growing academic literature.
They often fund Master's programs and there are examples of academic papers using data from the company, but it is difficult to find situations where research results directly impacted the current state of practice.
\end{tcolorbox}


%----------------------------------------------------------------------------------------
%	The Industry-Academia Knowledge Gap
%----------------------------------------------------------------------------------------
\chapter{Challenges Between Industry and Academia}\label{chap:gap}
\lhead{\emph{\nameref{chap:gap}}}

\todo{Is there a better title for this?}

While the review in \Cref{chap:literature_review} indicates that \rea is a growing concern among \rt researchers, it's still only being addressed with any depth on a minority of secondary studies.
It is clear that several authors believe \rea is a challenge worth addressing in research, but there is not a lot of available \rt literature focusing on the steps that need to be taken in order to improve academia-industry communication and shorten the technology transfer gap.

We conclude this work by highlighting some key challenges that we have identified, combining data found in the literature itself, in the authors' responses and in the practitioner survey.
These are challenges that may have been addressed in certain circumstances but remain unsolved in a broad sense, as they are still present in several recent works.
Along with each challenge, we make some suggestions that could be applied by Software Engineering researchers and/or Software Testing practitioners — these could be actionable steps for upcoming primary studies, or further avenues of investigation for secondary or meta studies.
\Cref{table:challenges} provides the summary of the challenges we identified, indicating the primary source of our observation (i.e. the literature, the authors and/or the practitioners).

\begin{table}[]
\centering
\scriptsize
\rowcolors{1}{}{gray!10}
%\setlength{\tabcolsep}{6pt}
\begin{tabular}{llllll|llllll}
\toprule
\textbf{ID} & \textbf{Title} & \textbf{L}  & \textbf{A} & \textbf{P} & \textbf{I} &
\textbf{ID} & \textbf{Title} & \textbf{L}  & \textbf{A} & \textbf{P} & \textbf{I}\\
\midrule
CH1 & Alignment of motivations             &   				&  			     & \fullcirc      & \fullcirc &
CH6 & Absence of TSR/TSA                   & \fullcirc 		&                & \fullcirc 	  & \fullcirc \\

CH2 & Realistic experimentation            & \fullcirc 		&                &                & &
CH7 & Clarity of target                    & \fullcirc 		&                &        	      & \\

CH3 & Scalability                          & \fullcirc 		&                &                & &
CH8 & Skepticism                           &                & 	 & 	  & \fullcirc \\

CH4 & Relevance of metrics                 & \fullcirc 		& \fullcirc 	 &                & &
CH9 & Data availability                    &                & 	 & 	  & \fullcirc \\

CH5 & Developing usable tools			   &            	& \fullcirc 	 & \fullcirc      & \fullcirc &
CH10 & Communication                       &                & \fullcirc 	 & \fullcirc 	  & \fullcirc \\
\bottomrule
\end{tabular}\\
\begin{flushleft}
\scriptsize Source(s): \textbf{L}: Literature; \textbf{A}: Author questionnaire; \textbf{P}: Practitioner survey; \textbf{I}: Industry partner.
\end{flushleft}
\caption{Summary of main challenges identified by this study.}
\label{table:challenges}
\end{table}

% ===================

\paragraph{CH1: Alignment of motivations}
When asked what would convince them to implement and use an~\rt tool, eight practitioners gave responses that can be synthesized into ``\textit{it would make my work easier}''.
So there exists a mismatch between academic motivations and industrial needs: research is concerned with discovering novel techniques that might provide marginal effectiveness gains over the state-of-the-art, while practitioners are mostly concerned with any solution that simplifies their workflow.
In other words, even if a \tcs technique has the potential to greatly reduce the testing time of a suite, practitioners will weigh those benefits against the effort required to implement the technique and adapt/maintain it for their needs.
This is not to say that the current research motivations are ill-informed: it is the role of academia to push the boundaries of what is possible in theory first, and sometimes this theory takes many years to find relevance in practice.

If the researchers have the objective of implementing their approach, they must be certain that it is addressing the current needs of practitioners.
An obvious way to achieve this, which is also confirmed by our study, is through partnerships between academic researchers and industrial practitioners (or even open-source communities).
These collaborative works, by their own nature, tend to produce results suitable for practical applications and could serve as a guideline for other, purely academic, approaches.

Naturally, not all research can be done with industrial partnerships, and in these cases there is difficulty in finding what exactly is relevant to current practitioners.
One possible source of this information is grey literature: information produced by experts in a field, but without necessarily following academic guidelines, in the form of blog posts, videos, magazine articles, talks etc.
Practitioners who produce grey literature can help inform researchers about the current state of practice, the main existing challenges in software development, and successful implementations of techniques (e.g. the aforementioned Netflix blog~\cite{netflixlerner}).

% ===================

\paragraph{CH2: Realistic experimentation}
It is clearly not possible for every research paper to feature practitioner co-authors or to rely on an industrial partnership for experimentations.
Selecting the right subject for experiments is a decisive point when writing a paper about a technique.
Older studies on \rt would often rely on the ``Siemens programs''~\cite{hutchins1994experiments}, which is believed to have caused an overfitting of results to a particular kind of software~\cite{do_recent_2016}.
More recently, the Software Infrastructure Repository (SIR)~\cite{do2005supporting} (e.g. \citepalias{schwartz_cost-effective_2016}) and Defects4J~\cite{just2014defects4j} (e.g. [\citetalias{noor_similarity-based_2016}, \citetalias{azizi_retest_2018}]) have been used to similar ends.
Having common subjects can provide replicability benefits when directly comparing techniques, although is not always clear if they approximate the difficulty of testing real software.
Authors who are able to collaborate directly with members of industry gain an enormous advantage if they are allowed to run experiments on production code, but it is also clear that not every paper will have that opportunity.

The most obvious alternative is to use large-scale open-source software (e.g. from the Mozilla \citepalias{zhou_beating_2020} and Apache 
[\citetalias{oqvist_extraction-based_2016}, 
\citetalias{bertolino_learning--rank_2020}, 
\citetalias{pan_dynamic_2020}, 
\citetalias{bagherzadeh_reinforcement_2022}, 
\citetalias{chen_context-aware_2021}]
 foundations) as subjects, since the communities developing these programs follow procedures much like the developers working for corporations .
This is also far from trivial.
The larger the software, the more time a researcher will need to dedicate in order to understand it and to adapt the technique to it, sacrificing the possibility of experimenting on a larger variety of subjects and thus again bringing the risk of overfitting.
Additionally, there is no established consensus regarding which properties an open-source program must satisfy in order to be a satisfactory subject.

Alleviating this issue would require effort from both researchers and practitioners.
For example, Google has an open dataset of testing results~\cite{googledataset}, and \citetalias{spieker_reinforcement_2017} combined it with one from ABB Robotics.
As a result, this combined dataset has already been used by other papers covering machine learning 
[\citetalias{wu_time_2019}, 
\citetalias{lima_multi-armed_2022}, 
\citetalias{pan_dynamic_2020}, 
\citetalias{sharif_deeporder_2021}, 
\citetalias{omri_learning_2022}].
Two practitioners mention that ``\textit{open source code/data is not provided}'' due to confidentiality reasons.
In those cases, our suggestion would be to provide some opaque information regarding the system, e.g. its programming language, the number of lines of code and/or tests, how many developers work on it, how frequently is the code updated, etc.
At the very least, this would help researchers choose subjects with similar characteristics.

% ===================

\paragraph{CH3: Scalability}
\rt techniques provide the most savings when applied to large-scale software projects, which can have multiple thousands of test cases.
Therefore, it is important that techniques are designed to scale up to any size of test suite, but few papers tackle this issue directly.
The trouble is that scalability is very hard to measure unless multiple subjects of different sizes are used.
One way to demonstrate scalability, beyond relying on industrial partners or large-scale open-source projects, is to artificially generate large datasets (e.g.~[\citetalias{miranda_fast_2018}, \citetalias{cruciani_scalable_2019}]), which are useful from the algorithmic perspective, but might not address other issues that arise in large-scale software development.
It is also worth mentioning that many \rt techniques can become \textit{disadvantageous} when applied to small test suites, as the cost of running the technique does not outweigh the savings in testing time.
So selecting the size of the experiment subject is important both to highlight the scalability of the tool in large software and also to consider whether the necessary overhead is a deal-breaker on small or medium projects.


% ===================
\paragraph{CH4: Relevance of metrics}
\Cref{sec:lit_rq1} shows that a wide variety of metrics has been used to evaluate the effectiveness of \rt techniques.
Some are used almost universally for a certain kind of challenge (e.g. APFD for \tcp), while others have nearly no presence beyond the paper that introduced them.

The abundant use of APFD and its variants indicate that, at least among researchers, there is a consensus of its utility and importance when evaluating \tcp approaches, although the usage of specific variants might hamper that benefit.
At the same time, it is not clear that a technique optimized for only APFD is sufficient to satisfy the needs of software developers in practice.
Still, APFD has been in use for over 20 years and it cannot simply be dismissed: at the very least it provides an agreed-upon method of directly comparing different techniques.

For the cases of \tcs and \tsr, there is less controversy on what are the most important metrics; reduction rate and fault detection loss appear to be the consensus among researchers, and there are fewer novel and single-use metrics.
As an example,~\citetalias{mehta_data-driven_2021} interviewed practitioners at Microsoft before deciding on their \tcs metrics, obtaining three main targets: reduction of cost, reduction of time and the failure detection rate.
We can observe in~\Cref{sec:rq1} that these concerns are reasonably addressed by \tcs techniques, although researchers still appear to prioritize reducing the selected set rather than ensuring all failures are detected.

The metrics of applicability and diagnosability \citepalias{correia_motsd_2019, zhou_beating_2020} are interesting propositions that consider other degrees of usefulness of a tool to developers.
Their existence indicates that some researchers still believe there is room for improved metrics that, perhaps, better map the requirements of real-world software, although these are rarely found in the literature.
Furthermore, ease-of-use is an important point to consider and, as far as we could detect, there is no established method of measuring it.

One practitioner stated: ``\textit{I don't think that academic tools are the best in a professional environment, I prefer commercial tools,}'' implying they believe academics are not measuring the results that matter most to them.
Indeed, managers allocating development funds will usually focus on the dollar savings a technique can bring, regardless of its theoretical effectiveness in fault-finding (as mentioned by respondent author \#43).

% ===================
\paragraph{CH5: Converting research into usable tools}
When techniques are designed in an academic context, they are normally developed as proof-of-concept works.
That is, the purpose is to show that the technique works and provides significant results according to some metrics.
However, this leads to two issues: either primary studies do not make their solution available for implementation, as we discussed in~\Cref{sec:rq2}, or their experiments do not thoroughly consider practical concerns such as efficiency or the data requirements of a proposed approach.
Finally, what seems to matter the most is time and budget for developing a tool.
Papers are usually written targeting a hard deadline and their prototypes often do not see further work past publication.
It is inevitable that researchers will move on to new challenges, but their contribution would be amplified if the tool is, at the very least, open-source and well-documented so that other interested parties can continue the work in the future if desired.

If an \rt technique is implemented as a prototype that is shown to work on a certain kind of software, it is much easier to get the attention from a practitioner and convert the solution into something used in practice.
If feasible, an available prototype with solid documentation and usage instructions can be valuable both for study replicability and as a way to get developers interested in using it. 
That said, the responsibility of developing fully functional tools should not fall solely upon researchers.
One practitioner stated that ``\textit{[\rt tools] need full security screening}'', and other said ``\textit{it requires an adaptation}''; these steps are not actionable by researchers in isolation.
As industry stands to benefit from scientific advances, it should be in its best interest to promote and fund the collaborations needed to continue development of promising prototypes.

\todo{Whose responsibility is it? A PhD student for example could work on this, but are there academic/publishing motivations for it?}

% ===================
\paragraph{CH6: Absence of \tsr/\tsa}
Out of \numpapers papers, only 8 are about \tsr and, surprisingly, only one covers \tsa 
 \citepalias{yoshida_fsx_2016}.
60\% of the surveyed practitioners claim that ``creating or updating tests'' is a major challenge in real-world \rt, so the desire for \tsa exists and there appears to be ample room for experimenting with new approaches and metrics.
47\% also mention the difficulty of refactoring and removing obsolete test cases as a pain point, which is something \tsr could remedy.
This can be an opportunity for researchers to develop novel methods and to progress in directions that are in need of exploration.

% ===================
\paragraph{CH7: Clarity of target}
Several of the papers we reviewed don't clearly state key characteristics of their SUT, such as its programming language or its scale (either in lines of code or test cases).
For practitioners and other researchers to consider a paper worthy of investigation, it is important to know for which kind of system a piece of research was designed.

As mentioned in~\Cref{sec:lit_rq2}, few \rt techniques are language-agnostic and many do not inform the target language at all.
Similarly, the type of software (web, mobile, embedded, distributed, etc.) or its development paradigm are important factors to mention, seen in studies such as~\citetalias{zhong_testsage:_2019} for web services and~\citetalias{lima_multi-armed_2022} for software developed and delivered through continuous integration.
Not every tool can be used in any type of software, and it is likely that specific types of software might require specific solutions, so it is important to state the particularities of certain subject programs.
This is akin to the point of ``context factors'' brought up by~\citet{bin_ali_search_2019}, which helps to alleviate the issue by introducing a base taxonomy that can be used to categorize techniques.

Critically, there is often ambiguity on the very definition of test case.
Software testing can include unit tests, integration tests, multi-component tests, system tests, end-to-end tests and so forth.
Most papers do not make it explicit which layer of testing it is addressing. While it can sometimes be inferred with some domain knowledge, it is difficult to be certain for most readers.
This information would be valuable for interested practitioners and also for researchers who are looking to identify gaps in the literature.
On top of that, some papers use the term ``test case'' to refer to test methods, while others use it when referring to test classes/files (which contain several test methods), so the granularity of the technique is not always clear, and this can impact both effectiveness and efficiency analysis.
This challenge can be solved by having a paragraph dedicated to explicitly describing the properties and context factors of the experiment subjects.

\paragraph{CH8: Skepticism}
In general, there is some degree of skepticism from practitioners regarding automated solutions.
For example, developers are concerned that a TCS solution might leave out an important test, or that TCP algorithms might always prioritize the same tests.
Developers are opposed to outright removing tests flagged by TSR, but are willing to put these tests in less-frequent rotations.
This calls back to a comment made by one of the responding authors in \Cref{sec:lit_rq2}: even if 99\% of faults are detected, the remaining 1\% of slips is unnerving to the people responsible for the tests.
\todo{Continue here}

\paragraph{CH9: Data availability}
\todo{Talk about the lack of data}

\paragraph{CH10: Communication}
The main challenge, which connects most of the previous ones, is communication.
Researchers and practitioners both lead busy lives, focusing on their day-to-day affairs, and ultimately communication between the two realms suffers.

There are some steps that can be taken to improve this.
Companies can start by having round-table discussions on recent research publications (e.g. the Google Journal Club \cite{googlejournal}) and, if possible, they should invite the author(s) to participate.
On the other side, universities can host lectures by practitioners in addition to other researchers.
This can start small — find people in the same city, perhaps alumni of the university, who are working on something interesting and have a conversation.

56\% of responding practitioners claimed they keep contact with a friend or colleague who is a researcher in Software Engineering.
After all, most academics have interacted with people who are currently practitioners during their education process, and vice-versa.
This means that both sides have an opportunity to network and communicate beyond their current professions, giving each other ideas of what is currently relevant in industrial software development and what is the latest state-of-the-art in academic research.

It can be a daunting idea to catch up to latest research trends, so larger companies could consider having employees dedicated to understanding the internal processes and challenges while searching for collaborations with academics.
Many researchers would be thrilled to receive a message inviting them for a joint effort with palpable outcomes.
%----------------------------------------------------------------------------------------
%	Live Repository
%----------------------------------------------------------------------------------------
\chapter{Live Repository}\label{chap:live}
\lhead{\emph{Live Repository}} % Set the left side page header to "Introduction"

The group of papers discussed in \Cref{chap:literature_review} is just the initial contents of the live repository that is made available online.
Over time, through updates to the review and submissions by authors, we expect this list to grow.


\section{Summary}
Finally, as the paper title indicates, this review is conceived
as a \textit{live} systematic review\footnote{We notice that our concept of a ``live'' systematic review, while inspired by similar aims, should not be confused with the much more formalized approach for conducting \textit{living systematic reviews} recently adopted in medicine, as illustrated by \url{https://community.cochrane.org/review-production/production-resources/living-systematic-reviews}.}. 
In our analysis of recent secondary studies, we noticed that a gap of one or even two years elapses between the covered period of literature and the year the review appears. 
This is understandable because the authors  may need substantial time for analyzing the selected studies and then writing the article, and then several months will typically be taken for the peer reviewing process. 
Moreover, even if this temporal gap is reduced to a minimum, as long as the investigated topic remains active, new primary studies will always appear, soon distancing any SLR from the status of literature.  
If the purpose of a SLR is to provide an up-to-date summary of existing work on a topic, frequent updates are required to include newly appearing relevant studies.
Also, it can happen that the original research questions and findings lose relevance, or are superseded by newer results.
As previously questioned in other disciplines~\cite{Garneri3507}, Mendes et al.~\cite{MENDES2020110607} have recently investigated the issue of SLRs in SE becoming obsolete and of when and how they would  need to be updated. 
Specifically, in line with~\cite{Garneri3507}, they conclude that SLRs should be updated based on two conditions: \textit{i)} new relevant methods, studies or information become available; or \textit{ii)} the adoption or inclusion of previous and new research cause an impact to the findings, conclusions or credibility of the original SLR.
In consideration of these challenges, rather than providing a static repository of the studies found while conducting the review, together with the article we release an \textit{open and updatable repository}, which is an integral part of this review work.
 
The intent is to invite the community to contribute with 
newly published works that are relevant in \rea approaches to \rt, or even with works published in the period we cover but that for some reason escaped our selection.
This will allow us not only to keep track of newly published studies, 
but also to recover papers with purely theoretical motivations that eventually provided benefits in practice.
At periodic intervals (planned to be once per year), we will check new results for queries and suggestions from the community and decide how to update the collection of studies.

\section{The Repository}
Literature reviews provide important information to researchers starting out in a field or practitioners who are curious to know the latest innovations, but do not have time to fully explore journals and conferences.
However, it is inevitable that a literature review such as this becomes outdated after some time, as new research comes out that cannot be included in the published paper.
This of course limits the long-term value of the work, since the text will no longer reflect the ongoing research in the field.

In order to aggregate long-term value to this work, we have made the list of papers and the information extracted from them available as an online live repository\footnote{Available at: \url{https://renangreca.github.io/literature-repository}.}.
The papers in \Cref{table:selected} serve as the starting point for a list that will continue to grow year over year.
We hope this website will serve as reference to anyone who is interested in practical applications of regression testing techniques in the coming years.

The main challenge is how to keep this repository alive in the long term.
It is unfeasible for us to add a relevant paper to the repository as soon as it is published, so our plan is to update the list in a yearly basis, re-running the query and screening steps detailed in \Cref{sec:methodology}.
That way, we can at least assure the most recent paper included is no more than one year old.
We are also looking into the possibility of getting automatic notifications when a paper that satisfies certain criteria is published in an online library.
For now, this work is done by the original authors of the literature review; according to future necessities, we will appoint other researchers or graduate students to help with the process.
In addition, we also encourage authors to submit their own work by filling a form linked on the website.

The repository also contains a separate section for relevant literature reviews.
This is initially populated by the reviews mentioned in \Cref{sec:related} and, upon publication, this very document.
With this we aim to provide a starting point for new researchers and a place to gather the overarching themes of the field.

It can also happen that, over the years, the definitions we selected for including a paper in the repository must be adjusted.
Whenever an author submits a paper, we will use the opportunity to consider whether or not the paper itself is a good fit for the repository, but also if there are new trends that our existing selection process does not account for.
There will likely be a point in the future when the industry/academia landscape has shifted and this study will no longer be needed.
When that happens, we will discuss the possibility of freezing the repository and stopping further expansions.

Aside from newer papers, it is always possible that we have missed some relevant papers for a variety of reasons, so the live repository is another way of mitigating that risk.
It is impossible to provide a complete and definitive overview of any field, but we believe that a live repository is the closest approximation that can be expected.

\todo{we want to keep the information summarized for future researchers/practitioners to gather data}

\todo{we want to keep it updated as to not need another systematic literature review on the same topic}

%\begin{figure}
%  \center
%  \includegraphics[width=\linewidth]{live_repository_screenshots_4.png}
%  \caption{Screenshots from the live repository. From left to right: 1) the main page listing the included papers; and 2) a single paper's page (\citetalias{srikanth_requirements_2016} used as example). }
%  \label{fig:live_repository}
%\end{figure}
%----------------------------------------------------------------------------------------
%	Conclusion
%----------------------------------------------------------------------------------------
\chapter{Conclusion}\label{chap:conclusion}
\lhead{\emph{\nameref{chap:conclusion}}}

This thesis provides data, insight and discussions centered around test suite orchestration.
Based on an investigation of the literature, a proof-of-concept implementation of orchestration, and \textit{in loco} interviews with software testing practitioners,
the result is a thorough investigation on the potential benefits of orchestration as well as the challenges that are currently faced by researchers and practitioners who wish to implement these techniques on real-world software.

In \Cref{chap:literature_review}, we provide a comprehensive literature review of academic papers with proven or apparent potential for real-world software usage.
We have shown that applicability is a growing concern in research but, when it comes to actually applying the techniques with an industrial partner, many challenges arise and very few approaches feature lasting longevity.

\Cref{chap:orchestration} shows an example of how test suite orchestration can be used to enhance software testing procedures, incorporating change-based \tcs and similarity-based \tcp  approaches as an initial step towards the goal of full orchestration.
Analysis is performed on the approach, which exhibits promise both in terms of effectiveness and of efficiency.
The challenge of full orchestration is left open, with a descriptive example of how multiple \rt could be combined.

First-hand accounts of the state of software testing in practice are quoted and discussed in \Cref{chap:industry}.
Thanks to the opportunity of collaborating with an industrial partner, we were able to understand the processes and challenges that exist today in software testing.
Reports indicate that testing is far from being a solved problem, showing there is ample opportunity for new techniques to improve workflows and reduce costs in the industry.

The results of \Cref{chap:literature_review}, \Cref{chap:orchestration} and \Cref{chap:industry} are used to form a list of essential challenges in \Cref{chap:gap} that researchers should consider and address if their goal is to provide techniques that are usable by practitioners.

To add longevity to the relevance of this work, \Cref{chap:live} describes the process of converting the literature review from \Cref{chap:literature_review} into a \textit{live} systematic literature review.
This repository of studies is available online and provides a starting point and bibliography for researchers to wish to pursue this research direction.

Ultimately, many of the challenges we investigate are as of yet unresolved, remaining a fertile direction for future research.
It is clear that software engineering academics and practitioners have distinct goals when it comes to the work they do to improve the quality and efficiency of testing, but there is also plenty of evidence indicating that it is possible to align these motivations and produce even better techniques as a result.

The closing message of this thesis is one of community and collaboration: in order to ensure that software testing research leads to tangible improvements to the lives of developers and users of computer software, it is essential that academia and industry form a tighter bond to explore each other's advantages and cover their weaknesses.
It is our sincere belief that such collaboration is possible and will at some point be achieved.
We hope that the conclusion and publication of this study will help this happen sooner rather than later.

This chapter contains two additional sections: \Cref{sec:threats} lists the threats to the validity of relevant parts of this study, and \Cref{sec:publications} briefly discusses the papers that were submitted and/or published during the development of this PhD thesis.

%There is still much work to be done by the software engineering research and development communities in order to completely close the gap that exists between them.
%To a great extent, the motivations of researchers and practitioners are not aligned --- while in academia, proposing theoretically sound novel approaches is encouraged to obtain publications, in industry there is a need for techniques that are proven to reduce effort and/or costs.
%This can only be solved by close collaboration between the two sides, yet a question of who is willing to fund these experiments remain.
%The data and discussions provided in this thesis show that, although difficult, this is not an impossible problem to solve and there are certain clear steps that can be taken by researchers and practitioners alike to begin addressing it.

\section{Threats to Validity}\label{sec:threats}

There are certain threats to the validity of Chapters \ref{chap:literature_review}, \ref{chap:orchestration}, \ref{chap:industry} and \ref{chap:gap}.
To the extent of possibility, we have worked to mitigate those threats and produce a robust, accurate, consistent and replicable study.
Nonetheless, the threats we have identified are listed and discussed as follows.

\subsection{Threats to \Cref{chap:literature_review} }

\paragraph{Construct validity} Despite our efforts to comprehensively find all primary studies that meet our selection criteria, we might have missed some.
To mitigate this threat, we performed a systematic search over five broad digital libraries and complemented the search with a snowballing cycle and a check with authors of all found studies, who in fact suggested a few additional entries.

As usual for this kind of study, our selection of papers was performed through queries, followed by manual filtering.
To diminish potential bias of the latter step, the filtering process was systematically reviewed and agreed upon among all the three authors.

\paragraph{Internal validity} The internal validity of this study is strongly dependent on the three research questions that guided all our analysis as well as the data extraction form we  built.
We took great care in ensuring that they properly reflect our objectives, although it is unavoidable that, by formulating  different questions or using other data extraction forms, we could have obtained other results.
We might also have overlooked or misinterpreted some important information or arguments in the primary studies, beyond our best efforts and accuracy in the full reading of all selected papers.
To mitigate such threats we provide all extracted data in traceable format, highlighting the main points we extracted from each primary study.
Furthermore, the responses we received directly from authors often provide additional context that reduce the risk of misinterpretation.
That said, we cannot make the full responses available due to non-disclosure requests from some authors.

\paragraph{Conclusion validity}
The conclusions we drew in terms of  the information  we summarize from the primary studies, the detected challenges we discuss in the above section and the recommendations we formulate in the conclusions might have been influenced by our background, and other authors might have reached different conclusions.
Such potential bias is unavoidable in this type of study, however we tried to mitigate it by aiming at full consensus of all authors behind each conclusion.
Furthermore, by documenting in detail the data extraction process, we ensure a fully transparent study that can be verified and replicated.
The survey sent to practitioners helps to validate our conclusions.
Although the sample of 23 responses is very small, it shows a degree of alignment among people working in six different countries.
A convenience sample was used to distribute the survey; thus, the practitioners we reached are more likely to have some contact with ongoing research.
To avoid excessive bias in that direction, we did not contact members of industry who are known to regularly publish in Software Engineering events.

\paragraph{External validity}
We do not make any claim of validity of our conclusions beyond the \numpapers papers analyzed.
As more primary studies are published, they should be read and analyzed on their own, and our conclusions should be revised accordingly. In consideration of this threat, in the aim of ensuring validity even in future, we are committed to keep the live repository up-to-date, taking into account the community inputs.
Moreover, we believe that the framework we developed consisting of the three research questions, the data extraction form and the structured tables for summarizing the approaches and the metrics could be still applicable also by other external authors.

\subsection{Threats to \Cref{chap:orchestration} }

We evaluated \fz using faults available in \dfj.  Our results and
conclusions could be different had we used another bug repository.
However, \dfj is among the most popular bug repositories and is
heavily used in research on regression testing.  Additionally, it
includes real faults, which strengthens our findings.

The fact that we use \dfj means that we were running experiments on
project versions that are potentially very far apart (e.g., years).
In this setup, \ek might select a very large number of tests, because
it was designed for small code changes between two
consecutive commits~\cite{gligoricEk, VasicETAL17EkstaziSharp}.  However, 
\ek ended up performing well
even in our setup.

We defined the testing budget as the number of tests that one can run
at each project version, which does not take into account the
differences in individual test execution time.  As we focus on unit
tests, we do not expect that there would be substantial differences in
execution time across tests.

To measure effectiveness, we used \ttff and \apfd. As known the \dfj subjects contain only one fault per version and hence the two measures behave similarly. 
To mitigate this issue, we need to perform more studies on subjects containing multiple faults, for which the \apfd measure becomes more valuable. 

In our experiments we assume that test execution is deterministic,
which we know  does not always hold in practice, i.e.,
tests are flaky~\cite{luo2014empirical,harman2018start}.
We have not observed any flaky behavior in our
experiments: only the expected set of tests was failing in each run.

\subsection{Threats to \Cref{chap:industry} }

\paragraph{Construct validity}
The observations in \Cref{chap:industry} were only possible due to a collaboration with an European technology company, facilitated by a person who is in frequent contact with both the company and the academic world.
It is important to note that the findings relate to an experience of only seven weeks at only one office of the company.
We did our best efforts to understand the internal testing strategies and workflows, but the overall complexity of the processes is too high to be completely comprehended in a short timeframe.

\paragraph{Internal validity} 
The interviews that were conducted for this chapter were based on the research questions presented here.
These research questions were designed to reflect our goals when designing that part of the research, and inevitably narrow the possible findings and guide the general outcome of the study.
The quotes included in this chapter are transcribed from verbal interviews with practitioners and edited by the author for legibility, clarity and cohesiveness with the text.
It is possible that a response was misinterpreted by the interviewer or that the meaning of a quote is not fully clear to a reader.
We took great care of transcribing the interviews as accurately as possible, and of ensuring that their meaning was not altered by cherry-picking quotes or adjusting their wording.
Mitigating this risk, the chapter has been read and approved by two people from the industrial partner, ensuring their point of view is correctly expressed.
Due to a non-disclosure agreement with the industrial partner, we cannot provide the full unedited transcripts of the interviews.

\paragraph{Conclusion validity}
The conclusions drawn from the interviews are a result of our own interpretation of the situation, based on the observed data, on discussions with members of the company, and related information extracted from the software engineering literature.
Inevitably these conclusions are influenced by the background of the researchers who, to the degree of possibility, allowed the newfound information to shape the conclusions, not the opposite.
Unfortunately, this study is not easily replicable, as core components of it are left undisclosed due to confidentiality concerns by the industrial partner. 

\paragraph{External validity}
These observations relate to one team at one office at a large company.
The conclusions we draw relate solely to that team and might not generalize to the practices observed in the rest of the company.
Furthermore, we make no claims that these findings generalize to software industry as a whole.
This threat can only be mitigated if more companies are willing to allow researchers to interview their employees and understand their testing processes and challenges.

\subsection{Threats to \Cref{chap:gap} }

The list of challenges assembled in \Cref{chap:gap} is based on our own observations of the current state of industry-academia relations from a multitude of sources: the literature, communications with other authors, feedback from practitioners and our own experiences developing techniques.
It is not meant to be a comprehensive and end-all checklist of challenges to solve, as other researchers following different sources would likely come to a divergent set of conclusions.
That said, we believe that most researchers performing a similar study would agree upon the majority of the listed challenges.
To the extent of our knowledge, these challenges are real and in need of further study, but there is no guarantee that addressing each one of them will solve all the problems with software engineering research.

\section{Publications}\label{sec:publications}

During the development of this thesis, the following papers were accepted for publication at software engineering conferences or journals.


\begin{refsection}[P]
	\newrefcontext[labelprefix=P]
	\nocite{*}
    \printbibliography[heading=P]
\end{refsection}


%The following papers produced and published during performed during this thesis.

%\nociteP{rossi2020defensive_p}

%\nociteP{greca_comparing_2022_p}

%\bibentry{greca_comparing_2022}
%
%\nociteP{greca_live_2022_p}

%\section{Conclusion}

%\input{chapters/intro}
%
%\input{chapters/background}
%
%\input{chapters/sota}
%
%\input{chapters/proposal}

%----------------------------------------------------------------------------------------
%	THESIS CONTENT - APPENDICES
%----------------------------------------------------------------------------------------

\addtocontents{toc}{\vspace{2em}} % Add a gap in the Contents, for aesthetics
%
\appendix % Cue to tell LaTeX that the following 'chapters' are Appendices

%% Include the appendices of the thesis as separate files from the Appendices folder
%% Uncomment the lines as you write the Appendices

%\lhead{\emph{Appendix A}} % Set the left side page header to "Appendix A"
%%----------------------------------------------------------------------------------------
%	Appendix A
%----------------------------------------------------------------------------------------
\chapter{Tables}\label{app:tables}
\lhead{\emph{Appendix A}} % Set the left side page header to "Introduction"

\todo{Write your Appendix content here, if needed.}
%----------------------------------------------------------------------------------------
%	Appendix B
%----------------------------------------------------------------------------------------
\chapter{Surveys}\label{app:surveys}
\lhead{\emph{Appendix \ref{app:surveys}: \nameref{app:surveys}}}

%\todo{Write your Appendix content here, if needed.}

\section{E-mail Template Sent to Authors of Surveyed Papers}
\label{sec:app_email}

\section{Questionnaire Sent to Practitioners During Literature Review}
\label{sec:app_prac_questionnaire}

\section{Questions for Practitioners at the Industrial Partner}
\label{sec:app_ind_questions}

\begin{table}[]
\centering
%\scriptsize
\rowcolors{1}{}{gray!10}
%\setlength{\tabcolsep}{6pt}
\begin{tabular}{p{0.05\linewidth}p{0.9\linewidth}}
\toprule
\textbf{Nº} & \textbf{Question} \\
\midrule
\textbf{1} & \textbf{Roles \& education} \\
\textbf{1.1} & What is your role at the company and what are the main ways you interact with the regression testing suite? \\
\textbf{1.2} & What is your education in testing? \\
\midrule
\textbf{2} & \textbf{Are these common issues when dealing with the test suite?} \\
\textbf{2.1} & Correctly detecting failures (avoiding slips) \\
\textbf{2.2} & Flaky (unreliable tests) \\
\textbf{2.3} & Running time of the test suite \\
\textbf{2.4} & Creating or updating tests (increasing test scope) \\
\textbf{2.5} & Cleaning up obsolete tests (reducing test scope) \\
\midrule
\textbf{3} & \textbf{Current practices} \\
\textbf{3.1} & Are you familiar with these terms? Do you use them? \\
\textbf{3.1.1} & Test Case Selection \\
\textbf{3.1.2} & Test Case Prioritization \\
\textbf{3.1.3} & Test Suite Reduction \\
\textbf{3.1.4} & Test Suite Augmentation \\
\textbf{3.2} & Does your team separate test execution into shorter, frequent runs and longer overnight runs? \\
\textbf{3.3} & Are there people in your team solely responsible for testing, or is it a job of the developer? \\
\textbf{3.4} & Do you believe the current test suite is effective in avoiding fault propagation upon delivery of an update? \\
\textbf{3.5} & How often are there failing tests from the overnight run in the morning? \\
\textbf{3.6} & How often is a test failure a sign of actual SUT error vs. poor test quality vs. environment misconfiguration? \\
\textbf{3.7} & What do you do when the same test fails many days in a row? \\
\textbf{3.8} & What is the standard procedure for dealing with potentially flaky tests? \\
\textbf{3.9} & Is there a policy in place for refactoring and/or removing aging tests? \\
\textbf{3.10} & Do you have an idea of how much money or how much electricity is needed to run a test suite? Is this concern increasing or reducing over the years? \\
\textbf{3.11} & Do you personally ever review test code? How much trust do you have that it's being done correctly and at high quality? \\
\midrule
\textbf{4} & \textbf{Interaction with academia} \\
\textbf{4.1} & Do you stay informed about recent research in Software Engineering? \\
\textbf{4.2} & Has a piece of Software Engineering research directly improved your workflow? \\
\textbf{4.3} & Do you keep contact with a friend or colleague doing research in Software Engineering? \\
\textbf{4.4} & Before me, have there been attempts of collaboration between members of your team and academic researchers? \\
\textbf{4.5} & What does a tool need to do/have in order to convince you to use it as a permanent part of your workflow? \\
\textbf{4.6} & What are the risks and benefits of using a new tool for regression testing improvement? \\
\textbf{4.7} & Have you ever used an open-source regression testing approach as-is, or is it always necessary to re-engineer it for usability/security/compatibility reasons? \\
\textbf{4.8} & If I came to you saying ``I have this algorithm that would help your testing workflow'', what would be your reaction? \\
\bottomrule
\end{tabular}\\
\caption{List of questions asked during interviews.}
\label{table:interview_questions}
\end{table}

\addtocontents{toc}{\vspace{2em}} % Add a gap in the Contents, for aesthetics

\backmatter

\clearpage

%----------------------------------------------------------------------------------------
%	BIBLIOGRAPHY
%----------------------------------------------------------------------------------------

\label{Bibliography}

%\newcites{P}{Papers related to this thesis}
%\newcites{C}{Conferences related to this thesis}

%\fullcite{greca_comparing_2022}

%\bibliographystylec{IEEEtran}
%
%\nocitec{greca_comparing_2022}

%\bibliography{bibliographies/production.bib}

%\printbibliography

\lhead{\emph{Bibliography}} % Change the page header to say "Bibliography"

%\bibliographystyleP{unsrtnat}
%\bibliographyP{production.bib}

%\bibliographystyleS{unsrtnat}
%\bibliographyS{slr.bib}

%\nociteS{srikanth_requirements_2016}
\nociteS{noor_similarity-based_2016}
\nociteS{schwartz_cost-effective_2016}
\nociteS{hirzel_graph-walk-based_2016}
\nociteS{lu_how_2016}
\nociteS{vost_trace-based_2016}
\nociteS{wang_enhancing_2016}
\nociteS{srikanth_test_2016}
\nociteS{blondeau_test_2017}
\nociteS{pradhan_search-based_2016}
\nociteS{buchgeher_improving_2016}
\nociteS{tahvili_dynamic_2016}
\nociteS{oqvist_extraction-based_2016}
\nociteS{magalhaes_automatic_2016}
\nociteS{aman_application_2016}
\nociteS{busjaeger_learning_2016}
\nociteS{yoshida_fsx_2016}
\nociteS{tahvili_cost-benefit_2016}
\nociteS{ramler_tool_2017}
\nociteS{strandberg_experience_2016}
\nociteS{marijan_effect_2016}
\nociteS{gotlieb_using_2017}
\nociteS{chi_multi-level_2017}
\nociteS{bach_coverage-based_2017}
\nociteS{spieker_reinforcement_2017}
\nociteS{vasic_file-level_2017}
\nociteS{celik_regression_2017}
\nociteS{ouriques_test_2018}
\nociteS{kwon_cost-effective_2017}
\nociteS{garousi_multi-objective_2018}
\nociteS{shi_evaluating_2018}
\nociteS{haghighatkhah_test_2018}
\nociteS{zhang_hybrid_2018}
\nociteS{miranda_fast_2018}
\nociteS{yilmaz_case_2018}
\nociteS{chen_optimizing_2018}
\nociteS{celik_regression_2018}
\nociteS{zhu_test_2018}
\nociteS{azizi_retest_2018}
\nociteS{guo_decomposing_2019}
\nociteS{zhong_testsage:_2019}
\nociteS{fu_resurgence_2019}
\nociteS{eda_efficient_2019}
\nociteS{goyal_test_2019}
\nociteS{yu_terminator_2019}
\nociteS{correia_motsd_2019}
\nociteS{machalica_predictive_2018}
\nociteS{najafi_improving_2019}
\nociteS{leong_assessing_2019}
\nociteS{cruciani_scalable_2019}
\nociteS{philip_fastlane:_2019}
\nociteS{magalhaes_hsp_2020}
\nociteS{wu_time_2019}
\nociteS{land_industrial_2019}
\nociteS{noemmer_evaluation_2020}
\nociteS{lubke_selecting_2020}
\nociteS{yackley_simultaneous_2019}
\nociteS{shi_understanding_2019}
\nociteS{lima_multi-armed_2022}
\nociteS{zhou_beating_2020}
\nociteS{peng_empirically_2020}
\nociteS{bertolino_learning--rank_2020}
\nociteS{chen_multi-objective_2021}
\nociteS{zarges_artificial_2021}
\nociteS{bagherzadeh_reinforcement_2022}
\nociteS{elsner_empirically_2021}
\nociteS{pan_dynamic_2020}
\nociteS{mehta_data-driven_2021}
\nociteS{xu_requirement-based_2021}
\nociteS{zhou_parallel_2022}
\nociteS{sharif_deeporder_2021}
\nociteS{li_aga_2021}
\nociteS{chen_context-aware_2021}
\nociteS{zhang_comparing_2022}
\nociteS{abdelkarim_tcp-net_2022}
\nociteS{cingil_black-box_2022}
\nociteS{yaraghi_scalable_2022}
\nociteS{omri_learning_2022}
\nociteS{greca_comparing_2022}


%\begin{refsection}[S]
%    \printbibliography[heading=S]
%\end{refsection}

%\begin{refsection}[S]
%	\newrefcontext[labelprefix=S]
%%	\nocite{*}
%    \printbibliography[heading=S]
%\end{refsection}

\newrefcontext[sorting=nyt]
\printbibliography
%\bibliographystyle{unsrtnat} % Use the "unsrtnat" BibTeX style for formatting the Bibliography
%\bibliography{secondaries.bib,bibliography.bib,production.bib,slr.bib} % This contains only the 
%\bibliography{}

\end{document}