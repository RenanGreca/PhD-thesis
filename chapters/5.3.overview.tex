\section{Overview of Testing at Ericsson}\label{sec:ind_overview}

Regression testing is a topic that has been extensively studied for decades.

It includes topics such as \textit{test case selection}, \textit{test case prioritization} and \textit{test suite reduction}.



\subsection{Regression Testing in the Literature}

Most academic studies are focused on unit tests or lower-level integration tests.

\subsection{Regression Testing at Ericsson}

In the mid-2000s there was a shift to agile development which affected the testing workflow.

Ericsson employs a ``test strategy pyramid'' that starts with unit tests at the lower end and goes up to network tests at the high end.

A substantial part of the testing cost comes from the high-level network tests, which are complex and expensive, as they involve multiple layers of software as well as physical and simulated hardware.

There is a ``shift left'' objective that aims to bring fault-finding into lower levels of the pyramid.

There are frequently executed ``checks'' (that may employ some sort of test case selection) and overnight ``tests''.

When an overnight test fails, it is often due to test flakiness or environment misconfiguration. Often it is not an error in the SUT proper.

In the integration/multi-component level, it is often the case that higher-level/more complex tests cover lower-level/simpler functionality by definition. However, there is no system in place to refactor and remove older tests that are no longer necessary.
