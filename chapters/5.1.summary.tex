\section{Summary}\label{sec:ind_summary}

As seen in the discussion from \Cref{chap:literature_review}, not many ideas proposed in academia make their way into practical usage.
To dig deeper into this problem and understand the underlying challenges, a crucial part of this research involves direct communication with members of industry who work on software testing.

Upon contact, members from a large technology company granted the opportunity to spend some time at their offices to observe practices, collect data and interview team members.
This chapter synthesizes the findings of a seven-week period which was spent in direct contact with the company.

The interacted team is responsible for a software system, which is an integral component of the company's delivered product.
Specifically, practices related to multi-component testing (MCT) were investigated; at the time, there were thousands of MCTs in the system.

The interviews cover a variety of topics, including education in software testing, current practices and procedures employed by the team, their relationship with members and research from academia, and the most notable challenges they face.
Analysis of the responses show that there is a strong desire to improve processes, which is hampered by reasons including technical challenges, bureaucratic hurdles and even skepticism by some team members of automated solutions.

\Cref{sec:ind_rqs} describes the questions that drive the discussions brought here.
\Cref{sec:ind_overview} provides an overview of the testing process at the company and of the system for which the interviewed team is responsible.
\Cref{sec:ind_interviews} brings quotes from the interviews to form a picture of the state of practice from the perspective of the practitioners themselves.
Finally, \Cref{sec:ind_observations} contains the answers to the research questions, in the form of a series of observations derived the interviews.

%Recently, there has been research specifically about the ``Industry-academia gap''.

%In this paper, we aim to bring more light into this topic by drawing some comparisons between points frequently discussed in the literature and the actual state of practice in an industrial partner.

%Section \ref{sec:background} gives some background on regression testing. Section \ref{sec:rqs} contains the research questions and their discussions. Section \ref{sec:related} highlights relevant related work. Section \ref{sec:threats} points potential threats to the validity of this study. Finally, \ref{sec:conclusion} offers our closing thoughts.
