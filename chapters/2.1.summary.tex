%\section{Summary}\label{sec:background_summary}

First of all, it is important to clarify the definitions of two sets of terms that are related, but offer distinct nuance to the discussion of software testing techniques.

\paragraph{Failure, fault and error: } these terms are widely used throughout this thesis.
While they may seem interchangeable, there are key distinctions that must be highlighted, as defined by the IEEE standard 1044-2009~\cite{IEEE1044-2009}.
A \textit{failure} occurs when a system produces an incorrect or unexpected outcome, leading to a failing test case --- without further investigation, the cause of the failure might be unknown.
A \textit{fault} is a specific part of the software that is incorrect; e.g. a logical or semantical issue in the source code that, when executed, will produce a failure.
%occurs when the SUT provides an unexpected output and, if the test suite is well-designed, a failure is triggered.
Finally, an \textit{error} is the human action that leads to a fault, such as having written incorrect code, or misinterpreted the system requirements.
Terms like \textit{anomaly}, \textit{defect} or \textit{bug} can be used when referring generally to unexpected software behavior.

\paragraph{Efficacy, effectiveness and efficiency: } these ``three Es'' are commonly used words when describing the results of a technique or experiment.
Here, we follow dictionary definitions closely \cite{dictionary_eff}.
Given a certain task and one or more techniques designed to accomplish it,
\textit{efficacy} is a binary assessment of whether a technique accomplishes the desired task at all;
\textit{effectiveness} provides a more nuanced description of how well the task is accomplished, and can be used to compare multiple alternatives; and
\textit{efficiency} is a quality related to the time, cost and/or effort spent to accomplish the task.