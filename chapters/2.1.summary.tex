\section{Summary}\label{sec:background_summary}

\paragraph{Failure, fault and error: } these terms are widely used throughout this thesis.
While they may seem interchangeable, there are key distinctions that must be highlighted.
A \textit{failure} is simply a test that fails instead of passing — without further investigation, the cause of the failure is unknown.
A \textit{fault} occurs when the SUT provides an unexpected output and, if the test suite is well-designed, a failure is triggered.
Finally, an \textit{error} is the underlying cause of a fault, such as logic or semantics errors in the code, or an ambiguity or misinterpretation of the system requirements.

\paragraph{Efficacy, effectiveness and efficiency: } these ``three Es'' are commonly used words when describing the results of a technique or experiment.
Here, we follow dictionary definitions closely \cite{dictionary_eff}.
Given a certain task and one or more techniques designed to accomplish it,
\textit{efficacy} is a binary assessment of whether a technique accomplishes the desired task at all;
\textit{effectiveness} provides a more nuanced description of how well the task is accomplished, and can be used to compare multiple alternatives; and
\textit{efficiency} is a quality related to the time, cost and/or effort spent to accomplish the task.