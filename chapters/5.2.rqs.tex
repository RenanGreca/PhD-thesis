\section{Research Questions}
\label{sec:ind_rqs}

The following research questions represent the main objectives of the collaborative interaction with the industrial partner.
Generally, the contribution of this period is the identification and explanation of notable challenges involved with industry-academia collaboration and the advancement of practical software testing techniques.

It is worth noting that these are not the questions asked directly to the practitioners (which are found in \Cref{app:surveys}), but they were used a starting point to design the interviews.

\paragraph{RQ\ref{chap:industry}.1}\label{rq:ind1} \textbf{What are the regression testing issues most frequently encountered by software testing practitioners?} --- We want to hear from the practitioners themselves what are the regression testing challenges they face on a daily basis.
It is important to understand whether these challenges are the same that motivate software testing research, or if there are concerns from industry that are yet unaddressed by researchers.

\paragraph{RQ\ref{chap:industry}.2}\label{rq:ind2} \textbf{What are the challenges that arise when trying to incorporate academic insight in practice?} --- Based on their previous experiences with regression testing techniques, we would like to know what worked and what didn't, and for what reasons.
We know beforehand that it is not trivial to convert academic knowledge and insights into a company environment, but we should understand with more details why this conversion is so difficult to accomplish.

\paragraph{RQ\ref{chap:industry}.3}\label{rq:ind3} \textbf{What are potential paths to improve collaboration between academics and practitioners?} --- With this we seek to capture what the practitioners are looking for when they are approached by academics.
This discussion should help researchers propose more relevant collaborative projects, as well as set the expectations of practitioners as to what is feasible to achieve when working together with academics.

%How can academics make their work more accessible for interested practitioners?
