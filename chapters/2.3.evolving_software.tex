\section{Evolving Software Systems}

In the early stages of commercial software development, computer programs were designed, produced and distributed mostly like physical retail goods.
That is, there was an initial planning and design phase, followed by an extensive development period and, on a certain deadline, the software was shipped embedded with hardware, or pressed onto floppy disks or CDs to be made available in shelves.
The advent of the Internet made it possible to completely alter this paradigm.
Now, these three phases still exist in commercial software, but happen much faster and can be repeated iteratively as needed.
In other words, software companies can initially design and develop the ``minimum viable product'' to be delivered to customers online and, with the software already in use, updates can be develop to add new features, improve existing ones, or correct bugs that can be detected\footnote{This is not the same for all types of software; embedded systems cannot always rely on the ability of pushing patches; and video games, for example, generally deliver complete products on a given deadline to account for distribution and marketing schedules, but even they almost always have an extensive post-release update cycle.}.

The shift to evolving software, which correlates to the pivot to agile development practices in the mid-2000s, also caused a significant change to how software testing is viewed and addressed.
Previously, it was common to have team members solely responsible for testing the developers' code, often times manually.

In this work, we are interested in regression testing in industrial settings.
We use ``industrial setting'' as a general term for large-scale software in the real world.
In practice, it can mean several different kinds of software, such as software developed as the primary product of a corporation (in the technology industry), software that provides essential features to other products (such as in the automotive or telecommunication industries), or open-source software that is developed by a community instead of a team within a company.


\todo{Explain context of modern evolving software}


\todo{Perhaps also explain here the categories of software in terms of scale or domain}