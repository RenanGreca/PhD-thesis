\section{Discussion}\label{sec:orch_discussion}

Software regression testing has undergone extensive research in the last several decades.
The largest part of solutions, though, addressed separately one dimension of the problem at a time.
While many \tcs and \tcp techniques have been proposed, they have not been directly compared, only
few authors look into integrated approaches for combined selection and prioritization, and no work empirically assessed the advantages of using \tcs and \tcp in combination over their individual application.
In contrast, we believe that, by merging differing criteria for selection and prioritization, we can achieve the most from the restricted subset of test cases that can be executed at each new release.

Towards this direction, we presented a study directly comparing 
two recent practical and effective approaches to \tcs and \tcp, namely 
file-based selection (by \ek) and similarity-based prioritization (by \fs).
Our results show that \ek generally outperforms \fs, although the effect size is negligible or small;
however, their orchestration by \fz outperforms both with a non-negligible effect. 
Moreover, considering a limited test budget, \fz exposed a higher effectiveness in consistent way. 
After assessing the overhead imposed by each of the studied approaches, we can conclude that \fz is quite practical: if we parallelize the preparation steps, the additional cost of similarity-based prioritization of the test cases selected by \ek is negligible.  

We aim at further improving the effectiveness and efficiency of \fz by refining several technical aspects. 
In particular, to make the approach more easily usable, it should be integrated into build systems as a plug-in as \ek is now.
In addition to that, we would also like to try orchestrating other \tcs and \tcp techniques from the literature to understand the resulting challenges and outcomes.

%More generally, this work paves the way to exploring a full range of potential strategies of combining differing criteria for selection and prioritization. 
%It can be worthwhile to also expand the study to the orchestration of techniques along other dimensions of regression testing, e.g., also test reduction or test amplification.
%Overall, we consider that for maximized efficacy under restricted budgets the problem of regression testing should be addressed in a holistic strategy that we called regression test orchestration. 

\subsection{Future directions for test suite orchestration}

\todo{Discuss possible ways of combining other \tcs and \tcp techniques along with \tsr and \tsa}