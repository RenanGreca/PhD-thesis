\section{Discussion}\label{sec:orch_discussion}

Software regression testing has undergone extensive research in the last several decades.
The largest part of solutions, though, addressed separately one dimension of the problem at a time.
While many \tcs and \tcp techniques have been proposed, they have not been directly compared, only
few authors look into integrated approaches for combined selection and prioritization, and no work empirically assessed the advantages of using \tcs and \tcp in combination over their individual application.
In contrast, we believe that, by merging differing criteria for selection and prioritization, we can achieve the most from the restricted subset of test cases that can be executed at each new release.

Towards this direction, we presented a study directly comparing 
two recent practical and effective approaches to \tcs and \tcp, namely 
file-based selection (by \ek) and similarity-based prioritization (by \fs).
Our results show that \ek generally outperforms \fs, although the effect size is negligible or small;
however, their orchestration by \fz outperforms both with a non-negligible effect. 
Moreover, considering a limited test budget, \fz exposed a higher effectiveness in consistent way. 
After assessing the overhead imposed by each of the studied approaches, we can conclude that \fz is quite practical: if we parallelize the preparation steps, the additional cost of similarity-based prioritization of the test cases selected by \ek is negligible.  

We aim at further improving the effectiveness and efficiency of \fz by refining several technical aspects. 
In particular, to make the approach more easily usable, it should be integrated into build systems as a plug-in as \ek is now.
In addition to that, we would also like to try orchestrating other \tcs and \tcp techniques from the literature to understand the resulting challenges and outcomes.

%More generally, this work paves the way to exploring a full range of potential strategies of combining differing criteria for selection and prioritization. 
%It can be worthwhile to also expand the study to the orchestration of techniques along other dimensions of regression testing, e.g., also test reduction or test amplification.
%Overall, we consider that for maximized efficacy under restricted budgets the problem of regression testing should be addressed in a holistic strategy that we called regression test orchestration. 

\subsection{Existing Examples of Test Suite Orchestration}

Although the nomenclature of Test Suite Orchestration is recent, the concept behind it is not entirely new.
The reader might have observed that the literature review in \Cref{chap:literature_review} contains some examples of papers that address some combination of \tcp, \tcs, and \tsr --- thus wondering if these are previous examples of orchestration in the literature.
This is true in certain cases, although not always.

Particularly, orchestration refers to strategies combining different approaches to certain aspects of \rt.
Conversely, most of the papers in the review covering more than one \rt challenges (in most cases, \tcp and \tcs) do so by applying a common algorithm to both problems.
Thus, they do not strictly fall under the definition of Test Suite Orchestration, although they could still be considered as a component of an orchestration strategy along with other approaches.

Aside from our own publication introducing \fz \cite{greca_comparing_2022}, a notable example of test suite orchestration is from \citet{shi2015comparing}, which is not discussed in \Cref{chap:literature_review} due to its publication date of 2015, who compare empirically \tsr and \tcs.
The authors observe that both techniques aim at running only a subset of the test suite, thus asking which one is better considering the size of the reduced test suite, and the loss in terms of detection capability of change-related faults.
For \tcs their study adopts the \ek tool above mentioned, whereas for \tsr they remove redundant test cases using a greedy heuristic based on statement coverage.
From the comparison they conclude that \tcs on average returns a smaller test suite size, with no loss in change-related faults detection for safe \tcs techniques.
In comparison, \tsr can miss a small percentage of change-related faults. They also evaluate a combination of the two techniques, specifically ``selection of reduction'' in which \tcs is applied on the reduced suite obtained by \tsr: this can further reduce the number of tests yielding the same loss in fault detection observed for \tsr alone.

We were unable to detect further examples of publications fitting the mold of Test Suite Orchestration.
This signifies that addressing more than one challenge with the same technique is an existing — even if infrequent — concern among researchers, but combining distinct approaches together is not.

\subsection{Future Directions for Test Suite Orchestration}

As previously discussed, the experiments we performed with \fz is a starting point for test suite orchestration.
This work paves the way to exploring a full range of potential strategies of combining differing criteria for selection and prioritization. 
It can be worthwhile to also expand the study to the orchestration of techniques along other dimensions of regression testing, e.g., also test suite reduction (\tsr) or test suite amplification (\tsa).

When combining multiple techniques into a cohesive orchestration strategy, the first and perhaps most important aspect to consider is the sequence of operations.
We see in \Cref{sec:orch_fastazi} that there are two ways of using \tcs and \tcp together: we can either select a set of test cases and the prioritize these, or prioritize the entire test suite and run the selected tests in that given order.
Including more techniques in the orchestration inevitably leads to more possible sequences.

For example, if we add \tsr to the orchestration, the operation could be performed before or after the selection and prioritization.
By using it before, we already restrict the number of test cases the other techniques must deal with; doing it afterwards, the results of the reduction will only be used in the next execution of the test suite.

The combination becomes more interesting when adding \tsa to the strategy.
\tsa could be the first technique to run, updating or adding test cases that will then serve as input for selection, prioritization and reduction.
Or, it could be placed in between selection and prioritization, modifying the suite only according to the results of the selection.
This could be desired if the \tsa process is costly and running it with fewer targets greatly reduces the time it consumes.

Continuing this line of thought, \Cref{fig:orchestration} shows an example of a fully orchestrated test suite execution.
In it, we consider three subsequent versions of the \sut (\textbf{v\textsubscript{i-1}}, \textbf{v\textsubscript{i}}, \textbf{v\textsubscript{i+1}}).
The chevron boxes represent some process being applied to the tests, while the cut rectangles represent variations of the test suite (e.g., a list of test cases).

The target of the orchestration is \textbf{T}, which is the test suite corresponding to version \textbf{v\textsubscript{i}} of the \sut.
The first technique to be applied is \tcs, generating a subset of tests \textbf{S}.
Additionally, from previous test execution logs, historical data, such as test that have recently failed, can be extracted, forming the set \textbf{H}.

This subset is then used as input for \tsa techniques, in this example displayed separately as augmentation and amplification.
The results are one set of \textit{newly generated} tests \textbf{G} and one set \textbf{A} containing the amplified versions of the tests in \textbf{S}.
At this point, information from \textbf{H}, \textbf{G} and \textbf{A} is merged into a list of tests \textbf{M}.

\textbf{M} is then used as input for three different techniques.
On one side, \tsr is used, using information from \textbf{M} and \textbf{T} to eliminate excessive redundancies in the suite and produces a tighter suite \textbf{R} that can be used as a starting point for the next cycle of orchestration (when it is time for version \textbf{v\textsubscript{i+1}} of the \sut to be tested).
On the other, \tcp prioritizes the test cases to \textbf{P} and a test flakiness detection technique provides a list \textbf{F} of potentially unreliable tests, which should be handled differently during execution.

Finally, the orchestrated test suite \textbf{O} is produced, which can be used to test the \sut version \textbf{v\textsubscript{i}}.


\begin{figure*}[h]
  \centering
  \includegraphics[width=\linewidth]{figures/Orchestration-Compact.pdf}
  \begin{flushleft}
	\footnotesize Legend: 
	\textbf{v\textsubscript{i-1}}, \textbf{v\textsubscript{i}}, \textbf{v\textsubscript{i+1}}: previous, current and next version of the \sut; 
	\textbf{H}: output of history-based criteria;
	\textbf{T}: the test suite as of version \textbf{v\textsubscript{i}};
	\textbf{S}: the output of \tcs;
	\textbf{G}, \textbf{A}: the outputs of test suite augmentation and amplification, respectively;
	\textbf{M}: a \textit{selected and enhanced} test suite combining the outputs of the previous steps;
	\textbf{R}: the output of \tsr;
	\textbf{P}: the output of \tcp;
	\textbf{F}: the output of a flaky test detection technique;
	\textbf{O}: the \textit{orchestrated test suite} that should be executed for \textbf{v\textsubscript{i}}.
  \end{flushleft}
  \caption{Diagram showing an example of a fully orchestrated approach to the test suite execution and evolution.}
  \label{fig:orchestration}
\end{figure*}

It is worthy of reiteration that this is simply an example, built upon the goals of each \rt technique and considering how they can be used to each others' advantages.
Validating such a model requires extensive experimentation, which unfortunately poses a technical challenge, as not every technique has an available and easily usable implementation.
Even when the tools exist, the way each one handles inputs and outputs can be incompatible, so some alteration is needed.

Some questions remain unanswered regarding a fully orchestrated strategy.
The possibility of executing all \rt techniques at each new version of the \sut largely depends on the intervals between versions; if new versions are committed frequently, there might not be enough time to execute the full process.
In such cases, an additional point to consider is which techniques are important for frequent execution, and which ones can become part of a nightly testing cycle.

%\todo{Discuss possible ways of combining other \tcs and \tcp techniques along with \tsr and \tsa}