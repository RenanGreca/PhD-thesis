\begin{table}[!hbt]
\centering
\begin{tabular}{lrrcrrc}
\toprule
\multicolumn{1}{l}{\multirow{2}{*}{Approach}} &
  \multicolumn{3}{c}{TTFF} &
  \multicolumn{3}{c}{APFD} \\ %\cmidrule(l){2-7} 
\multicolumn{1}{c}{} &
  \multicolumn{1}{c}{\textit{Med}} &
  \multicolumn{1}{c}{\textit{SD}} &
  \multicolumn{1}{c}{\textit{Group}} &
  \multicolumn{1}{c}{\textit{Med}} &
  \multicolumn{1}{c}{\textit{SD}} &
  \multicolumn{1}{c}{\textit{Group}} \\ %\cmidrule(r){1-1}
\midrule
\fz      & 0.25 & 0.27 & (a) & 0.75 & 0.27 & (a) \\
%\fzp      & 0.26 & 0.27 & (a) & 0.74 & 0.27 & (a) \\
\ekr & 0.39 & 0.14 & (b) & 0.62 & 0.14 & (b) \\
\fs        & 0.41 & 0.29 & (c) & 0.60 & 0.29 & (c) \\
Random         & 0.49 & 0.09 & (d) & 0.51 & 0.09 & (d) \\ \bottomrule
\end{tabular}
\label{tab:ttff_apfd}
\begin{flushleft}
\footnotesize
%\scriptsize 
\textit{Med} is the median, \textit{SD} is the standard deviation, and \textit{Group} displays the result for the pairwise comparisons after the Kruskal-Wallis test. 
\end{flushleft}
\caption{TTFF and APFD for the different approaches.}
%\vspace{-3mm}
\end{table}
