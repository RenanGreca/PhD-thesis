\newcommand{\rowextraction}[2]{
#1 & % Citation alias
#2 % Full title
\\ }
\newcommand{\rowextractionheader}[2]{
\textbf{#1} & % Citation alias
#2 % Full title
\\ }

\begin{table}[]
\centering
\scriptsize
\rowcolors{1}{}{gray!10}
\begin{tabular}{lp{0.8\textwidth}}
\toprule
%\textbf{Data} & \textbf{Description}  \\ \midrule
\rowextractionheader{Bibliographical data}{Basic information about the publication.}
\midrule
\rowextraction{Date}{The date the paper was made available online.}
\rowextraction{Authors}{The list of authors.}
\rowextraction{Title}{The title of the paper.}
\rowextraction{Abstract}{The abstract of the paper.}
\rowextraction{Venue and Publisher}{The conference or journal where it was published and its organization.}
\rowextraction{DOI}{The Digital Object Identifier of the paper.}
\midrule
\rowextractionheader{Categorization}{Details regarding the problem addressed by the paper.}
\midrule
\rowextraction{\rt challenges}{Whether the paper covers \tcp, \tcs, \tsr, \tsa or a combination.}
\rowextraction{Context}{The type of software targeted by the approach.}
\midrule
\rowextractionheader{Applicability concerns}{Properties of the paper related to its \rea.}
\midrule
\rowextraction{Industry motivation}{Whether the paper is clearly motivated by an industrially relevant problem.}
\rowextraction{Industry evaluation}{Whether the technique is evaluated in industrial software or sufficiently large-scale open-source projects.}
\rowextraction{Experiment subject(s)}{Which software or kind of software was used for the experimental evaluation of the technique, including the testing scale, the availability and the language in which tests are written.}
\rowextraction{Industry partner}{Which, if any, industrial partner collaborated with the development and/or evaluation of the technique.}
\rowextraction{Industrial author}{Whether one or more of the authors of the paper come from industry.}
\rowextraction{Practitioner feedback}{Whether practitioners were consulted to provide feedback to the results of the paper.}
\rowextraction{Available tool}{Whether the technique introduced in the paper is available to be used, either as a prototype or as a complete tool. If true, we also stored the relevant URLs.}
\rowextraction{Put into practice}{Whether the proposed tool has been adopted into the development process of a certain software.}
\midrule
\rowextractionheader{Findings}{Details of the proposed technique and remaining challenges.}
\midrule
\rowextraction{Approach}{What sort of algorithm and information the technique is using.}
\rowextraction{Metrics}{What criteria are being used for evaluating the techniques.}
\rowextraction{Open challenges}{What the authors list as next steps and unsolved issues related to the problem they addressed.}
\bottomrule
\end{tabular}
\caption{Data extraction form.}
\label{table:extraction}
\end{table}
